% !TeX program = lualatex

\documentclass[11pt]{report}

\usepackage[T1]{fontenc}
\usepackage{polyglossia}
\usepackage{amsmath}
\usepackage{amssymb}
\usepackage{amsthm}
\usepackage[protrusion = true, expansion]{microtype}
\usepackage{braket}
\usepackage{mathtools}
\usepackage{mathrsfs}
\usepackage{cancel}
\usepackage{caption}
\usepackage{booktabs}
\usepackage{tabularx}
\usepackage{tikz, pgfplots}
\usepackage[output-decimal-marker={,}, exponent-product={\cdot}]{siunitx}
\usepackage[notrig]{physics}

\newcommand*{\numberset}{\mathbb}
\newcommand*{\N}{\numberset{N}}
\newcommand*{\R}{\numberset{R}}
\newcommand*{\Q}{\numberset{Q}}
\newcommand*{\Z}{\numberset{Z}}
\newcommand*{\C}{\numberset{C}}
\newcommand*{\K}{\numberset{K}}
\newcommand*{\T}{\numberset{T}}
\newcommand*{\E}{\numberset{E}}
\newcommand*{\cont}{\mathscr{C}}
\newcommand*{\Parti}{\mathcal{P}}
\newcommand*{\uno}{\mathbf{1}}
\newcommand*{\zero}{\mathbf{0}}
\newcommand*{\ee}{\mathrm{e}}
\newcommand*{\ii}{\mathrm{i}\mkern1mu}
\newcommand*{\tra}[1]{\ensuremath{#1^\mathrm{T}}}
\newcommand*{\her}[1]{\ensuremath{#1^\mathrm{H}}}
\newcommand*{\gammanum}[1]{\gamma_{#1}^{\textup{\textsc{num}}}}
%\newcommand*{\meg}[1]{\text{\textsc{#1}}}

%\renewcommand{\vec}{\overline}
\renewcommand{\P}{\numberset{P}}

\DeclareMathOperator{\sgn}{sgn}
\DeclareMathOperator{\D}{D}
\DeclareMathOperator{\Jacob}{J}
\DeclareMathOperator{\Hess}{H}
\DeclareMathOperator{\varmod}{mod}

\theoremstyle{plain}
\newtheorem{teorema}{Teorema}[chapter]
\newtheorem{proposizione}[teorema]{Proposizione}
\newtheorem{lemma}[teorema]{Lemma}
\newtheorem{corollario}[teorema]{Corollario}
\newtheorem{algoritmo}[teorema]{Algoritmo}

\theoremstyle{definition}
\newtheorem{definizione}[teorema]{Definizione}
\newtheorem{esempio}[teorema]{Esempio}
\newtheorem{esercizio}[teorema]{Esercizio}

\theoremstyle{remark}
\newtheorem*{nota}{Nota}
\newtheorem{osservazione}[teorema]{Osservazione}

\numberwithin{equation}{chapter}
\numberwithin{figure}{chapter}
\numberwithin{table}{chapter}

\captionsetup{tableposition=top,figureposition=bottom,font=small}

\usetikzlibrary{positioning, decorations.markings, decorations.pathreplacing}

\pgfplotsset{compat=1.17}

\setmainlanguage{italian}

\begin{document}
	\begin{titlepage}
		\begin{center}
			\begin{LARGE}
				\textsc{Riccardo Cazzin}
			\end{LARGE}
		\end{center}
		
		\vspace{4.5em}
		
		\begin{center}
			\begin{large}
				Appunti di
			\end{large}
			
			\vspace{1.8em}
			
			\begin{huge}
				\textsc{Analisi Numerica}
			\end{huge}
		\end{center}
		
		\vfill
		
		\begin{center}
			\begin{large}
				estratti dalle lezioni del
			\end{large}
			
			\vspace{1em}
			
			\begin{LARGE}
				\textsc{Prof.\ Alvise Sommariva}
			\end{LARGE}
		\end{center}
		
		\vfill
		
		\begin{center}
			\begin{large}
				\textsc{Versione 0.1}
			\end{large}
		\end{center}
		
		\vfill
		
		\noindent Anno Accademico 2020-2021 \hfill Università degli Studî di Padova
	\end{titlepage}
	
	\tableofcontents
	
	\chapter{Approssimazione in max-norma}

\section{Polinomi di miglior approssimazione}

	\noindent Dato \(n \in \N\), chiamiamo \(\P_n = \Braket{x^i : i \in \Set{0, \dots, n}}\). Da questa definizione si nota subito che \(\P_0 \subset \P_1 \subset \cdots\). Dato che le funzioni polinomiali sono continue, si ottiene
	\begin{equation*}
		\bigcup_{n \in \N} \P_n \subseteq \cont ([\, a, b \,])
	\end{equation*}
	per ogni intervallo chiuso e limitato \([\, a, b \,]\). In questa sezione tratteremo \(\cont ([\, a, b\,])\) come spazio normato dotato della \(\max\)-norma \(\norm{\cdot}_\infty\).
	
	\begin{definizione}\label{def:ins-denso}
		Dato uno spazio topologico \(X\), un insieme \(S \subseteq X\) si dice \emph{denso} in \(X\) se per ogni \(x \in X\) esiste una successione \((s_n)_{n \in \N} \subseteq S\) tale che \(s_n \to x\) per \(n \to \infty\), ovvero se \(\overline{S} = X\). In particolare, se \((X, \norm{\cdot})\) è uno spazio normato, un insieme \(S \subseteq X\) si dice \emph{denso} in \(X\) se per ogni \(x \in X\) e per ogni \(\varepsilon > 0\) esiste \(s \in S\) tale che \(\norm{x - s} < \varepsilon\).
	\end{definizione}

	\begin{teorema}\label{th:errore-ins-denso}
		Dati uno spazio normato \((X, \norm{\cdot})\) e una successione di insiemi \((S_n)_{n \in \N} \subseteq \Parti (X)\) con \(S_0 \ne \varnothing\) e \(S_i \subset S_{i + 1}\) per ogni \(i \in \N\), si definisca
		\begin{equation}\label{eq:errore-approx}
			E_n (f) \coloneqq \inf_{\mathclap{p_n \in S_n}} \norm{p_n - f}
		\end{equation}
		per ogni \(f \in X\). Per qualsiasi \(f \in X\) si verifica \(\lim_{n \to \infty} E_n (f) = 0\) se e solo se \(\bigcup_{n \in \N} S_n\) è denso in \(X\).
	\end{teorema}

	\begin{proof}
		Mostriamo entrambe le implicazioni.
		\begin{description}
			\item[(\(\Rightarrow\))] Siano fissati \(f \in X\) e \(\varepsilon > 0\). Per ipotesi esiste \(n \in \N\) tale che, se \(E_n (f) < \varepsilon\), allora esiste \(p_n \in S_n\) tale che \(\norm{p_n - f} \le \varepsilon\) -- questo è giustificato dalle proprietà dell'estremo inferiore. Da ciò segue che \(\bigcup_{n \in \N} S_n\) è denso in \(X\) per definizione.
			\item[(\(\Leftarrow\))] Fissato \(f \in X\), la successione \(\qty(E_n (f))_{n \in \N}\) è monotona decrescente non negativa, quindi ammette limite; per ogni \(\varepsilon > 0\), quindi, esiste \(p \in \bigcup_{n \in \N} S_n\) tale che \(\norm{p - f} \le \varepsilon\). Scelto \(\bar{n}\) tale che \(p \in S_{\bar{n}}\), si ha \(E_n (f) \le \varepsilon\) per ogni \(n \ge \bar{n}\): per questo motivo \(\lim_{n \to \infty} E_n (f) \le \varepsilon\) e, data l'arbitrarietà di \(\varepsilon\), si conclude che \(\lim_{n \to \infty} E_n (f) = 0\). \qedhere
		\end{description}
	\end{proof}

	\begin{osservazione}
		In base al Teorema~\ref{th:errore-ins-denso}, per ogni \(n \in \N^*\) l'insieme \(A_n = \Set{\norm{p_n - f} : p_n \in S_n} \subseteq \R\) è non vuoto e limitato inferiormente da \(0\), quindi ammette estremo inferiore: la definizione nella \eqref{eq:errore-approx}, dunque, è ben posta.
	\end{osservazione}

	Un caso particolare del Teorema~\ref{th:errore-ins-denso} si ha coi polinomi nei confronti delle funzioni continue su un compatto di \(\R\).

	\begin{teorema}[di approssimazione di Weierstrass]\label{th:weierstrass-approx}
		Ogni funzione di \(\cont ([\, a, b \,])\) con \(a, b \in \R\) e \(a \le b\) è limite uniforme di una successione di polinomi.
	\end{teorema}

	Vogliamo mostrare che, sotto certe condizioni, esiste un elemento di miglior approssimazione -- ovvero l'estremo inferiore è in realtà un minimo.
	
	\begin{lemma}\label{lem:distanza-funzione-continua}
		Dati uno spazio normato \((X, \norm{\cdot})\) e un suo elemento \(f \in X\), se \(S \subseteq X\) è aperto, allora la funzione \(d (f, \cdot) = \norm{f - \cdot}\) è continua in \(S\).
	\end{lemma}

	\begin{proof}
		Osserviamo innanzitutto che, scelti \(x, y \in X\) qualunque, si verifica che \(\abs{\norm{x} - \norm{y}} \le \norm{x - y}\): supponendo, infatti, \(\norm{x} \ge \norm{y}\), si ha
		\begin{equation*}
			\norm{x} = \norm{x - y + y} \le \norm{x - y} + \norm{y} \iff 0 \le \norm{x} - \norm{y} \le \norm{x - y}
		\end{equation*}
		e, scambiando i ruoli di \(x\) e \(y\), si è provata l'osservazione.
		
		Fissato \(\varepsilon > 0\), poniamo \(\delta = \varepsilon\) e scegliamo \(x, y \in S\) tali che \(\norm{x - y} \le \delta\): si ha
		\begin{multline*}
				\abs{d (f, x) - d (f, y)} = \abs{\norm{f - x} - \norm{f - y}} \\
				\le \norm{(f - x) - (f - y)} = \norm{x - y} \le \delta = \varepsilon
		\end{multline*}
		il che permette di concludere.
	\end{proof}

	\begin{teorema}\label{th:miglior-approx-esiste}
		Dato uno spazio normato \((X, \norm{\cdot})\), sia \(f \in X\). Se un sottospazio vettoriale \(S \le X\) è di dimensione finita, allora esiste \(s^* \in S\) tale che
		\begin{equation*}
			\norm{f - s^*} = \min_{s \in S} \norm{f - s}
		\end{equation*}
	\end{teorema}

	\begin{proof}
		Visto che \(0_X \in S\) per ogni \(S \le X\), si ha
		\begin{equation*}
			E_n (f) = \inf_{p \in S} \norm{f - p} \le \norm{f - 0_X} = \norm{f}
		\end{equation*}
		ovvero che \(E_n (f) \in [\, 0, \norm{f} \,]\). Poiché per il Lemma~\ref{lem:distanza-funzione-continua} la funzione \(d (f, \cdot)\) è continua su \(S\) e dato che \(B_{\norm{f}} (f) \cap S\) è un insieme compatto di \(S\), per il teorema di Weierstrass \(d (f, \cdot)\) ammette minimo in tale insieme, ovvero esiste \(s^* \in B_{\norm{f}} (f) \cap S \subseteq S\) tale che \(E_n (f) = \norm{f - s^*}\).
	\end{proof}

	Limitando questo risultato al caso delle funzioni continue su un compatto di \(\R\), è provata l'esistenza di un polinomio di miglior approssimazione.

	\begin{corollario}
		Per ogni \(k \in \N\) e per ogni \(f \in \cont ([\, a, b \,])\) con \(a, b \in \R\) e \(a \le b\) esiste un polinomio \(p_k^* \in \P_n\) di miglior approssimazione relativamente alla \(\max\)-norma \(\norm{\cdot}_{\infty}\).
	\end{corollario}

	 Quanto visto non dimostra l'unicità dell'elemento di miglior approssimazione. Nel caso delle funzioni continue su un compatto di \(\R\), tuttavia, vale il seguente risultato, che riportiamo senza dimostrazione.
	 
	 \begin{teorema}[di equioscillazione di Chebyshev]\label{th:chebyshev-equiosc}
	 	Dato un intervallo chiuso e limitato \([\, a, b \,]\), per ogni \(f \in \cont ([\, a, b \,])\) e per ogni \(n \in \N\) esiste un unico \(p_n^* \in \P_n\) di miglior approssimazione. Esistono, inoltre, \(\sigma \in \Set{- 1, 1}\) e \(n + 2\) punti \(a \le x_0 < \cdots < x_{n + 1} \le b\) tali che per ogni \(j \in \Set{0, \dots, n + 1}\) si abbia
	 	\begin{equation}
	 		f (x_j) - p_n^* (x_j) = \sigma (-1)^j \norm{f - p_n^*}_{\infty}
	 	\end{equation}
	 \end{teorema}
 
 	\begin{figure}[tpb]
 		\centering
 		
 		\begin{tikzpicture}
 			\begin{axis}[scale = 1.3, samples = 400, axis lines = center, domain=0:10, ymin = -1.5, ymax = 1.5, xmax = 10.2, no marks, unit vector ratio = 1 2, legend entries = {\(\sin x\), \(p_5^* (x)\)}, legend pos = outer north east, xtick = {0, 2, ..., 10}]
 				\addplot {sin(deg(x))};
 				\addplot+[black, dashed, thin, forget plot] {sin(deg(x)) + 0.25945059765721};
 				\addplot+[black, dashed, thin, forget plot] {sin(deg(x)) - 0.25945059765721};
 				\addplot+[red] table {risorse/sin.dat};
 			\end{axis}
 		\end{tikzpicture}
 	
 		\caption{Confronto tra la funzione \(\sin x\) e il suo polinomio di miglior approssimazione di quinto grado nell'intervallo \([\, 0, 10 \,]\). Si notino le sette intersezioni di \(p_5^* (x)\) con \(\sin (x) \pm E_5 (\sin)\).}\label{fig:sin-migl-approx}
 	\end{figure}
 
 	Benché ne siano garantite l'esistenza e l'unicità, il polinomio di miglior approssimazione richiede un algoritmo abbastanza complicato, messo a punto da Remez, per essere trovato. Come è possibile notare nella Tabella~\ref{tab:remez-errore}, funzioni con regolarità diverse richiedono un grado diverso del polinomio di miglior interpolazione per essere approssimate con un certo errore assoluto in \(\max\)-norma. I teoremi di Jackson danno un'idea piú rigorosa dell'influenza della regolarità delle funzioni su tale errore.
 	
 	\begin{table}[tpb]
 		\centering
 		
 		\caption{Errore in \(\max\)-norma commesso dal polinomio di miglior interpolazione di alcune funzioni definite su \([\, -5, 5 \,]\), al variare del grado.}\label{tab:remez-errore}
 		
 		\begin{tabular}{SSSS}
 			\toprule
 			{\(n\)} & {\(\displaystyle E_n \qty(\frac{1}{1 + x^2})\)} & {\(E_n (\abs{x - 4})\)} & {\(E_n (\sin x)\)} \\
 			\midrule
			\num{5}   & \num{2.171584e-01} & \num{1.612220e-01} & \num{1.078946e-01} \\
			\num{10}  & \num{6.592293e-02} & \num{8.398083e-02} & \num{7.031736e-04} \\
			\num{15}  & \num{2.977669e-02} & \num{5.677524e-02} & \num{2.306162e-08} \\
			\num{20}  & \num{9.039331e-03} & \num{4.279954e-02} & \num{6.690029e-12} \\
			\num{25}  & \num{4.082970e-03} & \num{3.431831e-02} & \num{1.355775e-15} \\
			\num{30}  & \num{1.239470e-03} & \num{2.863059e-02} & \num{\le e-16} \\
			\num{35}  & \num{5.598556e-04} & \num{2.455346e-02} & \num{\le e-16} \\
			\num{40}  & \num{1.699558e-04} & \num{2.148833e-02} & \num{\le e-16} \\
			\num{45}  & \num{7.676723e-05} & \num{1.910021e-02} & \num{\le e-16} \\
			\num{50}  & \num{2.330428e-05} & \num{1.718714e-02} & \num{\le e-16} \\
			\num{55}  & \num{1.052630e-05} & \num{1.562016e-02} & \num{\le e-16} \\
			\num{60}  & \num{3.195476e-06} & \num{1.431308e-02} & \num{\le e-16} \\
			\num{65}  & \num{1.443363e-06} & \num{1.320615e-02} & \num{\le e-16} \\
			\num{70}  & \num{4.381627e-07} & \num{1.225662e-02} & \num{\le e-16} \\
			\num{75}  & \num{1.979134e-07} & \num{1.143309e-02} & \num{\le e-16} \\
			\num{80}  & \num{6.008073e-08} & \num{1.071203e-02} & \num{\le e-16} \\
			\num{85}  & \num{2.713784e-08} & \num{1.007540e-02} & \num{\le e-16} \\
			\num{90}  & \num{8.238251e-09} & \num{9.509177e-03} & \num{\le e-16} \\
			\num{95}  & \num{3.721132e-09} & \num{9.002281e-03} & \num{\le e-16} \\
			\num{100} & \num{1.129627e-09} & \num{8.545846e-03} & \num{\le e-16} \\
			\bottomrule
		\end{tabular}
	\end{table}

 	\begin{definizione}
 		Di una funzione \(f \colon [\, a, b \,] \to \R\) si dice \emph{modulo di continuità} la quantità
		\begin{equation}\label{eq:modulo-contin}
 			\omega (f, \delta) = \sup_{\mathclap{\substack{x, y \in [\, a, b \,] \\ \abs{x - y} \le \delta}}} \abs{f (x) - f (y)}
 		\end{equation}
	\end{definizione}

	La misura di continuità di una funzione è tanto maggiore quanto piú la funzione “oscilla”. Se la funzione in esame ha una qualche regolarità, la sua misura di continuità può essere stimata in modo semplice.
	
	\begin{esempio}[Misura di continuità per funzioni particolari]
		Se una funzione \(f \colon [\, a, b \,] \to \R\) è lipschitziana con costante di Lipschitz \(L\), allora
		\begin{equation*}
			\omega (f, \delta) = \sup_{\mathclap{\substack{x, y \in [\, a, b \,] \\ \abs{x - y} \le \delta}}} \abs{f (x) - f (y)} \le \sup_{\mathclap{\substack{x, y \in [\, a, b \,] \\ \abs{x - y} \le \delta}}} L \abs{x - y} = L \delta
		\end{equation*}
		Se \(f \in \cont^1 ([\, a, b\,])\), allora è anche lipschitziana ed ha costante di Lipschitz pari a \(L = \max_{x \in [\, a, b \,]} \abs{f' (x)}\).
		
		Se \(f\) è h\"olderiana con costante di H\"older \(\alpha \in (\, 0, 1 \,)\), ovvero esiste \(L \ge 0\) tale che \(\abs{f (x) - f (y)} \le L \abs{x - y}^\alpha\) per ogni \(x, y \in [\, a, b \,]\), allora
		\begin{equation*}
			\omega (f, \delta) = \sup_{\mathclap{\substack{x, y \in [\, a, b \,] \\ \abs{x - y} \le \delta}}} \abs{f (x) - f (y)} \le \sup_{\mathclap{\substack{x, y \in [\, a, b \,] \\ \abs{x - y} \le \delta}}} L \abs{x - y}^\alpha = L \delta^\alpha
		\end{equation*}
	\end{esempio}

	\begin{teorema}[Jackson]\label{th:jackson}
		Per ogni \(n \in \N^*\) e per ogni \(f \in \cont ([\, a, b \,])\) esiste una costante \(M \ge 0\) indipendente da \(a, b, n\) tale che
		\begin{equation}\label{eq:jackson}
			E_n (f) = \min_{p \in \P_n} \norm{f - p}_\infty \le M \omega \qty(f, \frac{b - a}{n})
		\end{equation}
	\end{teorema}

	\begin{corollario}\label{cor:jackson-lip-hold}
		Se alle ipotesi del Teorema~\ref{th:jackson} si aggiunge che \(f\) è lipschitziana, allora esiste \(M^*\) indipendente da \(a, b, n\) tale che
		\begin{equation}\label{eq:jackson-lip}
			E_n (f) \le M^* \frac{b - a}{n}
		\end{equation}
		Se, invece, si suppone che \(f\) sia h\"olderiana con costante di H\"older \(\alpha\), allora esiste \(\bar{M}\) indipendente da \(a, b, n\) tale che
		\begin{equation}\label{eq:jackson-hold}
			E_n (f) \le \bar{M} \qty(\frac{b - a}{n})^\alpha
		\end{equation}
	\end{corollario}

	\begin{teorema}[Jackson]\label{th:jackson-ck}
		Per ogni funzione \(f \in \cont^k ([\, a, b \,])\) con \(k \in \N\) e per ogni \(n > k\) esiste \(M \ge 0\) tale che
		\begin{equation}\label{eq:jackson-ck}
			E_n (f) \le M^{k + 1} \frac{(b - a)^k}{\prod_{i = 0}^{k - 1} (n - i)} \, \omega \qty(f^{(k)}, \frac{b - a}{n - k})
		\end{equation}
	\end{teorema}

	\begin{corollario}\label{cor:jackson-ck-hold}
		Se alle ipotesi del Teorema~\ref{th:jackson-ck} si aggiunge che \(f\) è h\"olderiana con costante di H\"older \(\alpha\) e costante \(L\) e che \(k > 0\), allora esiste \(M \ge 0\) tale che
		\begin{equation}\label{eq:jackson-ck-hold}
			E_n (f) \le L M^{k + 1} \frac{(b - a)^k}{\prod_{i = 0}^{k - 1} (n - i)} \qty(\frac{b - a}{n - k})^\alpha
		\end{equation}
	\end{corollario}

	\begin{teorema}[Jackson]\label{th:jackson-hold-d}
		Se una funzione \(f \in \cont^k ([\, a, b \,])\) è \(\alpha\)-h\"olderiana di costante \(M\), allora esiste una costante \(d_k\) indipendente da \(f\) e da \(n \in \N^*\) tale che
		\begin{equation}\label{eq:jackson-hold-d}
			E_n (f) \le \frac{M d_k}{n^{k + \alpha}}
		\end{equation}
	\end{teorema}

	Ricordiamo che una funzione complessa \(f \colon \varOmega \to \C\), con \(\varOmega\) regione del piano complesso, si dice \emph{analitica} in un punto \(z_0 \in \varOmega\) se esiste un intorno di \(z_0\) in \(\varOmega\) tale che per ogni \(z\) in tale intorno si abbia \(f (z) = \sum_{n = 0}^\infty a_n (z - z_0)^n\); tale funzione si dice \emph{analitica in \(\varOmega\)} se è analitica in ogni punto di \(\varOmega\).
	
	\begin{teorema}\label{th:errore-approx-analitica}
		Se una funzione \(f \colon [\, a, b \,] \to \R\) è analitica in un aperto \(\varOmega \subseteq \C\) che contenga \([\, a, b \,]\), allora esiste \(\vartheta \in (\, 0, 1 \,)\) tale che
		\begin{equation}\label{eq:errore-approx-analitica}
			E_n (f) = \order{\vartheta^n}
		\end{equation}
	\end{teorema}

	Se una funzione è analitica su \(\C\), ovvero se è \emph{intera}, l'errore commesso approssimando tale funzione col suo polinomio di miglior approssimazione decade piú che esponenzialmente.
	
	\begin{teorema}[Bernstein]\label{th:bernstein}
		Data una funzione \(f \colon [\, a, b \,] \to \R\), si ha
		\begin{equation}\label{eq:bernstein}
			\lim_{n \to \infty} \sqrt[n]{E_n (f)} = 0 \iff \text{\(f\) intera}
		\end{equation}
	\end{teorema}

	\begin{osservazione}
		Per quanto visto, la funzione di Runge \(f (x) \coloneqq \frac{1}{1 + x^2}\) è analitica in un aperto di \(\C\) che contenga \([\, -5, 5 \,]\). Una verifica sperimentale coi dati nella Tabella~\ref{tab:remez-errore} mostra che \(\vartheta \approx \num{0.814}\). Poiché, invece, la funzione \(\sin z\) è intera, l'errore di approssimazione con l'algoritmo di Remez scende alla precisione di macchina già per \(n\) non troppo alti.
		
		Per quanto riguarda la funzione \(f (x) \coloneqq \abs{x - 4}\), che è lipschitziana, dal Corollario~\ref{cor:jackson-lip-hold} segue che esiste \(M\) tale che \(E_n (f) = 10 M / n = \order{1 / n}\); con una verifica sperimentale si vede che \(E_n (f) \approx \num{0.85} / n\), ovvero che la convergenza del polinomio di miglior approssimazione è molto lenta.
	\end{osservazione}

\section{Polinomi di Chebyshev}
	
	\noindent Dato \(n \in \N\), consideriamo per \(x \in [\, -1, 1 \,]\) la funzione
	\begin{equation}\label{eq:polin-cheb-def}
		T_n (x) = \cos (n \arccos x)
	\end{equation}
	\emph{A priori} tale funzione può non essere un polinomio, ma si nota che
	\begin{gather*}
		T_0 (x) = \cos (0 \cdot \arccos x) = 1 \\
		T_1 (x) = \cos (1 \arccos x) = x
	\end{gather*}
	Dalle formule trigonometriche di addizione e sottrazione
	\begin{gather*}
		\cos ((n + 1) \vartheta) = \cos (n \vartheta) \cos \vartheta - \sin (n \vartheta) \sin \vartheta \\
		\cos ((n - 1) \vartheta) = \cos (n \vartheta) \cos \vartheta + \sin (n \vartheta) \sin \vartheta \\
	\end{gather*}
	si ottiene, sommando membro a membro,
	\begin{equation*}
		\cos ((n + 1) \vartheta) + \cos ((n - 1) \vartheta) = 2 \cos(n \vartheta) \cos \vartheta
	\end{equation*}
	Se ora si pone \(\vartheta = \arccos x\), si trova
	\begin{equation}\label{eq:polin-cheb-formula}
		T_{n + 1} (x) = 2 x \, T_n (x) - T_{n - 1} (x)
	\end{equation}
	e, dato che per \(n \in \Set{0, 1}\), i polinomi di Chebyshev corrispondenti sono effettivamente polinomi in senso classico, per ricorrenza \(\Set{T_n (x) : n \in \N}\) è una successione di polinomi, ove \(T_n (x)\) è di grado \(n\). Se \(n > 0\), il coefficiente del termine \(x^n\) è \(2^{n - 1}\).
	
	Troviamo ora gli zeri di un polinomio di Chebyshev \(T_n (x)\): essi sono i punti \(x_k\) tali che \(\cos (n \arccos x_k) = 0\); ricordando che il dominio della funzione \(\arccos\) è \([\, 0, \pi \,]\), gli zeri soddisfano
	\begin{equation*}
		n \arccos x_k = \frac{\pi}{2} + k \pi = \frac{(2 k + 1) \pi}{2}
	\end{equation*}
	ovvero
	\begin{equation*}
		\arccos x_k = \frac{(2 k + 1) \pi}{2 n}
	\end{equation*}
	e, applicando la funzione coseno ad ambo i membri, si ottiene per ogni \(k \in \Set{0, \dots, n - 1}\)
	\begin{equation}\label{eq:polin-cheb-zeri}
		x_k = \cos \qty(\frac{(2 k + 1) \pi}{2 n})
	\end{equation}
	Notiamo che gli zeri del polinomio di Chebyshev di grado \(n\) sono \(n\) punti distinti dell'intervallo aperto \((\, -1, 1 \,)\).
	
\section{Costanti di Lebesgue}
	
	\noindent Dati un intervallo chiuso e limitato \([\, a, b \,]\) e una funzione \(f \in \cont ([\, a, b \,])\), si consideri il polinomio \(p_n \in \P_n\) che interpola le \(n + 1\) coppie a due a due distinte \((x_k, f_k)\), con \(f_k = f (x_k)\) e \(k \in \Set{0, \dots, n}\). Definito per ogni \(k \in \Set{0, \dots, n}\) il \(k\)-esimo \emph{polinomio di Lagrange}
	\begin{equation}\label{eq:polin-lagrange}
		L_k (x) = \prod_{\mathclap{\substack{j = 0 \\ j \ne k}}}^n \frac{x - x_j}{x_k - x_j}
	\end{equation}
	è noto che
	\begin{equation}\label{eq:polin-interp}
		p_n (x) = \sum_{k = 0}^n f_k L_k (x)
	\end{equation}
	Se i valori \(f_k\) sono sostituiti con valori perturbati \(\tilde{f}_k\), allora, posto \(\tilde{p}_n (x) = \sum_{k = 0}^n \tilde{f}_k L_k (x)\), si ha
	\begin{multline*}
		\abs{p_n (x)  - \tilde{p}_n (x)} = \abs{\sum_{k = 0}^n (f_k - \tilde{f}_k) L_k (x)} \le \sum_{k = 0}^n \abs{f_k - \tilde{f}_k} \, \abs{L_k (x)} \\
		\le \qty(\max_{k \in \Set{0, \dots, n}} \abs{f_k - \tilde{f}_k}) \sum_{k = 0}^n \abs{L_k (x)}
	\end{multline*}
	da cui segue che
	\begin{equation*}
		\max_{x \in [\, a, b \,]} \abs{p_n (x)  - \tilde{p}_n (x)} \le \qty(\max_{k \in \Set{0, \dots, n}} \abs{f_k - \tilde{f}_k}) \max_{x \in [\, a, b \,]} \sum_{k = 0}^n \abs{L_k (x)}
	\end{equation*}
	Definita la quantità
	\begin{equation}\label{eq:cost-lebesgue}
		\varLambda_n \coloneqq \max_{x \in [\, a, b \,]} \sum_{k = 0}^n \abs{L_k (x)}
	\end{equation}
	si ricava la stima
	\begin{equation}\label{eq:errore-cost-lebesgue}
		\norm{p_n - \tilde{p}_n}_\infty \le \qty(\max_{k \in \Set{0, \dots, n}} \abs{f_k - \tilde{f}_k}) \varLambda_n
	\end{equation}

	Osserviamo che \(\varLambda_n\), detta \emph{costante di Lebesgue} dell'insieme di punti \(x_0, \dots, x_n\), dipende esclusivamente dai polinomi di Lagrange e, quindi, dai soli punti di interpolazione. È chiaro che ciò rende \(\varLambda_n\) un indice di stabilità dell'interpolazione di Lagrange.
	
	Si può mostrare che, se \(\mathcal{L}_n\) è l'operatore lineare e limitato che associa ad ogni funzione \(f \in \cont([\, a, b \,])\) il suo polinomio interpolatore \(p_n\) nei punti \(x_0, \dots, x_n\), allora
	\begin{equation}\label{eq:norma-interpolazione-lebesgue}
		\varLambda_n = \norm{\mathcal{L}_n}_\infty = \max_{\substack{g \in \cont ([\, a, b \,]) \\ g \ne 0}} \frac{\norm{\mathcal{L}_n (g)}_\infty}{\norm{g}_\infty}
	\end{equation}
	
	\begin{teorema}\label{th:errore-interpolaz-appross}
		Se \(p_n \in \P_n\) è il polinomio interpolatore di una funzione \(f \in \cont([\, a, b \,])\) relativo ai punti \(x_0, \dots, x_n\), allora
		\begin{equation}\label{eq:errore-interpolaz-appross}
			\norm{f - p_n}_\infty \le (1 + \varLambda_n) E_n (f)
		\end{equation}
	\end{teorema}

	\begin{proof}
		Se \(f \in \P_n\), allora \(f = p_n = p_n^*\) e l'asserto è banalmente verificato. Supponiamo ora \(f \notin \P_n\); si ha, quindi, \(f - q_n \ne 0\) per ogni \(q_n \in \P_n\). Dal momento che \(q_n \in \P_n\), si ha \(\mathcal{L}_n (q_n) = q_n\) per l'unicità del polinomio interpolatore e per il principio d'identità dei polinomi. Per la linearità di \(\mathcal{L}_n\), poi, si ha
		\begin{equation*}
			\mathcal{L}_n (f - q_n) = \mathcal{L}_n (f) - \mathcal{L}_n (q_n) = p_n - q_n
		\end{equation*}
		Da ciò e dalla \eqref{eq:norma-interpolazione-lebesgue} segue che
		\begin{equation*}
			\varLambda_n = \max_{\substack{g \in \cont ([\, a, b \,]) \\ g \ne 0}} \frac{\norm{\mathcal{L}_n (g)}_\infty}{\norm{g}_\infty} \ge \frac{\norm{\mathcal{L}_n (f - q_n)}_\infty}{\norm{f - q_n}_\infty} = \frac{\norm{p_n - q_n}_\infty}{\norm{f - q_n}_\infty}
		\end{equation*}
		e, quindi, per ogni \(q_n \in \P_n\) vale \(\norm{p_n - q_n}_\infty \le \varLambda_n \norm{f - q_n}_\infty\). Applicando la diseguaglianza triangolare, si trova
		\begin{multline*}
			\norm{f - p_n}_\infty = \norm{(f - q_n) + (q_n - p_n)}_\infty \le \norm{f - q_n}_\infty + \norm{q_n - p_n}_\infty \\
			\le \norm{f - q_n}_\infty + \varLambda_n \norm{f - q_n}_\infty = (1 + \varLambda_n) \norm{f - q_n}_\infty
		\end{multline*}
		da cui segue la \eqref{eq:errore-interpolaz-appross} scegliendo \(q_n = p_n^*\).
	\end{proof}

	Una conseguenza “informale” del Teorema~\ref{th:errore-interpolaz-appross} è che approssimare una funzione col suo polinomio interpolatore comporta un errore simile a quello compiuto approssimando col polinomio di miglior approssimazione se \(\varLambda_n\) è abbastanza piccola.
	
	\begin{esempio}
		Di alcune suddivisioni dell'intervallo \([\, -1, 1 \,]\) conosciamo il comportamento asintotico di \(\varLambda_n\) al crescere di \(n\). Se si usano \(n + 1\) punti equispaziati, si mostra che
		\begin{equation*}
			\varLambda_n \sim \frac{2^{n + 1}}{\ee n \log n}
		\end{equation*}
		Se invece si usano i punti di Chebyshev, di forma \(x_k = \cos \qty(\frac{2 k - 1}{2 (n + 1)} \pi)\) con \(k \in \Set{1, \dots, n + 1}\), si trova l'andamento asintotico
		\begin{equation*}
			\varLambda_n = \frac{2}{\pi} \qty[\log (n + 1) + \gamma + \log \qty(\frac{8}{\pi})] + \order{\frac{1}{(n + 1)^2}}
		\end{equation*}
		ove \(\gamma \approx \num{0.5772156649}\) è la costante di Eulero-Mascheroni.
		
		Se si usano i punti di Chebyshev estesi, di forma \(x_k = \frac{\cos \qty(\frac{2 k - 1}{2 (n + 1)} \pi)}{\cos \qty(\frac{1}{2 (n + 1)} \pi)}\) con \(k \in \Set{1, \dots, n + 1}\), si trova l'andamento asintotico
		\begin{equation*}
			\varLambda_n = \frac{2}{\pi} \qty[\log (n + 1) + \gamma + \log \qty(\frac{8}{\pi}) - \frac{2}{3}] + \order{\frac{1}{\log(n + 1)}}
		\end{equation*}
		Si può dimostrare, poi, che il minimo andamento asintotico per la costante di Lebesgue deve valere
		\begin{equation*}
			\varLambda_n = \frac{2}{\pi} \qty[\log (n + 1) + \gamma + \log \qty(\frac{8}{\pi})] + \order{\frac{\log \log (n + 1)}{\log (n + 1)}}
		\end{equation*}
	\end{esempio}

	\begin{figure}[tpb]
		\centering
		
		\begin{tikzpicture}
			\begin{axis}[ymode=log, domain=5:50, only marks, legend entries = {\(\varLambda_n^{\textup{equi}}\), \(\varLambda_n^{\textup{Cheb}}\)}, legend pos = outer north east]
				\addplot table {risorse/lebequi.dat};
				\addplot table {risorse/lebcheb.dat};
			\end{axis}
		\end{tikzpicture}
		
		\caption{Confronto tra gli andamenti asintotici di \(\varLambda_n\) usando punti equispaziati e punti di Chebyshev.}\label{fig:lebesgue-equisp-cheb}
	\end{figure}

	
	\chapter{Approssimazione in spazi euclidei}

\section{Richiami sugli spazi euclidei}

	\begin{definizione}[Spazio euclideo reale]
		Si dice \emph{spazio euclideo reale} uno spazio vettoriale \(\E\) su \(\R\) dotato di un'applicazione \((\cdot, \cdot) \colon \E \times \E \to \R\) tale che
		\begin{subequations}
			\begin{gather}
				\forall x \in \E \colon (x, x) \ge 0 \\
				\forall x \in \E \colon (x, x) = 0 \iff x = 0_\E \\
				\forall x, y \in \E \colon (x, y) = (y, x) \\
				\forall \lambda \in \R \colon \forall x, y \in \E \colon (\lambda x, y) = \lambda (x, y) \\
				\forall x, y, z \in \E \colon (x + y, z) = (x, z) + (y, z)
			\end{gather}
		\end{subequations}
	\end{definizione}
	
	\begin{definizione}[Spazio euclideo complesso]
		Si dice \emph{spazio euclideo complesso} uno spazio vettoriale \(\E\) su \(\C\) dotato di un'applicazione \((\cdot, \cdot) \colon \E \times \E \to \C\) tale che
		\begin{subequations}
			\begin{gather}
				\forall x \in \E \colon (x, x) \ge 0 \\
				\forall x \in \E \colon (x, x) = 0 \iff x = 0_\E \\
				\forall x, y \in \E \colon (x, y) = \overline{(y, x)} \\
				\forall \lambda \in \C \colon \forall x, y \in \E \colon (\lambda x, y) = \lambda (x, y) \\
				\forall \lambda \in \C \colon \forall x, y \in \E \colon (x, \lambda y) = \overline{\lambda} (x, y) \\
				\forall x, y, z \in \E \colon (x, y + z) = (x, y) + (x, z)
			\end{gather}
		\end{subequations}
	\end{definizione}

	A partire da queste due definizioni si può definire lo spazio normato \((\E, \norm{\cdot})\) dotato della norma \(\norm{f}_2 = \sqrt{(f, f)}\).
	
	\begin{osservazione}
		Si nota facilmente che
		\begin{gather*}
			\qty(x, \sum_{k = 1}^n y_k) = \sum_{k = 1}^n (x, y_k) \\
			\qty(\sum_{k = 1}^n x_k, y) = \overline{\qty(y, \sum_{k = 1}^n x_k)} = \overline{\sum_{k = 1}^n (y, x_k)} = \sum_{k = 1}^n \overline{(y, x_k)} = \sum_{k = 1}^n (x_k, y)
		\end{gather*}
	\end{osservazione}	

	\begin{esempio}
		\(\R^n\) dotato del prodotto scalare usuale è uno spazio euclideo; se \(e_1, \dots, e_n\) è una sua base ortonormale, ovvero si verifica \((e_j, e_k) = \delta_{j, k}\) per ogni \(j, k \in \Set{1, \dots, n}\), allora per ogni \(x \in \R^n\) vale la scrittura unica \(x = \sum_{k = 1}^n c_k e_k\), con \(c_k = (x, e_k)\).
		
		Lo spazio \(\cont ([\, a, b \,])\) delle funzioni continue su un compatto \([\, a, b \,]\) dotato del prodotto scalare \((f, g) = \int_a^b f (x) g (x) \dd{x}\) è uno spazio euclideo.
		
		Lo spazio \(L_\R^2 ([\, a, b \,])\) delle funzioni reali misurabili su un compatto \([\, a, b \,]\) e di modulo al quadrato integrabile dotato del prodotto scalare \((f, g) = \int_a^b f (x) g (x) \dd{x}\) è uno spazio euclideo completo, ovvero tale che ogni successione di Cauchy è convergente.
		
		Lo spazio \(L_\C^2 ([\, a, b \,])\) delle funzioni complesse misurabili su un compatto \([\, a, b \,]\) e di modulo al quadrato integrabile dotato del prodotto scalare \((f, g) = \int_a^b f (x) \overline{g (x)} \dd{x}\) è uno spazio euclideo completo.
	\end{esempio}
	
	\begin{teorema}[Pitagora]\label{th:pitagora}
		Dato uno spazio euclideo \(\E\), se \(f, g \in \E\) verificano \((f, g) = 0\), allora \(\norm{f + g}_2^2 = \norm{f}_2^2 + \norm{g}_2^2\).
	\end{teorema}
	
	\begin{proof}
		Con un conto diretto si vede che
		\begin{equation*}
			\begin{split}
				\norm{f + g}_2^2 = (f + g, f + g) &= (f, f) + (f, g) + (g, f) + (g, g) \\
				&= \norm{f}_2^2 + 0 + 0 + \norm{g}_2^2 \\
				&= \norm{f}_2^2 + \norm{g}_2^2\qedhere
			\end{split}
		\end{equation*}
	\end{proof}

	\begin{teorema}[Proiezione ortogonale]\label{th:proiez-ortog}
		Dato uno spazio euclideo complesso \(\E\), sia \(f \in \E\); se \(\Set{\varphi_j | i \in \Set{1, \dots, N}} \subset \E\) è un insieme finito di elementi linearmente indipendenti, allora \(f^* = \sum_{j = 1}^N c_j^* \varphi_j\), ove i \(c_j^* \!\) soddisfano le \emph{equazioni normali}
		\begin{equation}\label{eq:eq-normali}
			\forall j \in \Set{1, \dots, N} \colon \sum_{k = 1}^N (\varphi_j, \varphi_k) c_k^* = (\varphi_j, f)
		\end{equation}
		è tale che\label{eq:proiez-ortog}
		\begin{equation}
			\norm{f - f^*}_2 = \min_{g \in \Braket{\varphi_1, \dots, \varphi_k}} \norm{f - g}_2
		\end{equation}
		Questa soluzione è tale che \(f^* - f\) è ortogonale a tutti i \(\varphi_j\), o equivalentemente si ha \((f, \varphi_j) = (f^*, \varphi_j)\) per ogni \(j \in \Set{1, \dots, N}\).
	\end{teorema}

	\begin{proof}
		Mostriamo innanzitutto che la soluzione di miglior approssimazione è unica. Se \(f^* \in \E\) è tale che \(f^* - f \in \Braket{\varphi_1, \dots, \varphi_N}^\perp\) e \(\hat{f} \ne f^*\) è elemento di miglior approssimazione nel senso della \eqref{eq:proiez-ortog}, allora \(f^* - \hat{f} \in \Braket{\varphi_1, \dots, \varphi_N}\); per il Teorema~\ref{th:pitagora} si ha
		\begin{equation*}
			\norm{f - \hat{f}}_2^2 = \norm{(f - f^*) + (f^* - \hat{f})}_2^2 = \norm{f - f^*}_2^2 + \norm{f^* - \hat{f}}_2^2 > \norm{f - f^*}_2^2
		\end{equation*}
		il che è contro l'ipotesi che \(\hat{f}\) sia di miglior approssimazione.
		
		Rimane da mostrare l'esistenza di \(f^*\), ovvero di \(c_j^*\) che soddisfino per ogni \(k \in \Set{1, \dots, N}\)
		\begin{equation*}
			\begin{split}
				0 = \qty(\sum_{j = 1}^N c_j^* \varphi_j - f, \varphi_k) &= \qty(\sum_{j = 1}^N c_j^* (\varphi_j, \varphi_k)) - (f, \varphi_k) \\
				&= \sum_{j = 1}^N (\varphi_j, \varphi_k) c_j^* - (f, \varphi_k)
			\end{split}
		\end{equation*}
		Questa condizione, equivalente di fatto alla \eqref{eq:eq-normali}, è soddisfatta se e solo se la matrice \(G\) che ha come coordinata \((j, k)\) il prodotto interno \((\varphi_j, \varphi_k)\) è non singolare. Certamente \(G\) è hermitiana, in quanto \((\varphi_j, \varphi_k) = \overline{(\varphi_k, \varphi_j)}\); per questo motivo si ha per ogni \(v \in \C^N \setminus \Set{0_{\C^N}}\)
		\begin{equation*}
			\her{\qty(\her{v} G v)} = \her{v} \her{G} \her{\qty(\her{v})} = \her{v} G v \iff \her{v} G v \in \R
		\end{equation*}
		Scelto ora \(v = \sum_{j = 1}^N v_j \varphi_j \ne 0\) e definito \(u = \sum_{j = 1}^N \overline{v}_j \varphi_j \ne 0\), si ha
		\begin{equation*}
			\begin{split}
				\her{v} G v &= \sum_{j = 1}^N \overline{v}_j \sum_{k = 1}^N G_{j, k} v_k \\
				&= \sum_{j = 1}^N \overline{v}_j \sum_{k = 1}^N (\varphi_j, \varphi_k) v_k = \sum_{j = 1}^N \overline{v}_j \sum_{k = 1}^N (\varphi_j, \overline{v}_k \varphi_k) \\
				&= \sum_{j = 1}^N \overline{v}_j \qty(\varphi_j, \sum_{k = 1}^N \overline{v}_k \varphi_k) = \sum_{j = 1}^N \qty(\overline{v}_j \varphi_j, \sum_{k = 1}^N \overline{v}_k \varphi_k) \\
				&= \qty(\sum_{j = 1}^N \overline{v}_j \varphi_j, \sum_{k = 1}^N \overline{v}_k \varphi_k) = (u, u) = \norm{u}_2^2 > 0
			\end{split}
		\end{equation*}
		da cui è evidente che \(G\) è definita positiva e, quindi, non singolare.
	\end{proof}

	\begin{osservazione}
		Se l'insieme \(\Set{\varphi_1, \dots, \varphi_N}\) non è composto di vettori a due a due ortogonali, per ottenere l'elemento di miglior approssimazione bisogna:
		\begin{itemize}
			\item calcolare i prodotti scalari \(G_{j, k} = (\varphi_j, \varphi_k)\) per \(j, k \in \Set{1, \dots, N}\);
			\item calcolare i coefficienti \(b_j = (\varphi_j, f)\) per \(j \in \Set{1, \dots, N}\);
			\item risolvere numericamente il sistema lineare \(c^* G = b\).
		\end{itemize}
		Se, invece, l'insieme dei \(\varphi_j\) consta di elementi a due a due ortogonali, ovvero è una base ortogonale del suo sottospazio generato, si vede che la risoluzione delle equazioni \eqref{eq:eq-normali} è data dai \emph{coefficienti di Fourier}
		\begin{equation}\label{eq:coeff-fourier}
			c_k^* = \frac{(\varphi_k, f)}{(\varphi_k, \varphi_k)}
		\end{equation}
		al variare di \(k \in \Set{1, \dots, N}\), il che necessita di soli \(2 N\) prodotti interni e \(N\) divisioni. Il procedimento è ancora meno dispendioso se i \(\varphi_j\) formano una base ortonormale del proprio sottospazio generato, in quanto non è necessario effettuare le divisioni.
	\end{osservazione}

	I prossimi risultati servono a chiarire il concetto di \emph{base} di uno spazio euclideo di dimensione anche infinita e sotto quali ipotesi si possa ottenere una base numerabile oppure ortonormale.
	
	\begin{definizione}[Spazio separabile]
		Uno spazio euclideo \(\E\) si dice \emph{separabile} se esiste \(S \subseteq \E\) numerabile denso in \(\E\).
	\end{definizione}

	\begin{teorema}\label{th:base-sp-euclideo-separabile}
		In uno spazio euclideo separabile \(\E\) esiste un sottoinsieme \(\Set{\varphi_k | k \in \N}\) di cardinalità finita o al piú numerabile tale che, se \(x \in \E\), allora esistono \(c_k\) con \(k \in \N\) tali che
		\begin{equation*}
			x = \sum_{k \in \N} c_k \varphi_k
		\end{equation*}
		ovvero
		\begin{equation*}
			\lim_{n \to \infty} \norm{x - \sum_{k = 0}^n c_k \varphi_k}_2 = 0
		\end{equation*}
		L'insieme \(\Set{\varphi_k | k \in \N}\) prende il nome di \emph{base} di \(\E\).
	\end{teorema}

	D'ora in avanti supporremo sempre che lo spazio euclideo \(\E\) è separabile.
	
	Si può dimostrare il seguente teorema, che generalizza a spazi di dimensione infinita l'algoritmo di Gram-Schmidt.
	
	\begin{teorema}\label{th:gram-schmidt}
		Dato uno spazio euclideo \(\E\), se esiste un insieme numerabile \(\Set{f_n | n \in \N^*} \subseteq \E\) di elementi linearmente indipendenti, allora \(\E\) contiene un insieme \(\Set{\varphi_k | k \in \N^*}\) tale che:
		\begin{itemize}
			\item per ogni \(m, n \in \N^*\) si ha \((\varphi_m, \varphi_n) = \delta_{m, n}\);
			\item \(\varphi_n \in \Braket{f_1, \dots, f_n}\) per ogni \(n \in \N^*\);
			\item \(f_n \in \Braket{\varphi_1, \dots, \varphi_n}\) per ogni \(n \in \N^*\).
		\end{itemize}
	\end{teorema}

	Si osservi che l'insieme degli \(f_n\) può anche essere infinito, al contrario di ciò che è richiesto per ortogonalizzare le matrici, cosí come può non essere finito anche l'insieme dei \(\varphi_k\). Dal Teorema~\ref{th:gram-schmidt}, inoltre, segue che se \(\E\) ha una base numerabile di elementi linearmente indipendenti, allora ammette anche una base ortonormale numerabile.
	
	\begin{definizione}[Serie di Fourier]\label{def:serie-fourier}
		Dato uno spazio euclideo \(\E\) dotato di una successione di elementi ortonormali \(\Set{\varphi_k | k \in \N^*}\), se per \(f \in \E\) si definiscono \(c_k = (f, \varphi_k)\) per ogni \(k \in \N^*\), si dice \emph{serie di Fourier} di \(f\) la serie formale
		\begin{equation}\label{eq:serie-fourier}
			\sum_{k = 1}^\infty c_k \varphi_k
		\end{equation}
	\end{definizione}

	\begin{definizione}\label{def:chiuso}
		Sia \(\Set{\varphi_n | n \in \N^*}\) una successione di elementi ortonormali di uno spazio vettoriale normato \(X\). Se ogni \(f \in X\) può essere scritto formalmente come serie di Fourier mediante i \(\varphi_n\), allora l'insieme dei \(\varphi_n\) si dice \emph{chiuso in \(X\)}.
	\end{definizione}

	\begin{teorema}[Bessel-Parseval]\label{th:bessel-parseval}
		Data una successione \(\Set{\varphi_k | k \in \N^*}\) di elementi ortonormali di uno spazio euclideo \(\E\), se \(f \in \E\), allora
		\begin{equation}\label{eq:bessel-parseval}
			\min_{a_1, \dots, a_n} \norm{f - \sum_{k = 1}^n a_k \varphi_k}_2 = \sqrt{\norm{f}_2^2 - \sum_{k = 1}^n c_k^2}
		\end{equation}
		ove \(c_k = (f, \varphi_k)\) per ogni \(k \in \Set{1, \dots, n}\). Vale, inoltre, la diseguaglianza di Bessel
		\begin{equation}\label{eq:bessel}
			\sum_{k = 1}^\infty c_k^2 \le \norm{f}_2^2
		\end{equation}
		e vale l'uguaglianza, detta uguaglianza di Parseval, per ogni \(f \in \E\)
		\begin{equation}\label{eq:parseval}
			\sum_{k = 1}^\infty c_k^2 = \norm{f}_2^2
		\end{equation}
		se e solo se l'insieme \(\Set{\varphi_k | k \in \N^*}\) è chiuso in \(\E\).
	\end{teorema}

	\begin{osservazione}
		In base al Teorema~\ref{th:bessel-parseval}, la soluzione al problema di miglior approssimazione in norma euclidea esiste ed è unica, ed è determinata dai coefficienti di Fourier. In virtú della \eqref{eq:parseval}, poi, se \(\Set{c_1, \dots, c_n}\) determina l'elemento di miglior approssimazione in norma indotta dal prodotto scalare in \(S_n = \Braket{\varphi_1, \dots, \varphi_n}\), allora
		\begin{equation*}
			\lim_{n \to \infty} \norm{f -  \sum_{k = 1}^n c_k \varphi_k}_2 = \lim_{n \to \infty} \sqrt{\norm{f}_2^2 - \sum_{k = 1}^n c_k^2} = 0
		\end{equation*}
	\end{osservazione}

\section{Polinomi trigonometrici}
	
	\begin{definizione}[Polinomi trigonometrici reali]\label{def:polin-trig-reali}
		Si dice \emph{spazio dei polinomi trigonometrici reali di grado \(n\)} lo spazio vettoriale \(\T_n^\R\) costituito dalle combinazioni lineari delle funzioni
		\begin{subequations}
			\begin{align}\label{eq:polin-trig-reali}
				\varphi_0^* (x)         & = 1          \\
				\varphi_{2 k - 1}^* (x) & = \cos (k x) \\
				\varphi_{2 k}^* (x)     & = \sin (k x)
			\end{align}
		\end{subequations}
		per \(k \in \Set{1, \dots, n}\).
	\end{definizione}

	Dati \(n, m \in \N\), valgono le formule di Werner
		\begin{align*}
			\cos (n x) \cos (m x) &= \frac{\cos ((n + m) x) + \cos ((n - m) x)}{2} \\
			\sin (n x) \sin (m x) &= \frac{\cos ((n - m) x) - \cos ((n + m) x)}{2}
		\end{align*}
	e si ha
	\begin{equation*}
		\int_{- \pi}^\pi \cos (k x) \dd{x} =
		\begin{cases}
			0     & k \ne 0 \\
			2 \pi & k = 0
		\end{cases}
	\end{equation*}
	A partire da queste identità si trova che
		\begin{gather*}
			n \in \N^* \implies \int_{- \pi}^\pi \cos^2 (n x) \dd{x} = \int_{- \pi}^\pi \frac{\cos (2 n x) + 1}{2} \dd{x} = \pi \\
			n \in \N^* \implies \int_{- \pi}^\pi \sin^2 (n x) \dd{x} = \int_{- \pi}^\pi \qty(1 - \cos^2 (n x)) \dd{x} = \pi \\
			m, n \in \N^* \implies \int_{- \pi}^\pi \cos (n x) \sin (m x) \dd{x} = 0 \\
			m, n \in \N^*, m \ne n \implies \int_{- \pi}^\pi \cos (n x) \cos (m x) \dd{x} = 0 \\
			m, n \in \N^*, m \ne n \implies \int_{- \pi}^\pi \sin (n x) \sin (m x) \dd{x} = 0
		\end{gather*}

	Consideriamo ora lo spazio \(L^2_\R ([\,- \pi, \pi \,])\) delle funzioni \(f \colon [\,- \pi, \pi \,] \to \R\) misurabili tali che \(\abs{f}^2\) sia integrabile e dotiamolo del prodotto scalare
	\begin{equation}\label{eq:prod-scal-l2r}
		(f, g) \coloneqq \int_{- \pi}^\pi f (x) g (x) \dd{x}
	\end{equation}
	Si può dimostrare che le \(2 n + 1\) funzioni
	\begin{subequations}\label{eq:polin-trig-reali-normalizzati}
		\begin{align}
			\varphi_0 (t)         & = \frac{1}{\sqrt{2 \pi}}      \\
			\varphi_{2 k - 1} (t) & = \frac{\cos k t}{\sqrt{\pi}} \\
			\varphi_{2 k} (t)     & = \frac{\sin k t}{\sqrt{\pi}}
		\end{align}
	\end{subequations}
	definite per \(k \in \Set{1, \dots, n}\), formano una base ortonormale di \(\T_n^\R\), che dotiamo del prodotto scalare definito nella \eqref{eq:prod-scal-l2r}. Si può anche vedere che la successione \(\Set{\varphi_n | n \in \N}\) è chiusa in \(L^2_\R ([\,- \pi, \pi \,])\).
	
	In virtú di queste osservazioni e del Teorema~\ref{th:bessel-parseval} si può dimostrare il seguente asserto.
	
	\begin{teorema}
		Considerata la successione di elementi ortonormali in \(L^2_\R ([\,- \pi, \pi \,])\) definita nella \eqref{eq:polin-trig-reali-normalizzati}, i coefficienti di Fourier che determinano l'elemento di miglior approssimazione di \(f \in L^2_\R ([\,- \pi, \pi \,])\) secondo il prodotto scalare definito nella \eqref{eq:prod-scal-l2r} sono
		\begin{subequations}
			\begin{align}
				c_0         & = \frac{1}{\sqrt{2 \pi}} \int_{- \pi}^\pi f (x) \dd{x}          \\
				c_{2 k - 1} & = \frac{1}{\sqrt{\pi}} \int_{- \pi}^\pi f (x) \cos (k x) \dd{x} \\
				c_{2 k}     & = \frac{1}{\sqrt{\pi}} \int_{- \pi}^\pi f (x) \sin (k x) \dd{x}
			\end{align}
		\end{subequations}
		definiti per \(k \in \N^*\) e si verifica \(\lim_{n \to \infty} E_n (f) = 0\).
	\end{teorema}

	\begin{definizione}[Polinomi trigonometrici complessi]\label{def:polin-trig-complessi}
		Si dice \emph{spazio dei polinomi trigonometrici complessi di grado \(n\)} lo spazio vettoriale \(\T_n^\C\) costituito dalle combinazioni lineari delle funzioni
		\begin{subequations}\label{eq:polin-trig-complessi}
			\begin{align}
				\varphi_0^* (x)         & = 1              \\
				\varphi_{2 k - 1}^* (x) & = \exp (- \ii k x) \\
				\varphi_{2 k}^* (x)     & = \exp (\ii k x)
			\end{align}
		\end{subequations}
		per \(k \in \Set{1, \dots, n}\), ove \(\ii\) indica la costante immaginaria.
	\end{definizione}

	\begin{teorema}
		La successione \(\Set{\varphi_k^* | k \in \N}\) definita nella \eqref{eq:polin-trig-complessi} è composta di elementi ortogonali di \(L^2_\C ([\, 0, 2 \pi \,])\).
	\end{teorema}
	
	\begin{proof}
		In base all'identità di Eulero \(\exp (\ii t) = \cos t + \ii \sin t\), valida per ogni \(t \in \R\), si ha per ogni \(k \in \Z\) che
		\begin{equation*}
			\overline{\exp (\ii k x)} = \overline{\cos (k x) + \ii \sin (k x)} = \cos (k x) - \ii \sin (k x) = \exp (- \ii k x)
		\end{equation*}
		Dall'uguaglianza \(\exp (\ii j x) \exp (\ii k x) = \exp (\ii (j + k) x)\), valida per ogni \(j, k \in \Z\), si ricava
		\begin{equation*}
			\int_0^{\mathrlap{2 \pi}} \exp (\ii j x) \, \overline{\exp(\ii k x)} \dd{x} = \int_0^{\mathrlap{2 \pi}} \exp (\ii (j - k) x) \dd{x}
		\end{equation*}
		Se \(j = k\), l'integrale sopra vale \(2 \pi\), mentre se \(j \ne k\) si ha
		\begin{equation*}
			\begin{split}
				\int_0^{\mathrlap{2 \pi}} \exp (\ii (j - k) x) \dd{x} &= \int_0^{\mathrlap{2 \pi}} \cos ((j - k) x) + \ii \sin ((j - k) x) \dd{x} \\
				&= \frac{1}{j - k} \int_0^{\mathrlap{2 (j - k) \pi}} \quad \cos t + \ii \sin t \dd{t} \\
				&= \frac{1}{j - k} \eval[\sin t - \ii \cos t|_{t = 0}^{2 (j - k) \pi} = 0
			\end{split}
		\end{equation*}
		dato che \(\sin\) e \(\cos\) sono funzioni di periodo \(2 \pi\).
	\end{proof}

	Dal momento che la successione \(\Set{\varphi_k^* | k \in \N}\) è chiusa in \(L_\C^2 ([\, 0, 2 \pi \,])\), si può enunciare il seguente teorema.
	
	\begin{teorema}
		Considerata la successione \(\Set{\varphi_k | k \in \N}\) di elementi ortonormali definita da
		\begin{subequations}
			\begin{align}
				\varphi_0 (x)         & = \frac{1}{\sqrt{2 \pi}}                \\
				\varphi_{2 k - 1} (x) & = \frac{\exp (- \ii k x)}{\sqrt{2 \pi}} \\
				\varphi_{2 k} (x)     & = \frac{\exp (\ii k x)}{\sqrt{2 \pi}}
			\end{align}
		\end{subequations}
		i coefficienti di Fourier che determinano l'elemento di miglior approssimazione di una funzione \(f \in L_\C^2 ([\, 0, 2 \pi \,])\) secondo il prodotto scalare definito da \((f, g) \coloneqq \int_0^{2 \pi} f (x) \overline{g (x)} \dd{x}\) sono
		\begin{subequations}
			\begin{align}
				c_0         & = \frac{1}{\sqrt{2 \pi}} \int_0^{\mathrlap{2 \pi}} f (x) \dd{x}                  \\
				c_{2 k - 1} & = \frac{1}{\sqrt{2 \pi}} \int_0^{\mathrlap{2 \pi}} f (x) \exp (\ii k x) \dd{x}   \\
				c_{2 k}     & = \frac{1}{\sqrt{2 \pi}} \int_0^{\mathrlap{2 \pi}} f (x) \exp (- \ii k x) \dd{x}
			\end{align}
		\end{subequations}
		definiti per \(k \in \N^*\) e si verifica \(\lim_{n \to \infty} E_n (f) = 0\).
	\end{teorema}

	Di solito le serie di Fourier sono scritte attraverso la serie bilatera
	\begin{equation*}
		f (x) = \sum_{k = - \infty}^\infty \gamma_k \exp (\ii k x)
	\end{equation*}
	ove si è posto
	\begin{equation}\label{eq:coeff-fourier-complessi}
		\gamma_k = \frac{1}{2 \pi} \int_0^{\mathrlap{2 \pi}} f (x) \exp (- \ii k x) \dd{x}
	\end{equation}
	Si ha, infatti,
	\begin{equation}\label{eq:fourier-formula-complessa}
		\begin{split}
			f (x) &= \sum_{k = - \infty}^\infty \frac{1}{\sqrt{2 \pi}} \qty(\int_0^{\mathrlap{2 \pi}} f (x) \exp (- \ii k x) \dd{x}) \frac{\exp (\ii k x)}{\sqrt{2 \pi}} \\
			&= \sum_{k = - \infty}^\infty \frac{1}{2 \pi} \qty(\int_0^{\mathrlap{2 \pi}} f (x) \exp (- \ii k x) \dd{x}) \exp (\ii k x)
		\end{split}
	\end{equation}

	Data una funzione \(f \in L_\C^2 ([\, 0, 2 \pi \,])\) continua in \([\, 0, 2 \pi \,]\) e tale che \(f (0) = f (2 \pi)\), essa ammette una rappresentazione formale come nella \eqref{eq:fourier-formula-complessa}. In pratica, però, non si calcola tutta la serie, ma si considera un'approssimazione del tipo
	\begin{equation}\label{eq:funz-fourier-approx}
		f_M (x) = \sum_{k = - M}^M \frac{1}{2 \pi} \qty(\int_0^{\mathrlap{2 \pi}} f (x) \exp (- \ii k x) \dd{x}) \exp (\ii k x)
	\end{equation}
	con un \(M \in \N\) sufficientemente grande. Questa approssimazione trigonometrica richiede che si calcolino numericamente i coefficienti
	\begin{equation}\label{eq:coeff-approx-fourier}
		I_k \coloneqq \frac{1}{2 \pi} \int_0^{\mathrlap{2 \pi}} f (x) \exp (- \ii k x) \dd{x}
	\end{equation}
	per \(k \in \Set{-M, \dots, M}\). È possibile dimostrare che, se \(f\) è continua e periodica, ovvero \(f (0) = f (2 \pi)\), è vantaggioso calcolare i coefficienti nella \eqref{eq:coeff-approx-fourier} mediante la \emph{formula dei trapezi composta}
	\begin{equation*}
		\int_a^b g(x) \dd{x} \approx \frac{h}{2} (g (a) + g (b)) + h \sum_{j = 1}^{M^* - 1} f (x_j)
	\end{equation*}
	ove \(x_j = a + j h\) per \(j \in \Set{0, \dots, M^*}\) e \(h = (b - a) / M^*\). Nel caso particolare in cui \(g \colon [\, 0, 2 \pi \,] \to \R\) sia continua e periodica, ovvero \(g (0) = g (2 \pi)\), usando la formula dei trapezi composta su \(M^* + 1\) nodi equispaziati \(x_j = j h\), ove \(j \in \Set{0, \dots, M^*}\) e \(h = 2 \pi / M^*\), si trova
	\begin{equation*}
		\begin{split}
			\int_0^{2 \pi} g (x) \dd{x} &\approx \frac{h}{2} (g (0) + g (2 \pi)) + h \sum_{j = 1}^{M^* - 1} g (x_j) \\
			&= h \, g (2 \pi) + h \sum_{j = 1}^{M^* - 1} g (x_j) \\
			&= h \sum_{j = 1}^{M^*} g (x_j) = \frac{2 \pi}{M^*} \sum_{j = 1}^{M^*} g (x_j) \\
			&= \frac{2 \pi}{M^*} \sum_{j = 1}^{M^*} g \qty(\frac{2 \pi j}{M^*})
		\end{split}
	\end{equation*}
	Supponendo, ora, che \(f\) sia continua e periodica in \([\, 0, 2 \pi \,]\), se per ogni \(k \in \Set{-M, \dots, M}\) consideriamo \(g (x) = f (x) \exp (- \ii k x)\) e definiamo \(M^* = 2 M + 1\), per come sono definiti i \(\gamma_k\) nella \eqref{eq:coeff-fourier-complessi} si ottiene
	\begin{equation}
		\gamma_k \approx \frac{1}{2 M + 1} \sum_{j = 1}^{2 M + 1} f \qty(\frac{2 \pi j}{2 M + 1}) \, \exp \qty(- \ii k \frac{2 \pi j}{2 M + 1})
	\end{equation}
	Se, poi, definiamo per \(j \in \Set{1, \dots, 2 M + 1}\)
	\begin{equation*}
		\mathcal{X}_j = \frac{1}{2 M + 1} f \qty(\frac{2 \pi j}{2 M + 1})
	\end{equation*}
	e a partire da essi
	\begin{equation}\label{eq:coeff-fourier-approx}
		\gammanum{k} = \sum_{j = 1}^{2 M + 1} \mathcal{X}_j \exp \qty(- \ii k \frac{2 \pi j}{2 M + 1})
	\end{equation}
	si vede che \(\gamma_k \approx \gammanum{k}\) per ogni \(k \in \Set{-M, \dots, M}\). Sotto opportune condizioni, si può ricorrere all'algoritmo della trasformata rapida di Fourier (\textsc{fft}) per calcolare i \(2 M + 1\) coefficienti \(\gammanum{-M}, \dots, \gammanum{M}\) in \(\mathcal{O} (M \log M)\) operazioni anziché in \(\mathcal{O} (M^2)\), come si riuscirebbe con altri algoritmi.
	
	\begin{teorema}[Polinomio trigonometrico interpolante]\label{th:polin-trig-interp}
		Se una funzione \(f \colon [\, 0, 2 \pi \,] \to \R\) è continua e periodica, allora il polinomio trigonometrico
		\begin{equation}\label{eq:polin-trig-interp}
			p_M (x) = \sum_{k = - M}^M \gammanum{k} \exp(\ii k x)
		\end{equation}
		con \(\gammanum{k}\) definito come nella \eqref{eq:coeff-fourier-approx}, interpola \(f\) nei nodi equispaziati
		\begin{equation*}
			x_j = \frac{2 \pi j}{2 M + 1}
		\end{equation*}
		definiti per \(j \in \Set{0, \dots, 2 M + 1}\).
	\end{teorema}

	Con i risultati seguenti ci assicuriamo del margine di errore in norma euclidea della funzione \(f_M\) definita nella \eqref{eq:funz-fourier-approx} e dell'interpolante \(p_M\) definito nella \eqref{eq:polin-trig-interp} e, di conseguenza, della natura dell'errore commesso nell'approssimare numericamente i \(\gamma_k\) coi \(\gammanum{k}\).
	
	\begin{definizione}\label{def:variaz-limit}
		Una funzione \(f \colon [\, a, b \,] \to \R\) si dice \emph{a variazione limitata} se
		\begin{equation}
			T_a^b (f) \coloneqq \sup \Set{\sum_{i = 1}^n \abs{f (t_i) - f (t_{i - 1})} : a = t_0 < \dots < t_n = b} < + \infty
		\end{equation}
		La quantità \(T_a^b (f)\) prende il nome di \emph{variazione} di \(f\).
	\end{definizione}

	\begin{esempio}
		Le funzioni lipschitziane su \([\, a, b \,]\) sono a variazione limitata, perché esiste \(L \in \R_{\ge 0}\) tale che \(T_a^b (f) \le L (b - a)\). Per lo stesso motivo le funzioni di classe \(\cont^1 ([\, a, b \,])\) sono anch'esse a variazione limitata.
	\end{esempio}
	
	Nei seguenti teoremi usiamo la notazione
	\begin{equation}
		S (\alpha) \coloneqq \Set{x + i y \in \C : -\alpha < y < \alpha}
	\end{equation}
	definita per ogni \(\alpha \in \R_{\ge 0}\).
	
	\begin{teorema}
		Se \(f \colon [\, 0, 2 \pi \,] \to \R\) è una funzione differenziabile \(\eta \in \N^*\) volte, periodica e tale che \(f^{(\eta)}\) sia periodica e a variazione limitata \(V\) in \([\, 0, 2 \pi \,]\), definiti
		\begin{gather*}
			I (f) = \int_0^{2 \pi} f (x) \dd{x} \\
			I_N (f) = \frac{\pi}{N} (f (0) + f (2 \pi)) + \frac{2 \pi}{N} \sum_{j = 2}^{N - 1} f \qty(\frac{2 \pi j}{N})
		\end{gather*}
		allora
		\begin{equation}
			\abs{I_N (f) - I (f)} \le \frac{4 V}{N^{\eta + 1}}
		\end{equation}
		Se, poi, \(f\) è analitica in \(S (\alpha)\) e ivi \(\abs{f (t)} \le \tilde{M}\), allora
		\begin{equation}
			\abs{I_N (f) - I (f)} \le \frac{4 \pi \tilde{M}}{\ee^{\alpha N} - 1}
		\end{equation}
	\end{teorema}

	\begin{teorema}
		Se \(f \colon [\, 0, 2 \pi \,] \to \R\) è una funzione differenziabile \(\eta \in \N^*\) volte, periodica e tale che \(f^{(\eta)}\) sia periodica e a variazione limitata \(V\) in \([\, 0, 2 \pi \,]\), allora
		\begin{equation}
			\abs{\gamma_k} \le \frac{V}{2 \pi \abs{k}^{\eta + 1}}
		\end{equation}
		Se, poi, \(f\) è analitica in \(S (\alpha)\) e ivi \(\abs{f (t)} \le \tilde{M}\), allora
		\begin{equation}
			\abs{\gamma_k} \le \tilde{M} \ee^{- \alpha \abs{k}}
		\end{equation}
	\end{teorema}

	\begin{teorema}
		Se \(f \colon [\, 0, 2 \pi \,] \to \R\) è una funzione differenziabile \(\eta \in \N^*\) volte, periodica e tale che \(f^{(\eta)}\) sia periodica e a variazione limitata \(V\) in \([\, 0, 2 \pi \,]\), allora \(f_M\) e \(p_M\) definiti nella \eqref{eq:funz-fourier-approx} e nella \eqref{eq:polin-trig-interp} rispettivamente verificano
		\begin{align}
			\norm{f - f_M}_\infty &\le \frac{V}{\pi \eta M^\eta} &
			\norm{f - p_M}_\infty &\le \frac{2 V}{\pi \eta M^\eta}
		\end{align}
	\end{teorema}
	
	\chapter{Polinomi ortogonali}

	\begin{definizione}
		Si dice \emph{spazio di Hilbert} uno spazio euclideo completo, separabile e di dimensione infinita.
	\end{definizione}

	\begin{esempio}
		Lo spazio \((L^2 ((\, a, b \,)), \norm{\cdot}_2)\) delle funzioni reali misurabili di modulo quadrato integrabile su un intervallo anche illimitato \((\, a, b \,)\) con la norma definita a partire dal prodotto scalare descritto nella \eqref{eq:prod-scal-l2r} è uno spazio di Hilbert.
	\end{esempio}

	\begin{esempio}
		Data una funzione misurabile positiva \(w \colon (\, a, b \,) \to \R\), lo spazio \((L^2_w ((\, a, b \,)), \norm{\cdot}_{2, w})\) delle funzioni misurabili \(f\) tali che
		\begin{equation*}
			\int_a^b \abs{f (x)}^2 w (x) \dd{x} < + \infty
		\end{equation*}
		è uno spazio di Hilbert dotato del prodotto scalare
		\begin{equation}\label{eq:prod-scal-l2w}
			(f, g)_{2, w} \coloneqq \int_a^b f (x) g (x) w (x) \dd{x}
		\end{equation}
	\end{esempio}

	\begin{definizione}
		Si dice \emph{funzione peso} una funzione \(w \colon (\, a, b \,) \to \R\) non negativa tale che per ogni \(n \in \N\)
		\begin{subequations}
			\begin{equation}
				\int_a^b \abs{x}^n w (x) \dd{x} < + \infty
			\end{equation}
			e che per ogni \(g\) continua e non negativa si abbia
			\begin{equation}
				\int_a^b g (x) w (x) \dd{x} = 0 \implies g \equiv 0
			\end{equation}
		\end{subequations}
		
	\end{definizione}

	\begin{esempio}[Funzioni peso classiche]
		Riportiamo alcuni esempi di funzioni peso comuni.
			\begin{itemize}
				\item Peso di Legendre: \(w (x) = 1\) per \(x \in [\, -1, 1\,]\).
				\item Peso di Chebyshev: \(w (x) = 1 / \sqrt{1 - x^2}\) per \(x \in (\, -1, 1 \,)\).
				\item Peso di Gegenbauer: \(w (x) = \qty(1 - x^2)^{\gamma - 1 / 2}\) per \(x \in (\, -1, 1 \,)\), con \(\gamma > (- 1 / 2)\).
				\item Peso di Jacobi: \(w (x) = (1 - x)^\alpha (1 + x)^\beta\) per \(x \in (\, -1, 1 \,)\), con \(\alpha, \beta > - 1\).
				\item Peso di Laguerre: \(w (x) = \ee^{-x}\) per \(x \in (\, 0, + \infty \,)\).
				\item Peso di Hermite: \(w (x) = \ee^{- x^2}\) per \(x \in \R\).
			\end{itemize}
	\end{esempio}

	\begin{teorema}
		Per ogni \(n \in \N\) si ha \(\P_n \subseteq L_w^2 ((\, a, b \,))\).
	\end{teorema}

	\begin{proof}
		Scelto un qualunque polinomio \(p_n (x) = \sum_{k = 0}^n a_k x^k\) di grado \(n\), per la disuguaglianza triangolare e per il fatto che
		\begin{equation*}
			\norm{x^k}_{2, w}^2 = \int_a^b \abs{x}^{2 k} w (x) \dd{x} < + \infty
		\end{equation*}
		si ottiene
		\begin{equation*}
			\norm{p_n}_{2, w} = \norm{\sum_{k = 0}^n a_k x^k}_{2, w} \le \sum_{k = 0}^n \abs{a_k} \, \norm{x^k}_{2, w} < + \infty
		\end{equation*}
		da cui si conclude che \(p_n \in L_w^2 ((\, a, b \,))\).
	\end{proof}

	\begin{osservazione}
		Se \(a\) e \(b\) sono finiti, per il teorema di Weierstrass
		\begin{equation*}
			\norm{x^n}_\infty = \max_{x \in [\, a, b \,]} \abs{x}^n < + \infty
		\end{equation*}
		e quindi, poiché \(w (x) \ge 0\) per ogni \(x \in (\, a, b \,)\),
		\begin{equation*}
			\int_a^b \abs{x}^n w (x) \dd{x} \le \norm{x^n}_\infty \int_a^b w (x) \dd{x}
		\end{equation*}
		In base a ciò, si conclude che per ogni \(n \in \N\)
		\begin{equation}
			\int_a^b w (x) \dd{x} < + \infty \implies \int_a^b \abs{x}^n w (x) \dd{x} < + \infty
		\end{equation}
	\end{osservazione}

	Fissati \(f \in L_w^2 ((\, a, b \,))\) e \(n \in \N\), vogliamo risolvere il \emph{problema dei minimi quadrati} nel continuo, ovvero determinare il polinomio \(p_n^* \in \P_n\), se esiste, che minimizzi al variare di \(p_n \in \P_n\) la quantità
	\begin{equation*}
		\norm{f - p_n}_{2, w} = \int_a^b \abs{f (x) - p_n (x)}^2 w (x) \dd{x}
	\end{equation*}
	
	Si può dimostrare che, nel caso in cui \(a, b\) siano finiti e \(f \in \cont ([\, a, b \,])\), si ha \(\norm{f - p_n^*}_{2, w} \to 0\) per \(n \to + \infty\).
	
	Dal momento che \(L_w^2 ([\, a, b \,])\) è uno spazio euclideo, individuata una base \(\varphi_0, \dots, \varphi_n\) di \(\P_n\), abbiamo già visto che la soluzione del problema di trovare
	\begin{equation*}
		\norm{f - f^*}_{2, w} = \; \min_{g \in \Braket{\varphi_0, \dots, \varphi_n}} \norm{f - g}_{2, w}
	\end{equation*}
	è data da \(f^* = \sum_{j = 0}^n \gamma_j^* \varphi_j\), ove i \(\gamma_j^*\) verificano le equazioni normali definite nella \eqref{eq:eq-normali}. Abbiamo anche visto che \(f^*\) è tale che \((f, \varphi_j)_{2, w} = (f^*, \varphi_j)_{2, w}\) per ogni \(j \in \Set{0, \dots, n}\).
	
	\begin{definizione}
		Una sequenza di polinomi \(\varphi_0, \dots, \varphi_n\) triangolare, ovvero tale che \(\deg (\varphi_k) = k\) per ogni \(k \in \Set{0, \dots, n}\), si dice \emph{ortogonale} rispetto alla funzione peso \(w\) se verifica per ogni \(i, j \in \Set{0, \dots, n}\)
		\begin{equation}
			(\varphi_i, \varphi_j)_{2, w} = c_i \delta_{i, j}
		\end{equation}
		ove \(\delta_{i, j}\) è il delta di Kronecker e \(c_i > 0\) per ogni \(i \in \Set{0, \dots, n}\).
	\end{definizione}

	\begin{osservazione}
		Attraverso il procedimento di Gram-Schmidt, si può dimostrare che esiste una tale famiglia di polinomi e costruirla direttamente. Si può osservare, inoltre, che ogni polinomio di grado \(n\) si può scrivere univocamente come combinazione lineare di \(\varphi_0\, dots, \varphi_n\). In base a queste osservazioni, per ogni \(p_m = \sum_{j = 0}^m a_j x^j\) con \(m < n\) dalla bilinearità del prodotto scalare segue che
		\begin{equation}
			\qty(\varphi_{m + 1}, p_m)_{2, w} = \qty(\varphi_{m + 1}, \sum_{j = 0}^m a_j x^j)_{\! 2, w} \hspace{-.7em} = \sum_{j = 0}^m a_j \qty(\varphi_{m + 1}, \varphi_j)_{2, w} = 0
		\end{equation}
	\end{osservazione}
	
	\begin{teorema}[Christoffel]
		Se una famiglia triangolare di polinomi \(\varphi_0, \dots, \varphi_n\) è ortogonale in \((\, a, b \,)\) rispetto a una funzione peso \(w\), allora il polinomio \(\varphi_n\) ha esattamente \(n\) zeri, i quali hanno tutti molteplicità \(1\) ed appartengono all'intervallo \((\, a, b \,)\).
	\end{teorema}

	\begin{proof}
		Chiamati \(x_1, \dots, x_m\) con \(m \le n\) gli zeri di \(\varphi_n\) interni ad \((\, a, b \,)\) e chiamate \(\alpha_1, \dots, \alpha_m\) le molteplicità rispettive, esiste \(a_n\) tale che
		\begin{equation*}
			\varphi_n (x) = a_n \qty[\prod_{k = 1}^m \qty(x - x_k)^{\alpha_k}] r (x)
		\end{equation*}
		ove si suppone che \(\prod_{k = 1}^m (x - x_k)^{\alpha_k} \equiv 1\) se \(\varphi_n\) non ammette zeri interni ad \((\, a, b \,)\). Per costruzione, \(r\) non ammette zeri in \((\, a, b \,)\) e, dato che è una funzione continua, è di segno costante.
		
		Consideriamo ora il polinomio di grado al piú \(n\)
		\begin{equation*}
			q (x) = \prod_{k = 1}^m \qty(x - x_k)^{\varmod_2 \alpha_k}
		\end{equation*}
		Si osserva che, se uno zero di \(\varphi_n\) interno ad \((\, a, b \,)\) ha molteplicità dispari ma maggiore di \(1\), allora \(\deg q < n\); se uno zero di \(\varphi_n\) interno ad \((\, a, b \,)\) ha molteplicità pari, allora \(\deg q < n\); se uno zero di \(\varphi_n\) è esterno ad \((\, a, b \,)\), allora \(\deg q < n\), perché \(r\) sarebbe di grado almeno \(1\). Ricordiamo, poi, che \(\alpha_k + \varmod_2 \alpha_k\) è pari per ogni \(\alpha_k \in \N\).
		
		Per assurdo sia \(q \in \P_{n - 1}\). Per come sono definiti \(\varphi_n\) e \(q\), si avrebbe
		\begin{equation*}
			\varphi_n (x) \, q(x) = a_n \qty[\prod_{k = 1}^m \qty(x - x_k)^{\alpha_k + \varmod_2 \alpha_k}] r (x)
		\end{equation*}
		il quale avrebbe segno costante e grado almeno \(n\), non coincidendo col polinomio. Dato che, però,
		\begin{multline*}
			0 = \qty(\varphi_n, q)_{2, w} = \int_a^b \varphi_n (x) q (x) w (x) \dd{x} \\
			= \int_a^b a_n \qty[\prod_{k = 1}^m \qty(x - x_k)^{\alpha_k + \varmod_2 \alpha_k}] r (x) w (x) \dd{x} \ne 0
		\end{multline*}
		si è trovato un assurdo. In virtú di ciò, si può concludere.
	\end{proof}

	Non è possibile che \(\varphi_n\) abbia come zeri \(a\) oppure \(b\): nella dimostrazione, infatti, basterebbe mettere in risalto che \(r (x)\) si annulla in \(a\) oppure \(b\), rimanendo comunque di segno costante in \((\, a, b \,)\).
	
	\begin{definizione}
		Un polinomio \(p (x) = \sum_{j = 0}^n a_j x^j\) di grado \(n\) si dice \emph{monico} se \(a_n = 1\).
	\end{definizione}

	\begin{teorema}[Ricorsione a tre termini]
		Se una famiglia triangolare \(\varphi_0, \dots, \varphi_n\) di polinomi monici in \((\, a, b \,)\) è ortogonale rispetto a una funzione peso \(w\), allora, supposti \(\varphi_{-1} (x) = 0\) e \(\varphi_0 (x) = 1\), vale per ogni \(n \in \N^*\)
		\begin{equation}\label{eq:ricorsione-tre-termini}
			\varphi_{n + 1} (x) = (x - \beta_n) \varphi_n (x) - \gamma_n \varphi_{n - 1} (x)
		\end{equation} 
		con
		\begin{align}
			\beta_n &= \frac{(x \varphi_n, \varphi_n)_{2, w}}{(\varphi_n, \varphi_n)_{2, w}} &
			\gamma_n &= \frac{(x \varphi_{n - 1}, \varphi_n)_{2, w}}{(\varphi_{n - 1}, \varphi_{n - 1})_{2, w}}
		\end{align}
	\end{teorema}

	Osserviamo che, scelti i polinomi ortogonali \(\varphi_0\) e \(\varphi_1\), la procedura descritta tramite la \eqref{eq:ricorsione-tre-termini} determina la famiglia triangolare di polinomi ortogonali di grado superiore, dopo aver calcolato \(\beta_k\) e \(\gamma_k\). Se, poi, \(\varphi_n\) verifica \((\varphi_n, \varphi_k)_{2, w} = 0\) per ogni \(k \in \Set{0, \dots, n - 1}\), allora tale condizione è soddisfatta anche da \(\tilde{\varphi}_n = \tau \varphi_n\).
	
	Qualora sia richiesta una famiglia di polinomi \(\hat{\varphi}_0, \dots, \hat{\varphi}_n\) ortonormale, si può partire da una famiglia ortogonale di polinomi monici e ragionare come segue. Dal momento che, infatti, \(\gamma_0 = \int_a^b w (x) \dd{x}\), dalla \eqref{eq:ricorsione-tre-termini} segue che
	\begin{equation*}
		\norm{\varphi_{n + 1}}_{2, w}^2 = \prod_{i = 0}^n \gamma_i \implies \hat{\varphi}_{n + 1} = \frac{\varphi_{n + 1}}{\norm{\varphi_{n + 1}}_{2, w}}
	\end{equation*}
	
\end{document}