% !TeX program = lualatex

\documentclass[11pt]{report}

%% PACCHETTI
\usepackage[T1]{fontenc}
\usepackage{polyglossia}
\usepackage{amsmath}
\usepackage{amssymb}
\usepackage{amsthm}
\usepackage[protrusion = true, expansion]{microtype}
\usepackage{braket}
\usepackage{mathtools}
\usepackage{mathrsfs}
\usepackage{cancel}
\usepackage{caption}
\usepackage{booktabs}
\usepackage{tabularx}
\usepackage{tikz}
\usepackage{pgfplots}
\usepackage[output-decimal-marker={,}, exponent-product={\cdot}]{siunitx}
\usepackage{physics}
\usepackage{matlab-prettifier}
\usepackage{hyperref}

%% DISATTIVA I RIQUADRI DI HYPERREF
\hypersetup{hidelinks}

%% ZONA FONT ALTERNATIVO
%% COMMENTARE QUESTE RIGHE PER RISTABILIRE IL FONT STANDARD
%% È FORTEMENTE CONSIGLIATO USARE 12pt ANZICHÉ 11pt
%% CONTROLLARE GLI EVENTUALI SCONFINAMENTI OLTRE IL MARGINE DEL FOGLIO
%\usepackage{fontspec}
%\usepackage{ebgaramond}
%\usepackage[cmintegrals, cmbraces]{newtxmath}
%\usepackage{ebgaramond-maths}
%% FINE ZONA FONT

%% SIMBOLI MATEMATICI FREQUENTI
\newcommand*{\numberset}{\mathbb}
\newcommand*{\N}{\numberset{N}}
\newcommand*{\R}{\numberset{R}}
\newcommand*{\Q}{\numberset{Q}}
\newcommand*{\Z}{\numberset{Z}}
\newcommand*{\C}{\numberset{C}}
\newcommand*{\T}{\numberset{T}}
\newcommand*{\E}{\numberset{E}}
\newcommand*{\cont}{\mathscr{C}}
\newcommand*{\Parti}{\mathcal{P}}
\newcommand*{\uno}{\mathbf{1}}
\newcommand*{\zero}{\mathbf{0}}
\newcommand*{\ee}{\mathrm{e}\mkern1mu}
\newcommand*{\ii}{\mathrm{i}\mkern1mu}
\newcommand{\tra}[1]{\ensuremath{#1^\mathrm{T}}\relax}
\newcommand{\her}[1]{\ensuremath{#1^\mathrm{H}}\relax}
\newcommand{\gammanum}[1]{\gamma_{#1}^{\textup{\textsc{num}}}}
\newcommand*{\krylov}{\mathcal{K}}

\renewcommand{\vec}{\mathbf}
\renewcommand{\P}{\numberset{P}}

\DeclareMathOperator{\sgn}{sgn}
\DeclareMathOperator{\D}{D}
\DeclareMathOperator{\Jacob}{J}
\DeclareMathOperator{\Hess}{H}
\DeclareMathOperator{\varmod}{mod}
\DeclareMathOperator{\raggio}{\varrho}

%% TEOREMI, LEMMI ET CETERA
\theoremstyle{plain}
\newtheorem{teorema}{Teorema}[chapter]
\newtheorem{proposizione}[teorema]{Proposizione}
\newtheorem{lemma}[teorema]{Lemma}
\newtheorem{corollario}[teorema]{Corollario}

\theoremstyle{definition}
\newtheorem{definizione}[teorema]{Definizione}
\newtheorem{esempio}[teorema]{Esempio}

\theoremstyle{remark}
\newtheorem*{nota}{Nota}
\newtheorem{osservazione}[teorema]{Osservazione}

%% AGGIUSTAMENTI A FIGURE, TABELLE ED EQUAZIONI
\numberwithin{equation}{chapter}
\numberwithin{figure}{chapter}
\numberwithin{table}{chapter}

\captionsetup{tableposition=top,figureposition=bottom,font=small}

%% CONFIGURAZIONI TIKZ-PGFPLOTS
\usetikzlibrary{positioning, decorations.markings, decorations.pathreplacing}
\usetikzlibrary{pgfplots.groupplots}
\pgfplotsset{compat=1.17}
\newcommand*{\drawsinlua}[1]{\directlua{tex.print(tostring(math.sin(#1)))}}

%% Il comando standard per disegnare una parabola sembra rotto se il vertice è oltre il punto finale desiderato.
%% Da Stack Exchange copio questo codice per creare una parabola dati tre punti.
\makeatletter
\def\pt@get#1#2{
	\tikz@scan@one@point\pgfutil@firstofone#2\relax%
	\csname pgf@x#1\endcsname=\pgf@x%
	\csname pgf@y#1\endcsname=\pgf@y%
}
\tikzset{
	parabola through/.style={
		to path={{[x={(\pgf@xc,\pgf@yc)}, y=\parabola@y, shift=(\tikztostart)]
				-- (0,0) .. controls (1/3,1/3) and (2/3,1/3) .. (1,0) \tikztonodes}--(\tikztotarget)}
	},
	parabola through/.prefix code={
		\pt@get{a}{(\tikztostart)}\pt@get{b}{#1}\pt@get{c}{(\tikztotarget)}%
		\advance\pgf@xb by-\pgf@xa\advance\pgf@yb by-\pgf@ya%
		\advance\pgf@xc by-\pgf@xa\advance\pgf@yc by-\pgf@ya%
		\pgfmathsetmacro\parabola@y{(\pgf@yc-\pgf@xc/\pgf@xb*\pgf@yb)%
			/(\pgf@xb-\pgf@xc)*\pgf@xc}%
	}
}
\makeatother

%% IMPOSTAZIONI DI LINGUA
\setmainlanguage{italian}
\setotherlanguage{english}

\newcommand{\inglese}[1]{\textenglish{\emph{#1}}}

\addto\captionsitalian{%
	\renewcommand{\lstlistingname}{Listato}
	\renewcommand{\lstlistlistingname}{Elenco dei listati}}

%% INIZIO DOCUMENTO
\begin{document}
	\begin{titlepage}
		\begin{center}
			\begin{LARGE}
				\textsc{Riccardo Cazzin}
			\end{LARGE}
		\end{center}
		
		\vspace{4.5em}
		
		\begin{center}
			\begin{large}
				Appunti di
			\end{large}
			
			\vspace{1.8em}
			
			\begin{huge}
				\textsc{Analisi Numerica}
			\end{huge}
		\end{center}
		
		\vfill
		
		\begin{center}
			\begin{large}
				estratti dalle lezioni del
			\end{large}
			
			\vspace{1em}
			
			\begin{LARGE}
				\textsc{Prof.\ Alvise Sommariva}
			\end{LARGE}
		\end{center}
		
		\vfill
		
		\begin{center}
			\begin{large}
				\textsc{Versione 0.2}
			\end{large}
		\end{center}
		
		\vfill
		
		\noindent Anno Accademico 2020-2021 \hfill Università degli Studî di Padova
	\end{titlepage}
	
	\tableofcontents
	
	\chapter{Approssimazione in max-norma}

\section{Polinomi di miglior approssimazione}

	\noindent Dato \(n \in \N\), chiamiamo \(\P_n = \Braket{x^i : i \in \Set{0, \dots, n}}\). Da questa definizione si nota subito che \(\P_0 \subset \P_1 \subset \cdots\). Dato che le funzioni polinomiali sono continue, si ottiene
	\begin{equation*}
		\bigcup_{n \in \N} \P_n \subseteq \cont ([\, a, b \,])
	\end{equation*}
	per ogni intervallo chiuso e limitato \([\, a, b \,]\). In questa sezione tratteremo \(\cont ([\, a, b\,])\) come spazio normato dotato della \(\max\)-norma \(\norm{\cdot}_\infty\).
	
	\begin{definizione}\label{def:ins-denso}
		Dato uno spazio topologico \(X\), un insieme \(S \subseteq X\) si dice \emph{denso} in \(X\) se per ogni \(x \in X\) esiste una successione \((s_n)_{n \in \N} \subseteq S\) tale che \(s_n \to x\) per \(n \to \infty\), ovvero se \(\overline{S} = X\). In particolare, se \((X, \norm{\cdot})\) è uno spazio normato, un insieme \(S \subseteq X\) si dice \emph{denso} in \(X\) se per ogni \(x \in X\) e per ogni \(\varepsilon > 0\) esiste \(s \in S\) tale che \(\norm{x - s} < \varepsilon\).
	\end{definizione}

	\begin{teorema}\label{th:errore-ins-denso}
		Dati uno spazio normato \((X, \norm{\cdot})\) e una successione di insiemi \((S_n)_{n \in \N} \subseteq \Parti (X)\) con \(S_0 \ne \varnothing\) e \(S_i \subset S_{i + 1}\) per ogni \(i \in \N\), si definisca
		\begin{equation}\label{eq:errore-approx}
			E_n (f) \coloneqq \inf_{\mathclap{p_n \in S_n}} \norm{p_n - f}
		\end{equation}
		per ogni \(f \in X\). Per qualsiasi \(f \in X\) si verifica \(\lim_{n \to \infty} E_n (f) = 0\) se e solo se \(\bigcup_{n \in \N} S_n\) è denso in \(X\).
	\end{teorema}

	\begin{proof}
		Mostriamo entrambe le implicazioni.
		\begin{description}
			\item[(\(\Rightarrow\))] Siano fissati \(f \in X\) e \(\varepsilon > 0\). Per ipotesi esiste \(n \in \N\) tale che, se \(E_n (f) < \varepsilon\), allora esiste \(p_n \in S_n\) tale che \(\norm{p_n - f} \le \varepsilon\) -- questo è giustificato dalle proprietà dell'estremo inferiore. Da ciò segue che \(\bigcup_{n \in \N} S_n\) è denso in \(X\) per definizione.
			\item[(\(\Leftarrow\))] Fissato \(f \in X\), la successione \(\qty(E_n (f))_{n \in \N}\) è monotona decrescente non negativa, quindi ammette limite; per ogni \(\varepsilon > 0\), quindi, esiste \(p \in \bigcup_{n \in \N} S_n\) tale che \(\norm{p - f} \le \varepsilon\). Scelto \(\bar{n}\) tale che \(p \in S_{\bar{n}}\), si ha \(E_n (f) \le \varepsilon\) per ogni \(n \ge \bar{n}\): per questo motivo \(\lim_{n \to \infty} E_n (f) \le \varepsilon\) e, data l'arbitrarietà di \(\varepsilon\), si conclude che \(\lim_{n \to \infty} E_n (f) = 0\). \qedhere
		\end{description}
	\end{proof}

	\begin{osservazione}
		In base al Teorema~\ref{th:errore-ins-denso}, per ogni \(n \in \N^*\) l'insieme \(A_n = \Set{\norm{p_n - f} : p_n \in S_n} \subseteq \R\) è non vuoto e limitato inferiormente da \(0\), quindi ammette estremo inferiore: la definizione nella \eqref{eq:errore-approx}, dunque, è ben posta.
	\end{osservazione}

	Un caso particolare del Teorema~\ref{th:errore-ins-denso} si ha coi polinomi nei confronti delle funzioni continue su un compatto di \(\R\).

	\begin{teorema}[di approssimazione di Weierstrass]\label{th:weierstrass-approx}
		Ogni funzione di \(\cont ([\, a, b \,])\) con \(a, b \in \R\) e \(a \le b\) è limite uniforme di una successione di polinomi.
	\end{teorema}

	Vogliamo mostrare che, sotto certe condizioni, esiste un elemento di miglior approssimazione -- ovvero l'estremo inferiore è in realtà un minimo.
	
	\begin{lemma}\label{lem:distanza-funzione-continua}
		Dati uno spazio normato \((X, \norm{\cdot})\) e un suo elemento \(f \in X\), se \(S \subseteq X\) è aperto, allora la funzione \(d (f, \cdot) = \norm{f - \cdot}\) è continua in \(S\).
	\end{lemma}

	\begin{proof}
		Osserviamo innanzitutto che, scelti \(x, y \in X\) qualunque, si verifica che \(\abs{\norm{x} - \norm{y}} \le \norm{x - y}\): supponendo, infatti, \(\norm{x} \ge \norm{y}\), si ha
		\begin{equation*}
			\norm{x} = \norm{x - y + y} \le \norm{x - y} + \norm{y} \iff 0 \le \norm{x} - \norm{y} \le \norm{x - y}
		\end{equation*}
		e, scambiando i ruoli di \(x\) e \(y\), si è provata l'osservazione.
		
		Fissato \(\varepsilon > 0\), poniamo \(\delta = \varepsilon\) e scegliamo \(x, y \in S\) tali che \(\norm{x - y} \le \delta\): si ha
		\begin{multline*}
				\abs{d (f, x) - d (f, y)} = \abs{\norm{f - x} - \norm{f - y}} \\
				\le \norm{(f - x) - (f - y)} = \norm{x - y} \le \delta = \varepsilon
		\end{multline*}
		il che permette di concludere.
	\end{proof}

	\begin{teorema}\label{th:miglior-approx-esiste}
		Dato uno spazio normato \((X, \norm{\cdot})\), sia \(f \in X\). Se un sottospazio vettoriale \(S \le X\) è di dimensione finita, allora esiste \(s^* \in S\) tale che
		\begin{equation*}
			\norm{f - s^*} = \min_{s \in S} \norm{f - s}
		\end{equation*}
	\end{teorema}

	\begin{proof}
		Visto che \(0_X \in S\) per ogni \(S \le X\), si ha
		\begin{equation*}
			E_n (f) = \inf_{p \in S} \norm{f - p} \le \norm{f - 0_X} = \norm{f}
		\end{equation*}
		ovvero che \(E_n (f) \in [\, 0, \norm{f} \,]\). Poiché per il Lemma~\ref{lem:distanza-funzione-continua} la funzione \(d (f, \cdot)\) è continua su \(S\) e dato che \(B_{\norm{f}} (f) \cap S\) è un insieme compatto di \(S\), per il teorema di Weierstrass \(d (f, \cdot)\) ammette minimo in tale insieme, ovvero esiste \(s^* \in B_{\norm{f}} (f) \cap S \subseteq S\) tale che \(E_n (f) = \norm{f - s^*}\).
	\end{proof}

	Limitando questo risultato al caso delle funzioni continue su un compatto di \(\R\), è provata l'esistenza di un polinomio di miglior approssimazione.

	\begin{corollario}
		Per ogni \(k \in \N\) e per ogni \(f \in \cont ([\, a, b \,])\) con \(a, b \in \R\) e \(a \le b\) esiste un polinomio \(p_k^* \in \P_n\) di miglior approssimazione relativamente alla \(\max\)-norma \(\norm{\cdot}_{\infty}\).
	\end{corollario}

	 Quanto visto non dimostra l'unicità dell'elemento di miglior approssimazione. Nel caso delle funzioni continue su un compatto di \(\R\), tuttavia, vale il seguente risultato, che riportiamo senza dimostrazione.
	 
	 \begin{teorema}[di equioscillazione di Chebyshev]\label{th:chebyshev-equiosc}
	 	Dato un intervallo chiuso e limitato \([\, a, b \,]\), per ogni \(f \in \cont ([\, a, b \,])\) e per ogni \(n \in \N\) esiste un unico \(p_n^* \in \P_n\) di miglior approssimazione. Esistono, inoltre, \(\sigma \in \Set{- 1, 1}\) e \(n + 2\) punti \(a \le x_0 < \cdots < x_{n + 1} \le b\) tali che per ogni \(j \in \Set{0, \dots, n + 1}\) si abbia
	 	\begin{equation}
	 		f (x_j) - p_n^* (x_j) = \sigma (-1)^j \norm{f - p_n^*}_{\infty}
	 	\end{equation}
	 \end{teorema}
 
 	\begin{figure}[tpb]
 		\centering
 		
 		\begin{tikzpicture}
 			\begin{axis}[scale = 1.3, samples = 400, axis lines = center, domain=0:10, ymin = -1.5, ymax = 1.5, xmax = 10.2, no marks, unit vector ratio = 1 2, legend entries = {\(\sin x\), \(p_5^* (x)\)}, legend pos = outer north east, xtick = {0, 2, ..., 10}]
 				\addplot {sin(deg(x))};
 				\addplot+[black, dashed, thin, forget plot] {sin(deg(x)) + 0.25945059765721};
 				\addplot+[black, dashed, thin, forget plot] {sin(deg(x)) - 0.25945059765721};
 				\addplot+[red] table {risorse/sin.dat};
 			\end{axis}
 		\end{tikzpicture}
 	
 		\caption{Confronto tra la funzione \(\sin x\) e il suo polinomio di miglior approssimazione di quinto grado nell'intervallo \([\, 0, 10 \,]\). Si notino le sette intersezioni di \(p_5^* (x)\) con \(\sin (x) \pm E_5 (\sin)\).}\label{fig:sin-migl-approx}
 	\end{figure}
 
 	Benché ne siano garantite l'esistenza e l'unicità, il polinomio di miglior approssimazione richiede un algoritmo abbastanza complicato, messo a punto da Remez, per essere trovato. Come è possibile notare nella Tabella~\ref{tab:remez-errore}, funzioni con regolarità diverse richiedono un grado diverso del polinomio di miglior interpolazione per essere approssimate con un certo errore assoluto in \(\max\)-norma. I teoremi di Jackson danno un'idea piú rigorosa dell'influenza della regolarità delle funzioni su tale errore.
 	
 	\begin{table}[tpb]
 		\centering
 		
 		\caption{Errore in \(\max\)-norma commesso dal polinomio di miglior interpolazione di alcune funzioni definite su \([\, -5, 5 \,]\), al variare del grado.}\label{tab:remez-errore}
 		
 		\begin{tabular}{SSSS}
 			\toprule
 			{\(n\)} & {\(\displaystyle E_n \qty(\frac{1}{1 + x^2})\)} & {\(E_n (\abs{x - 4})\)} & {\(E_n (\sin x)\)} \\
 			\midrule
			\num{5}   & \num{2.171584e-01} & \num{1.612220e-01} & \num{1.078946e-01} \\
			\num{10}  & \num{6.592293e-02} & \num{8.398083e-02} & \num{7.031736e-04} \\
			\num{15}  & \num{2.977669e-02} & \num{5.677524e-02} & \num{2.306162e-08} \\
			\num{20}  & \num{9.039331e-03} & \num{4.279954e-02} & \num{6.690029e-12} \\
			\num{25}  & \num{4.082970e-03} & \num{3.431831e-02} & \num{1.355775e-15} \\
			\num{30}  & \num{1.239470e-03} & \num{2.863059e-02} & \num{\le e-16} \\
			\num{35}  & \num{5.598556e-04} & \num{2.455346e-02} & \num{\le e-16} \\
			\num{40}  & \num{1.699558e-04} & \num{2.148833e-02} & \num{\le e-16} \\
			\num{45}  & \num{7.676723e-05} & \num{1.910021e-02} & \num{\le e-16} \\
			\num{50}  & \num{2.330428e-05} & \num{1.718714e-02} & \num{\le e-16} \\
			\num{55}  & \num{1.052630e-05} & \num{1.562016e-02} & \num{\le e-16} \\
			\num{60}  & \num{3.195476e-06} & \num{1.431308e-02} & \num{\le e-16} \\
			\num{65}  & \num{1.443363e-06} & \num{1.320615e-02} & \num{\le e-16} \\
			\num{70}  & \num{4.381627e-07} & \num{1.225662e-02} & \num{\le e-16} \\
			\num{75}  & \num{1.979134e-07} & \num{1.143309e-02} & \num{\le e-16} \\
			\num{80}  & \num{6.008073e-08} & \num{1.071203e-02} & \num{\le e-16} \\
			\num{85}  & \num{2.713784e-08} & \num{1.007540e-02} & \num{\le e-16} \\
			\num{90}  & \num{8.238251e-09} & \num{9.509177e-03} & \num{\le e-16} \\
			\num{95}  & \num{3.721132e-09} & \num{9.002281e-03} & \num{\le e-16} \\
			\num{100} & \num{1.129627e-09} & \num{8.545846e-03} & \num{\le e-16} \\
			\bottomrule
		\end{tabular}
	\end{table}

 	\begin{definizione}
 		Di una funzione \(f \colon [\, a, b \,] \to \R\) si dice \emph{modulo di continuità} la quantità
		\begin{equation}\label{eq:modulo-contin}
 			\omega (f, \delta) = \sup_{\mathclap{\substack{x, y \in [\, a, b \,] \\ \abs{x - y} \le \delta}}} \abs{f (x) - f (y)}
 		\end{equation}
	\end{definizione}

	La misura di continuità di una funzione è tanto maggiore quanto piú la funzione “oscilla”. Se la funzione in esame ha una qualche regolarità, la sua misura di continuità può essere stimata in modo semplice.
	
	\begin{esempio}[Misura di continuità per funzioni particolari]
		Se una funzione \(f \colon [\, a, b \,] \to \R\) è lipschitziana con costante di Lipschitz \(L\), allora
		\begin{equation*}
			\omega (f, \delta) = \sup_{\mathclap{\substack{x, y \in [\, a, b \,] \\ \abs{x - y} \le \delta}}} \abs{f (x) - f (y)} \le \sup_{\mathclap{\substack{x, y \in [\, a, b \,] \\ \abs{x - y} \le \delta}}} L \abs{x - y} = L \delta
		\end{equation*}
		Se \(f \in \cont^1 ([\, a, b\,])\), allora è anche lipschitziana ed ha costante di Lipschitz pari a \(L = \max_{x \in [\, a, b \,]} \abs{f' (x)}\).
		
		Se \(f\) è h\"olderiana con costante di H\"older \(\alpha \in (\, 0, 1 \,)\), ovvero esiste \(L \ge 0\) tale che \(\abs{f (x) - f (y)} \le L \abs{x - y}^\alpha\) per ogni \(x, y \in [\, a, b \,]\), allora
		\begin{equation*}
			\omega (f, \delta) = \sup_{\mathclap{\substack{x, y \in [\, a, b \,] \\ \abs{x - y} \le \delta}}} \abs{f (x) - f (y)} \le \sup_{\mathclap{\substack{x, y \in [\, a, b \,] \\ \abs{x - y} \le \delta}}} L \abs{x - y}^\alpha = L \delta^\alpha
		\end{equation*}
	\end{esempio}

	\begin{teorema}[Jackson]\label{th:jackson}
		Per ogni \(n \in \N^*\) e per ogni \(f \in \cont ([\, a, b \,])\) esiste una costante \(M \ge 0\) indipendente da \(a, b, n\) tale che
		\begin{equation}\label{eq:jackson}
			E_n (f) = \min_{p \in \P_n} \norm{f - p}_\infty \le M \omega \qty(f, \frac{b - a}{n})
		\end{equation}
	\end{teorema}

	\begin{corollario}\label{cor:jackson-lip-hold}
		Se alle ipotesi del Teorema~\ref{th:jackson} si aggiunge che \(f\) è lipschitziana, allora esiste \(M^*\) indipendente da \(a, b, n\) tale che
		\begin{equation}\label{eq:jackson-lip}
			E_n (f) \le M^* \frac{b - a}{n}
		\end{equation}
		Se, invece, si suppone che \(f\) sia h\"olderiana con costante di H\"older \(\alpha\), allora esiste \(\bar{M}\) indipendente da \(a, b, n\) tale che
		\begin{equation}\label{eq:jackson-hold}
			E_n (f) \le \bar{M} \qty(\frac{b - a}{n})^\alpha
		\end{equation}
	\end{corollario}

	\begin{teorema}[Jackson]\label{th:jackson-ck}
		Per ogni funzione \(f \in \cont^k ([\, a, b \,])\) con \(k \in \N\) e per ogni \(n > k\) esiste \(M \ge 0\) tale che
		\begin{equation}\label{eq:jackson-ck}
			E_n (f) \le M^{k + 1} \frac{(b - a)^k}{\prod_{i = 0}^{k - 1} (n - i)} \, \omega \qty(f^{(k)}, \frac{b - a}{n - k})
		\end{equation}
	\end{teorema}

	\begin{corollario}\label{cor:jackson-ck-hold}
		Se alle ipotesi del Teorema~\ref{th:jackson-ck} si aggiunge che \(f\) è h\"olderiana con costante di H\"older \(\alpha\) e costante \(L\) e che \(k > 0\), allora esiste \(M \ge 0\) tale che
		\begin{equation}\label{eq:jackson-ck-hold}
			E_n (f) \le L M^{k + 1} \frac{(b - a)^k}{\prod_{i = 0}^{k - 1} (n - i)} \qty(\frac{b - a}{n - k})^\alpha
		\end{equation}
	\end{corollario}

	\begin{teorema}[Jackson]\label{th:jackson-hold-d}
		Se una funzione \(f \in \cont^k ([\, a, b \,])\) è \(\alpha\)-h\"olderiana di costante \(M\), allora esiste una costante \(d_k\) indipendente da \(f\) e da \(n \in \N^*\) tale che
		\begin{equation}\label{eq:jackson-hold-d}
			E_n (f) \le \frac{M d_k}{n^{k + \alpha}}
		\end{equation}
	\end{teorema}

	Ricordiamo che una funzione complessa \(f \colon \varOmega \to \C\), con \(\varOmega\) regione del piano complesso, si dice \emph{analitica} in un punto \(z_0 \in \varOmega\) se esiste un intorno di \(z_0\) in \(\varOmega\) tale che per ogni \(z\) in tale intorno si abbia \(f (z) = \sum_{n = 0}^\infty a_n (z - z_0)^n\); tale funzione si dice \emph{analitica in \(\varOmega\)} se è analitica in ogni punto di \(\varOmega\).
	
	\begin{teorema}\label{th:errore-approx-analitica}
		Se una funzione \(f \colon [\, a, b \,] \to \R\) è analitica in un aperto \(\varOmega \subseteq \C\) che contenga \([\, a, b \,]\), allora esiste \(\vartheta \in (\, 0, 1 \,)\) tale che
		\begin{equation}\label{eq:errore-approx-analitica}
			E_n (f) = \order{\vartheta^n}
		\end{equation}
	\end{teorema}

	Se una funzione è analitica su \(\C\), ovvero se è \emph{intera}, l'errore commesso approssimando tale funzione col suo polinomio di miglior approssimazione decade piú che esponenzialmente.
	
	\begin{teorema}[Bernstein]\label{th:bernstein}
		Data una funzione \(f \colon [\, a, b \,] \to \R\), si ha
		\begin{equation}\label{eq:bernstein}
			\lim_{n \to \infty} \sqrt[n]{E_n (f)} = 0 \iff \text{\(f\) intera}
		\end{equation}
	\end{teorema}

	\begin{osservazione}
		Per quanto visto, la funzione di Runge \(f (x) \coloneqq \frac{1}{1 + x^2}\) è analitica in un aperto di \(\C\) che contenga \([\, -5, 5 \,]\). Una verifica sperimentale coi dati nella Tabella~\ref{tab:remez-errore} mostra che \(\vartheta \approx \num{0.814}\). Poiché, invece, la funzione \(\sin z\) è intera, l'errore di approssimazione con l'algoritmo di Remez scende alla precisione di macchina già per \(n\) non troppo alti.
		
		Per quanto riguarda la funzione \(f (x) \coloneqq \abs{x - 4}\), che è lipschitziana, dal Corollario~\ref{cor:jackson-lip-hold} segue che esiste \(M\) tale che \(E_n (f) = 10 M / n = \order{1 / n}\); con una verifica sperimentale si vede che \(E_n (f) \approx \num{0.85} / n\), ovvero che la convergenza del polinomio di miglior approssimazione è molto lenta.
	\end{osservazione}

\section{Polinomi di Chebyshev}
	
	\noindent Dato \(n \in \N\), consideriamo per \(x \in [\, -1, 1 \,]\) la funzione
	\begin{equation}\label{eq:polin-cheb-def}
		T_n (x) = \cos (n \arccos x)
	\end{equation}
	\emph{A priori} tale funzione può non essere un polinomio, ma si nota che
	\begin{gather*}
		T_0 (x) = \cos (0 \cdot \arccos x) = 1 \\
		T_1 (x) = \cos (1 \arccos x) = x
	\end{gather*}
	Dalle formule trigonometriche di addizione e sottrazione
	\begin{gather*}
		\cos ((n + 1) \vartheta) = \cos (n \vartheta) \cos \vartheta - \sin (n \vartheta) \sin \vartheta \\
		\cos ((n - 1) \vartheta) = \cos (n \vartheta) \cos \vartheta + \sin (n \vartheta) \sin \vartheta \\
	\end{gather*}
	si ottiene, sommando membro a membro,
	\begin{equation*}
		\cos ((n + 1) \vartheta) + \cos ((n - 1) \vartheta) = 2 \cos(n \vartheta) \cos \vartheta
	\end{equation*}
	Se ora si pone \(\vartheta = \arccos x\), si trova
	\begin{equation}\label{eq:polin-cheb-formula}
		T_{n + 1} (x) = 2 x \, T_n (x) - T_{n - 1} (x)
	\end{equation}
	e, dato che per \(n \in \Set{0, 1}\), i polinomi di Chebyshev corrispondenti sono effettivamente polinomi in senso classico, per ricorrenza \(\Set{T_n (x) : n \in \N}\) è una successione di polinomi, ove \(T_n (x)\) è di grado \(n\). Se \(n > 0\), il coefficiente del termine \(x^n\) è \(2^{n - 1}\).
	
	Troviamo ora gli zeri di un polinomio di Chebyshev \(T_n (x)\): essi sono i punti \(x_k\) tali che \(\cos (n \arccos x_k) = 0\); ricordando che il dominio della funzione \(\arccos\) è \([\, 0, \pi \,]\), gli zeri soddisfano
	\begin{equation*}
		n \arccos x_k = \frac{\pi}{2} + k \pi = \frac{(2 k + 1) \pi}{2}
	\end{equation*}
	ovvero
	\begin{equation*}
		\arccos x_k = \frac{(2 k + 1) \pi}{2 n}
	\end{equation*}
	e, applicando la funzione coseno ad ambo i membri, si ottiene per ogni \(k \in \Set{0, \dots, n - 1}\)
	\begin{equation}\label{eq:polin-cheb-zeri}
		x_k = \cos \qty(\frac{(2 k + 1) \pi}{2 n})
	\end{equation}
	Notiamo che gli zeri del polinomio di Chebyshev di grado \(n\) sono \(n\) punti distinti dell'intervallo aperto \((\, -1, 1 \,)\).
	
\section{Costanti di Lebesgue}
	
	\noindent Dati un intervallo chiuso e limitato \([\, a, b \,]\) e una funzione \(f \in \cont ([\, a, b \,])\), si consideri il polinomio \(p_n \in \P_n\) che interpola le \(n + 1\) coppie a due a due distinte \((x_k, f_k)\), con \(f_k = f (x_k)\) e \(k \in \Set{0, \dots, n}\). Definito per ogni \(k \in \Set{0, \dots, n}\) il \(k\)-esimo \emph{polinomio di Lagrange}
	\begin{equation}\label{eq:polin-lagrange}
		L_k (x) = \prod_{\mathclap{\substack{j = 0 \\ j \ne k}}}^n \frac{x - x_j}{x_k - x_j}
	\end{equation}
	è noto che
	\begin{equation}\label{eq:polin-interp}
		p_n (x) = \sum_{k = 0}^n f_k L_k (x)
	\end{equation}
	Se i valori \(f_k\) sono sostituiti con valori perturbati \(\tilde{f}_k\), allora, posto \(\tilde{p}_n (x) = \sum_{k = 0}^n \tilde{f}_k L_k (x)\), si ha
	\begin{multline*}
		\abs{p_n (x)  - \tilde{p}_n (x)} = \abs{\sum_{k = 0}^n (f_k - \tilde{f}_k) L_k (x)} \le \sum_{k = 0}^n \abs{f_k - \tilde{f}_k} \, \abs{L_k (x)} \\
		\le \qty(\max_{k \in \Set{0, \dots, n}} \abs{f_k - \tilde{f}_k}) \sum_{k = 0}^n \abs{L_k (x)}
	\end{multline*}
	da cui segue che
	\begin{equation*}
		\max_{x \in [\, a, b \,]} \abs{p_n (x)  - \tilde{p}_n (x)} \le \qty(\max_{k \in \Set{0, \dots, n}} \abs{f_k - \tilde{f}_k}) \max_{x \in [\, a, b \,]} \sum_{k = 0}^n \abs{L_k (x)}
	\end{equation*}
	Definita la quantità
	\begin{equation}\label{eq:cost-lebesgue}
		\varLambda_n \coloneqq \max_{x \in [\, a, b \,]} \sum_{k = 0}^n \abs{L_k (x)}
	\end{equation}
	si ricava la stima
	\begin{equation}\label{eq:errore-cost-lebesgue}
		\norm{p_n - \tilde{p}_n}_\infty \le \qty(\max_{k \in \Set{0, \dots, n}} \abs{f_k - \tilde{f}_k}) \varLambda_n
	\end{equation}

	Osserviamo che \(\varLambda_n\), detta \emph{costante di Lebesgue} dell'insieme di punti \(x_0, \dots, x_n\), dipende esclusivamente dai polinomi di Lagrange e, quindi, dai soli punti di interpolazione. È chiaro che ciò rende \(\varLambda_n\) un indice di stabilità dell'interpolazione di Lagrange.
	
	Si può mostrare che, se \(\mathcal{L}_n\) è l'operatore lineare e limitato che associa ad ogni funzione \(f \in \cont([\, a, b \,])\) il suo polinomio interpolatore \(p_n\) nei punti \(x_0, \dots, x_n\), allora
	\begin{equation}\label{eq:norma-interpolazione-lebesgue}
		\varLambda_n = \norm{\mathcal{L}_n}_\infty = \max_{\substack{g \in \cont ([\, a, b \,]) \\ g \ne 0}} \frac{\norm{\mathcal{L}_n (g)}_\infty}{\norm{g}_\infty}
	\end{equation}
	
	\begin{teorema}\label{th:errore-interpolaz-appross}
		Se \(p_n \in \P_n\) è il polinomio interpolatore di una funzione \(f \in \cont([\, a, b \,])\) relativo ai punti \(x_0, \dots, x_n\), allora
		\begin{equation}\label{eq:errore-interpolaz-appross}
			\norm{f - p_n}_\infty \le (1 + \varLambda_n) E_n (f)
		\end{equation}
	\end{teorema}

	\begin{proof}
		Se \(f \in \P_n\), allora \(f = p_n = p_n^*\) e l'asserto è banalmente verificato. Supponiamo ora \(f \notin \P_n\); si ha, quindi, \(f - q_n \ne 0\) per ogni \(q_n \in \P_n\). Dal momento che \(q_n \in \P_n\), si ha \(\mathcal{L}_n (q_n) = q_n\) per l'unicità del polinomio interpolatore e per il principio d'identità dei polinomi. Per la linearità di \(\mathcal{L}_n\), poi, si ha
		\begin{equation*}
			\mathcal{L}_n (f - q_n) = \mathcal{L}_n (f) - \mathcal{L}_n (q_n) = p_n - q_n
		\end{equation*}
		Da ciò e dalla \eqref{eq:norma-interpolazione-lebesgue} segue che
		\begin{equation*}
			\varLambda_n = \max_{\substack{g \in \cont ([\, a, b \,]) \\ g \ne 0}} \frac{\norm{\mathcal{L}_n (g)}_\infty}{\norm{g}_\infty} \ge \frac{\norm{\mathcal{L}_n (f - q_n)}_\infty}{\norm{f - q_n}_\infty} = \frac{\norm{p_n - q_n}_\infty}{\norm{f - q_n}_\infty}
		\end{equation*}
		e, quindi, per ogni \(q_n \in \P_n\) vale \(\norm{p_n - q_n}_\infty \le \varLambda_n \norm{f - q_n}_\infty\). Applicando la diseguaglianza triangolare, si trova
		\begin{multline*}
			\norm{f - p_n}_\infty = \norm{(f - q_n) + (q_n - p_n)}_\infty \le \norm{f - q_n}_\infty + \norm{q_n - p_n}_\infty \\
			\le \norm{f - q_n}_\infty + \varLambda_n \norm{f - q_n}_\infty = (1 + \varLambda_n) \norm{f - q_n}_\infty
		\end{multline*}
		da cui segue la \eqref{eq:errore-interpolaz-appross} scegliendo \(q_n = p_n^*\).
	\end{proof}

	Una conseguenza “informale” del Teorema~\ref{th:errore-interpolaz-appross} è che approssimare una funzione col suo polinomio interpolatore comporta un errore simile a quello compiuto approssimando col polinomio di miglior approssimazione se \(\varLambda_n\) è abbastanza piccola.
	
	\begin{esempio}
		Di alcune suddivisioni dell'intervallo \([\, -1, 1 \,]\) conosciamo il comportamento asintotico di \(\varLambda_n\) al crescere di \(n\). Se si usano \(n + 1\) punti equispaziati, si mostra che
		\begin{equation*}
			\varLambda_n \sim \frac{2^{n + 1}}{\ee n \log n}
		\end{equation*}
		Se invece si usano i punti di Chebyshev, di forma \(x_k = \cos \qty(\frac{2 k - 1}{2 (n + 1)} \pi)\) con \(k \in \Set{1, \dots, n + 1}\), si trova l'andamento asintotico
		\begin{equation*}
			\varLambda_n = \frac{2}{\pi} \qty[\log (n + 1) + \gamma + \log \qty(\frac{8}{\pi})] + \order{\frac{1}{(n + 1)^2}}
		\end{equation*}
		ove \(\gamma \approx \num{0.5772156649}\) è la costante di Eulero-Mascheroni.
		
		Se si usano i punti di Chebyshev estesi, di forma \(x_k = \frac{\cos \qty(\frac{2 k - 1}{2 (n + 1)} \pi)}{\cos \qty(\frac{1}{2 (n + 1)} \pi)}\) con \(k \in \Set{1, \dots, n + 1}\), si trova l'andamento asintotico
		\begin{equation*}
			\varLambda_n = \frac{2}{\pi} \qty[\log (n + 1) + \gamma + \log \qty(\frac{8}{\pi}) - \frac{2}{3}] + \order{\frac{1}{\log(n + 1)}}
		\end{equation*}
		Si può dimostrare, poi, che il minimo andamento asintotico per la costante di Lebesgue deve valere
		\begin{equation*}
			\varLambda_n = \frac{2}{\pi} \qty[\log (n + 1) + \gamma + \log \qty(\frac{8}{\pi})] + \order{\frac{\log \log (n + 1)}{\log (n + 1)}}
		\end{equation*}
	\end{esempio}

	\begin{figure}[tpb]
		\centering
		
		\begin{tikzpicture}
			\begin{axis}[ymode=log, domain=5:50, only marks, legend entries = {\(\varLambda_n^{\textup{equi}}\), \(\varLambda_n^{\textup{Cheb}}\)}, legend pos = outer north east]
				\addplot table {risorse/lebequi.dat};
				\addplot table {risorse/lebcheb.dat};
			\end{axis}
		\end{tikzpicture}
		
		\caption{Confronto tra gli andamenti asintotici di \(\varLambda_n\) usando punti equispaziati e punti di Chebyshev.}\label{fig:lebesgue-equisp-cheb}
	\end{figure}

	
	\chapter{Approssimazione in spazi euclidei}

\section{Richiami sugli spazi euclidei}

	\begin{definizione}[Spazio euclideo reale]
		Si dice \emph{spazio euclideo reale} uno spazio vettoriale \(\E\) su \(\R\) dotato di un'applicazione \((\cdot, \cdot) \colon \E \times \E \to \R\) tale che
		\begin{subequations}
			\begin{gather}
				\forall x \in \E \colon (x, x) \ge 0 \\
				\forall x \in \E \colon (x, x) = 0 \iff x = 0_\E \\
				\forall x, y \in \E \colon (x, y) = (y, x) \\
				\forall \lambda \in \R \colon \forall x, y \in \E \colon (\lambda x, y) = \lambda (x, y) \\
				\forall x, y, z \in \E \colon (x + y, z) = (x, z) + (y, z)
			\end{gather}
		\end{subequations}
	\end{definizione}
	
	\begin{definizione}[Spazio euclideo complesso]
		Si dice \emph{spazio euclideo complesso} uno spazio vettoriale \(\E\) su \(\C\) dotato di un'applicazione \((\cdot, \cdot) \colon \E \times \E \to \C\) tale che
		\begin{subequations}
			\begin{gather}
				\forall x \in \E \colon (x, x) \ge 0 \\
				\forall x \in \E \colon (x, x) = 0 \iff x = 0_\E \\
				\forall x, y \in \E \colon (x, y) = \overline{(y, x)} \\
				\forall \lambda \in \C \colon \forall x, y \in \E \colon (\lambda x, y) = \lambda (x, y) \\
				\forall \lambda \in \C \colon \forall x, y \in \E \colon (x, \lambda y) = \overline{\lambda} (x, y) \\
				\forall x, y, z \in \E \colon (x, y + z) = (x, y) + (x, z)
			\end{gather}
		\end{subequations}
	\end{definizione}

	A partire da queste due definizioni si può definire lo spazio normato \((\E, \norm{\cdot})\) dotato della norma \(\norm{f}_2 = \sqrt{(f, f)}\).
	
	\begin{osservazione}
		Si nota facilmente che
		\begin{gather*}
			\qty(x, \sum_{k = 1}^n y_k) = \sum_{k = 1}^n (x, y_k) \\
			\qty(\sum_{k = 1}^n x_k, y) = \overline{\qty(y, \sum_{k = 1}^n x_k)} = \overline{\sum_{k = 1}^n (y, x_k)} = \sum_{k = 1}^n \overline{(y, x_k)} = \sum_{k = 1}^n (x_k, y)
		\end{gather*}
	\end{osservazione}	

	\begin{esempio}
		\(\R^n\) dotato del prodotto scalare usuale è uno spazio euclideo; se \(e_1, \dots, e_n\) è una sua base ortonormale, ovvero si verifica \((e_j, e_k) = \delta_{j, k}\) per ogni \(j, k \in \Set{1, \dots, n}\), allora per ogni \(x \in \R^n\) vale la scrittura unica \(x = \sum_{k = 1}^n c_k e_k\), con \(c_k = (x, e_k)\).
		
		Lo spazio \(\cont ([\, a, b \,])\) delle funzioni continue su un compatto \([\, a, b \,]\) dotato del prodotto scalare \((f, g) = \int_a^b f (x) g (x) \dd{x}\) è uno spazio euclideo.
		
		Lo spazio \(L_\R^2 ([\, a, b \,])\) delle funzioni reali misurabili su un compatto \([\, a, b \,]\) e di modulo al quadrato integrabile dotato del prodotto scalare \((f, g) = \int_a^b f (x) g (x) \dd{x}\) è uno spazio euclideo completo, ovvero tale che ogni successione di Cauchy è convergente.
		
		Lo spazio \(L_\C^2 ([\, a, b \,])\) delle funzioni complesse misurabili su un compatto \([\, a, b \,]\) e di modulo al quadrato integrabile dotato del prodotto scalare \((f, g) = \int_a^b f (x) \overline{g (x)} \dd{x}\) è uno spazio euclideo completo.
	\end{esempio}
	
	\begin{teorema}[Pitagora]\label{th:pitagora}
		Dato uno spazio euclideo \(\E\), se \(f, g \in \E\) verificano \((f, g) = 0\), allora \(\norm{f + g}_2^2 = \norm{f}_2^2 + \norm{g}_2^2\).
	\end{teorema}
	
	\begin{proof}
		Con un conto diretto si vede che
		\begin{equation*}
			\begin{split}
				\norm{f + g}_2^2 = (f + g, f + g) &= (f, f) + (f, g) + (g, f) + (g, g) \\
				&= \norm{f}_2^2 + 0 + 0 + \norm{g}_2^2 \\
				&= \norm{f}_2^2 + \norm{g}_2^2\qedhere
			\end{split}
		\end{equation*}
	\end{proof}

	\begin{teorema}[Proiezione ortogonale]\label{th:proiez-ortog}
		Dato uno spazio euclideo complesso \(\E\), sia \(f \in \E\); se \(\Set{\varphi_j | i \in \Set{1, \dots, N}} \subset \E\) è un insieme finito di elementi linearmente indipendenti, allora \(f^* = \sum_{j = 1}^N c_j^* \varphi_j\), ove i \(c_j^* \!\) soddisfano le \emph{equazioni normali}
		\begin{equation}\label{eq:eq-normali}
			\forall j \in \Set{1, \dots, N} \colon \sum_{k = 1}^N (\varphi_j, \varphi_k) c_k^* = (\varphi_j, f)
		\end{equation}
		è tale che\label{eq:proiez-ortog}
		\begin{equation}
			\norm{f - f^*}_2 = \min_{g \in \Braket{\varphi_1, \dots, \varphi_k}} \norm{f - g}_2
		\end{equation}
		Questa soluzione è tale che \(f^* - f\) è ortogonale a tutti i \(\varphi_j\), o equivalentemente si ha \((f, \varphi_j) = (f^*, \varphi_j)\) per ogni \(j \in \Set{1, \dots, N}\).
	\end{teorema}

	\begin{proof}
		Mostriamo innanzitutto che la soluzione di miglior approssimazione è unica. Se \(f^* \in \E\) è tale che \(f^* - f \in \Braket{\varphi_1, \dots, \varphi_N}^\perp\) e \(\hat{f} \ne f^*\) è elemento di miglior approssimazione nel senso della \eqref{eq:proiez-ortog}, allora \(f^* - \hat{f} \in \Braket{\varphi_1, \dots, \varphi_N}\); per il Teorema~\ref{th:pitagora} si ha
		\begin{equation*}
			\norm{f - \hat{f}}_2^2 = \norm{(f - f^*) + (f^* - \hat{f})}_2^2 = \norm{f - f^*}_2^2 + \norm{f^* - \hat{f}}_2^2 > \norm{f - f^*}_2^2
		\end{equation*}
		il che è contro l'ipotesi che \(\hat{f}\) sia di miglior approssimazione.
		
		Rimane da mostrare l'esistenza di \(f^*\), ovvero di \(c_j^*\) che soddisfino per ogni \(k \in \Set{1, \dots, N}\)
		\begin{equation*}
			\begin{split}
				0 = \qty(\sum_{j = 1}^N c_j^* \varphi_j - f, \varphi_k) &= \qty(\sum_{j = 1}^N c_j^* (\varphi_j, \varphi_k)) - (f, \varphi_k) \\
				&= \sum_{j = 1}^N (\varphi_j, \varphi_k) c_j^* - (f, \varphi_k)
			\end{split}
		\end{equation*}
		Questa condizione, equivalente di fatto alla \eqref{eq:eq-normali}, è soddisfatta se e solo se la matrice \(G\) che ha come coordinata \((j, k)\) il prodotto interno \((\varphi_j, \varphi_k)\) è non singolare. Certamente \(G\) è hermitiana, in quanto \((\varphi_j, \varphi_k) = \overline{(\varphi_k, \varphi_j)}\); per questo motivo si ha per ogni \(v \in \C^N \setminus \Set{0_{\C^N}}\)
		\begin{equation*}
			\her{\qty(\her{v} G v)} = \her{v} \her{G} \her{\qty(\her{v})} = \her{v} G v \iff \her{v} G v \in \R
		\end{equation*}
		Scelto ora \(v = \sum_{j = 1}^N v_j \varphi_j \ne 0\) e definito \(u = \sum_{j = 1}^N \overline{v}_j \varphi_j \ne 0\), si ha
		\begin{equation*}
			\begin{split}
				\her{v} G v &= \sum_{j = 1}^N \overline{v}_j \sum_{k = 1}^N G_{j, k} v_k \\
				&= \sum_{j = 1}^N \overline{v}_j \sum_{k = 1}^N (\varphi_j, \varphi_k) v_k = \sum_{j = 1}^N \overline{v}_j \sum_{k = 1}^N (\varphi_j, \overline{v}_k \varphi_k) \\
				&= \sum_{j = 1}^N \overline{v}_j \qty(\varphi_j, \sum_{k = 1}^N \overline{v}_k \varphi_k) = \sum_{j = 1}^N \qty(\overline{v}_j \varphi_j, \sum_{k = 1}^N \overline{v}_k \varphi_k) \\
				&= \qty(\sum_{j = 1}^N \overline{v}_j \varphi_j, \sum_{k = 1}^N \overline{v}_k \varphi_k) = (u, u) = \norm{u}_2^2 > 0
			\end{split}
		\end{equation*}
		da cui è evidente che \(G\) è definita positiva e, quindi, non singolare.
	\end{proof}

	\begin{osservazione}
		Se l'insieme \(\Set{\varphi_1, \dots, \varphi_N}\) non è composto di vettori a due a due ortogonali, per ottenere l'elemento di miglior approssimazione bisogna:
		\begin{itemize}
			\item calcolare i prodotti scalari \(G_{j, k} = (\varphi_j, \varphi_k)\) per \(j, k \in \Set{1, \dots, N}\);
			\item calcolare i coefficienti \(b_j = (\varphi_j, f)\) per \(j \in \Set{1, \dots, N}\);
			\item risolvere numericamente il sistema lineare \(c^* G = b\).
		\end{itemize}
		Se, invece, l'insieme dei \(\varphi_j\) consta di elementi a due a due ortogonali, ovvero è una base ortogonale del suo sottospazio generato, si vede che la risoluzione delle equazioni \eqref{eq:eq-normali} è data dai \emph{coefficienti di Fourier}
		\begin{equation}\label{eq:coeff-fourier}
			c_k^* = \frac{(\varphi_k, f)}{(\varphi_k, \varphi_k)}
		\end{equation}
		al variare di \(k \in \Set{1, \dots, N}\), il che necessita di soli \(2 N\) prodotti interni e \(N\) divisioni. Il procedimento è ancora meno dispendioso se i \(\varphi_j\) formano una base ortonormale del proprio sottospazio generato, in quanto non è necessario effettuare le divisioni.
	\end{osservazione}

	I prossimi risultati servono a chiarire il concetto di \emph{base} di uno spazio euclideo di dimensione anche infinita e sotto quali ipotesi si possa ottenere una base numerabile oppure ortonormale.
	
	\begin{definizione}[Spazio separabile]
		Uno spazio euclideo \(\E\) si dice \emph{separabile} se esiste \(S \subseteq \E\) numerabile denso in \(\E\).
	\end{definizione}

	\begin{teorema}\label{th:base-sp-euclideo-separabile}
		In uno spazio euclideo separabile \(\E\) esiste un sottoinsieme \(\Set{\varphi_k | k \in \N}\) di cardinalità finita o al piú numerabile tale che, se \(x \in \E\), allora esistono \(c_k\) con \(k \in \N\) tali che
		\begin{equation*}
			x = \sum_{k \in \N} c_k \varphi_k
		\end{equation*}
		ovvero
		\begin{equation*}
			\lim_{n \to \infty} \norm{x - \sum_{k = 0}^n c_k \varphi_k}_2 = 0
		\end{equation*}
		L'insieme \(\Set{\varphi_k | k \in \N}\) prende il nome di \emph{base} di \(\E\).
	\end{teorema}

	D'ora in avanti supporremo sempre che lo spazio euclideo \(\E\) è separabile.
	
	Si può dimostrare il seguente teorema, che generalizza a spazi di dimensione infinita l'algoritmo di Gram-Schmidt.
	
	\begin{teorema}\label{th:gram-schmidt}
		Dato uno spazio euclideo \(\E\), se esiste un insieme numerabile \(\Set{f_n | n \in \N^*} \subseteq \E\) di elementi linearmente indipendenti, allora \(\E\) contiene un insieme \(\Set{\varphi_k | k \in \N^*}\) tale che:
		\begin{itemize}
			\item per ogni \(m, n \in \N^*\) si ha \((\varphi_m, \varphi_n) = \delta_{m, n}\);
			\item \(\varphi_n \in \Braket{f_1, \dots, f_n}\) per ogni \(n \in \N^*\);
			\item \(f_n \in \Braket{\varphi_1, \dots, \varphi_n}\) per ogni \(n \in \N^*\).
		\end{itemize}
	\end{teorema}

	Si osservi che l'insieme degli \(f_n\) può anche essere infinito, al contrario di ciò che è richiesto per ortogonalizzare le matrici, cosí come può non essere finito anche l'insieme dei \(\varphi_k\). Dal Teorema~\ref{th:gram-schmidt}, inoltre, segue che se \(\E\) ha una base numerabile di elementi linearmente indipendenti, allora ammette anche una base ortonormale numerabile.
	
	\begin{definizione}[Serie di Fourier]\label{def:serie-fourier}
		Dato uno spazio euclideo \(\E\) dotato di una successione di elementi ortonormali \(\Set{\varphi_k | k \in \N^*}\), se per \(f \in \E\) si definiscono \(c_k = (f, \varphi_k)\) per ogni \(k \in \N^*\), si dice \emph{serie di Fourier} di \(f\) la serie formale
		\begin{equation}\label{eq:serie-fourier}
			\sum_{k = 1}^\infty c_k \varphi_k
		\end{equation}
	\end{definizione}

	\begin{definizione}\label{def:chiuso}
		Sia \(\Set{\varphi_n | n \in \N^*}\) una successione di elementi ortonormali di uno spazio vettoriale normato \(X\). Se ogni \(f \in X\) può essere scritto formalmente come serie di Fourier mediante i \(\varphi_n\), allora l'insieme dei \(\varphi_n\) si dice \emph{chiuso in \(X\)}.
	\end{definizione}

	\begin{teorema}[Bessel-Parseval]\label{th:bessel-parseval}
		Data una successione \(\Set{\varphi_k | k \in \N^*}\) di elementi ortonormali di uno spazio euclideo \(\E\), se \(f \in \E\), allora
		\begin{equation}\label{eq:bessel-parseval}
			\min_{a_1, \dots, a_n} \norm{f - \sum_{k = 1}^n a_k \varphi_k}_2 = \sqrt{\norm{f}_2^2 - \sum_{k = 1}^n c_k^2}
		\end{equation}
		ove \(c_k = (f, \varphi_k)\) per ogni \(k \in \Set{1, \dots, n}\). Vale, inoltre, la diseguaglianza di Bessel
		\begin{equation}\label{eq:bessel}
			\sum_{k = 1}^\infty c_k^2 \le \norm{f}_2^2
		\end{equation}
		e vale l'uguaglianza, detta uguaglianza di Parseval, per ogni \(f \in \E\)
		\begin{equation}\label{eq:parseval}
			\sum_{k = 1}^\infty c_k^2 = \norm{f}_2^2
		\end{equation}
		se e solo se l'insieme \(\Set{\varphi_k | k \in \N^*}\) è chiuso in \(\E\).
	\end{teorema}

	\begin{osservazione}
		In base al Teorema~\ref{th:bessel-parseval}, la soluzione al problema di miglior approssimazione in norma euclidea esiste ed è unica, ed è determinata dai coefficienti di Fourier. In virtú della \eqref{eq:parseval}, poi, se \(\Set{c_1, \dots, c_n}\) determina l'elemento di miglior approssimazione in norma indotta dal prodotto scalare in \(S_n = \Braket{\varphi_1, \dots, \varphi_n}\), allora
		\begin{equation*}
			\lim_{n \to \infty} \norm{f -  \sum_{k = 1}^n c_k \varphi_k}_2 = \lim_{n \to \infty} \sqrt{\norm{f}_2^2 - \sum_{k = 1}^n c_k^2} = 0
		\end{equation*}
	\end{osservazione}

\section{Polinomi trigonometrici}
	
	\begin{definizione}[Polinomi trigonometrici reali]\label{def:polin-trig-reali}
		Si dice \emph{spazio dei polinomi trigonometrici reali di grado \(n\)} lo spazio vettoriale \(\T_n^\R\) costituito dalle combinazioni lineari delle funzioni
		\begin{subequations}
			\begin{align}\label{eq:polin-trig-reali}
				\varphi_0^* (x)         & = 1          \\
				\varphi_{2 k - 1}^* (x) & = \cos (k x) \\
				\varphi_{2 k}^* (x)     & = \sin (k x)
			\end{align}
		\end{subequations}
		per \(k \in \Set{1, \dots, n}\).
	\end{definizione}

	Dati \(n, m \in \N\), valgono le formule di Werner
		\begin{align*}
			\cos (n x) \cos (m x) &= \frac{\cos ((n + m) x) + \cos ((n - m) x)}{2} \\
			\sin (n x) \sin (m x) &= \frac{\cos ((n - m) x) - \cos ((n + m) x)}{2}
		\end{align*}
	e si ha
	\begin{equation*}
		\int_{- \pi}^\pi \cos (k x) \dd{x} =
		\begin{cases}
			0     & k \ne 0 \\
			2 \pi & k = 0
		\end{cases}
	\end{equation*}
	A partire da queste identità si trova che
		\begin{gather*}
			n \in \N^* \implies \int_{- \pi}^\pi \cos^2 (n x) \dd{x} = \int_{- \pi}^\pi \frac{\cos (2 n x) + 1}{2} \dd{x} = \pi \\
			n \in \N^* \implies \int_{- \pi}^\pi \sin^2 (n x) \dd{x} = \int_{- \pi}^\pi \qty(1 - \cos^2 (n x)) \dd{x} = \pi \\
			m, n \in \N^* \implies \int_{- \pi}^\pi \cos (n x) \sin (m x) \dd{x} = 0 \\
			m, n \in \N^*, m \ne n \implies \int_{- \pi}^\pi \cos (n x) \cos (m x) \dd{x} = 0 \\
			m, n \in \N^*, m \ne n \implies \int_{- \pi}^\pi \sin (n x) \sin (m x) \dd{x} = 0
		\end{gather*}

	Consideriamo ora lo spazio \(L^2_\R ([\,- \pi, \pi \,])\) delle funzioni \(f \colon [\,- \pi, \pi \,] \to \R\) misurabili tali che \(\abs{f}^2\) sia integrabile e dotiamolo del prodotto scalare
	\begin{equation}\label{eq:prod-scal-l2r}
		(f, g) \coloneqq \int_{- \pi}^\pi f (x) g (x) \dd{x}
	\end{equation}
	Si può dimostrare che le \(2 n + 1\) funzioni
	\begin{subequations}\label{eq:polin-trig-reali-normalizzati}
		\begin{align}
			\varphi_0 (t)         & = \frac{1}{\sqrt{2 \pi}}      \\
			\varphi_{2 k - 1} (t) & = \frac{\cos k t}{\sqrt{\pi}} \\
			\varphi_{2 k} (t)     & = \frac{\sin k t}{\sqrt{\pi}}
		\end{align}
	\end{subequations}
	definite per \(k \in \Set{1, \dots, n}\), formano una base ortonormale di \(\T_n^\R\), che dotiamo del prodotto scalare definito nella \eqref{eq:prod-scal-l2r}. Si può anche vedere che la successione \(\Set{\varphi_n | n \in \N}\) è chiusa in \(L^2_\R ([\,- \pi, \pi \,])\).
	
	In virtú di queste osservazioni e del Teorema~\ref{th:bessel-parseval} si può dimostrare il seguente asserto.
	
	\begin{teorema}
		Considerata la successione di elementi ortonormali in \(L^2_\R ([\,- \pi, \pi \,])\) definita nella \eqref{eq:polin-trig-reali-normalizzati}, i coefficienti di Fourier che determinano l'elemento di miglior approssimazione di \(f \in L^2_\R ([\,- \pi, \pi \,])\) secondo il prodotto scalare definito nella \eqref{eq:prod-scal-l2r} sono
		\begin{subequations}
			\begin{align}
				c_0         & = \frac{1}{\sqrt{2 \pi}} \int_{- \pi}^\pi f (x) \dd{x}          \\
				c_{2 k - 1} & = \frac{1}{\sqrt{\pi}} \int_{- \pi}^\pi f (x) \cos (k x) \dd{x} \\
				c_{2 k}     & = \frac{1}{\sqrt{\pi}} \int_{- \pi}^\pi f (x) \sin (k x) \dd{x}
			\end{align}
		\end{subequations}
		definiti per \(k \in \N^*\) e si verifica \(\lim_{n \to \infty} E_n (f) = 0\).
	\end{teorema}

	\begin{definizione}[Polinomi trigonometrici complessi]\label{def:polin-trig-complessi}
		Si dice \emph{spazio dei polinomi trigonometrici complessi di grado \(n\)} lo spazio vettoriale \(\T_n^\C\) costituito dalle combinazioni lineari delle funzioni
		\begin{subequations}\label{eq:polin-trig-complessi}
			\begin{align}
				\varphi_0^* (x)         & = 1              \\
				\varphi_{2 k - 1}^* (x) & = \exp (- \ii k x) \\
				\varphi_{2 k}^* (x)     & = \exp (\ii k x)
			\end{align}
		\end{subequations}
		per \(k \in \Set{1, \dots, n}\), ove \(\ii\) indica la costante immaginaria.
	\end{definizione}

	\begin{teorema}
		La successione \(\Set{\varphi_k^* | k \in \N}\) definita nella \eqref{eq:polin-trig-complessi} è composta di elementi ortogonali di \(L^2_\C ([\, 0, 2 \pi \,])\).
	\end{teorema}
	
	\begin{proof}
		In base all'identità di Eulero \(\exp (\ii t) = \cos t + \ii \sin t\), valida per ogni \(t \in \R\), si ha per ogni \(k \in \Z\) che
		\begin{equation*}
			\overline{\exp (\ii k x)} = \overline{\cos (k x) + \ii \sin (k x)} = \cos (k x) - \ii \sin (k x) = \exp (- \ii k x)
		\end{equation*}
		Dall'uguaglianza \(\exp (\ii j x) \exp (\ii k x) = \exp (\ii (j + k) x)\), valida per ogni \(j, k \in \Z\), si ricava
		\begin{equation*}
			\int_0^{\mathrlap{2 \pi}} \exp (\ii j x) \, \overline{\exp(\ii k x)} \dd{x} = \int_0^{\mathrlap{2 \pi}} \exp (\ii (j - k) x) \dd{x}
		\end{equation*}
		Se \(j = k\), l'integrale sopra vale \(2 \pi\), mentre se \(j \ne k\) si ha
		\begin{equation*}
			\begin{split}
				\int_0^{\mathrlap{2 \pi}} \exp (\ii (j - k) x) \dd{x} &= \int_0^{\mathrlap{2 \pi}} \cos ((j - k) x) + \ii \sin ((j - k) x) \dd{x} \\
				&= \frac{1}{j - k} \int_0^{\mathrlap{2 (j - k) \pi}} \quad \cos t + \ii \sin t \dd{t} \\
				&= \frac{1}{j - k} \eval[\sin t - \ii \cos t|_{t = 0}^{2 (j - k) \pi} = 0
			\end{split}
		\end{equation*}
		dato che \(\sin\) e \(\cos\) sono funzioni di periodo \(2 \pi\).
	\end{proof}

	Dal momento che la successione \(\Set{\varphi_k^* | k \in \N}\) è chiusa in \(L_\C^2 ([\, 0, 2 \pi \,])\), si può enunciare il seguente teorema.
	
	\begin{teorema}
		Considerata la successione \(\Set{\varphi_k | k \in \N}\) di elementi ortonormali definita da
		\begin{subequations}
			\begin{align}
				\varphi_0 (x)         & = \frac{1}{\sqrt{2 \pi}}                \\
				\varphi_{2 k - 1} (x) & = \frac{\exp (- \ii k x)}{\sqrt{2 \pi}} \\
				\varphi_{2 k} (x)     & = \frac{\exp (\ii k x)}{\sqrt{2 \pi}}
			\end{align}
		\end{subequations}
		i coefficienti di Fourier che determinano l'elemento di miglior approssimazione di una funzione \(f \in L_\C^2 ([\, 0, 2 \pi \,])\) secondo il prodotto scalare definito da \((f, g) \coloneqq \int_0^{2 \pi} f (x) \overline{g (x)} \dd{x}\) sono
		\begin{subequations}
			\begin{align}
				c_0         & = \frac{1}{\sqrt{2 \pi}} \int_0^{\mathrlap{2 \pi}} f (x) \dd{x}                  \\
				c_{2 k - 1} & = \frac{1}{\sqrt{2 \pi}} \int_0^{\mathrlap{2 \pi}} f (x) \exp (\ii k x) \dd{x}   \\
				c_{2 k}     & = \frac{1}{\sqrt{2 \pi}} \int_0^{\mathrlap{2 \pi}} f (x) \exp (- \ii k x) \dd{x}
			\end{align}
		\end{subequations}
		definiti per \(k \in \N^*\) e si verifica \(\lim_{n \to \infty} E_n (f) = 0\).
	\end{teorema}

	Di solito le serie di Fourier sono scritte attraverso la serie bilatera
	\begin{equation*}
		f (x) = \sum_{k = - \infty}^\infty \gamma_k \exp (\ii k x)
	\end{equation*}
	ove si è posto
	\begin{equation}\label{eq:coeff-fourier-complessi}
		\gamma_k = \frac{1}{2 \pi} \int_0^{\mathrlap{2 \pi}} f (x) \exp (- \ii k x) \dd{x}
	\end{equation}
	Si ha, infatti,
	\begin{equation}\label{eq:fourier-formula-complessa}
		\begin{split}
			f (x) &= \sum_{k = - \infty}^\infty \frac{1}{\sqrt{2 \pi}} \qty(\int_0^{\mathrlap{2 \pi}} f (x) \exp (- \ii k x) \dd{x}) \frac{\exp (\ii k x)}{\sqrt{2 \pi}} \\
			&= \sum_{k = - \infty}^\infty \frac{1}{2 \pi} \qty(\int_0^{\mathrlap{2 \pi}} f (x) \exp (- \ii k x) \dd{x}) \exp (\ii k x)
		\end{split}
	\end{equation}

	Data una funzione \(f \in L_\C^2 ([\, 0, 2 \pi \,])\) continua in \([\, 0, 2 \pi \,]\) e tale che \(f (0) = f (2 \pi)\), essa ammette una rappresentazione formale come nella \eqref{eq:fourier-formula-complessa}. In pratica, però, non si calcola tutta la serie, ma si considera un'approssimazione del tipo
	\begin{equation}\label{eq:funz-fourier-approx}
		f_M (x) = \sum_{k = - M}^M \frac{1}{2 \pi} \qty(\int_0^{\mathrlap{2 \pi}} f (x) \exp (- \ii k x) \dd{x}) \exp (\ii k x)
	\end{equation}
	con un \(M \in \N\) sufficientemente grande. Questa approssimazione trigonometrica richiede che si calcolino numericamente i coefficienti
	\begin{equation}\label{eq:coeff-approx-fourier}
		I_k \coloneqq \frac{1}{2 \pi} \int_0^{\mathrlap{2 \pi}} f (x) \exp (- \ii k x) \dd{x}
	\end{equation}
	per \(k \in \Set{-M, \dots, M}\). È possibile dimostrare che, se \(f\) è continua e periodica, ovvero \(f (0) = f (2 \pi)\), è vantaggioso calcolare i coefficienti nella \eqref{eq:coeff-approx-fourier} mediante la \emph{formula dei trapezi composta}
	\begin{equation*}
		\int_a^b g(x) \dd{x} \approx \frac{h}{2} (g (a) + g (b)) + h \sum_{j = 1}^{M^* - 1} f (x_j)
	\end{equation*}
	ove \(x_j = a + j h\) per \(j \in \Set{0, \dots, M^*}\) e \(h = (b - a) / M^*\). Nel caso particolare in cui \(g \colon [\, 0, 2 \pi \,] \to \R\) sia continua e periodica, ovvero \(g (0) = g (2 \pi)\), usando la formula dei trapezi composta su \(M^* + 1\) nodi equispaziati \(x_j = j h\), ove \(j \in \Set{0, \dots, M^*}\) e \(h = 2 \pi / M^*\), si trova
	\begin{equation*}
		\begin{split}
			\int_0^{2 \pi} g (x) \dd{x} &\approx \frac{h}{2} (g (0) + g (2 \pi)) + h \sum_{j = 1}^{M^* - 1} g (x_j) \\
			&= h \, g (2 \pi) + h \sum_{j = 1}^{M^* - 1} g (x_j) \\
			&= h \sum_{j = 1}^{M^*} g (x_j) = \frac{2 \pi}{M^*} \sum_{j = 1}^{M^*} g (x_j) \\
			&= \frac{2 \pi}{M^*} \sum_{j = 1}^{M^*} g \qty(\frac{2 \pi j}{M^*})
		\end{split}
	\end{equation*}
	Supponendo, ora, che \(f\) sia continua e periodica in \([\, 0, 2 \pi \,]\), se per ogni \(k \in \Set{-M, \dots, M}\) consideriamo \(g (x) = f (x) \exp (- \ii k x)\) e definiamo \(M^* = 2 M + 1\), per come sono definiti i \(\gamma_k\) nella \eqref{eq:coeff-fourier-complessi} si ottiene
	\begin{equation}
		\gamma_k \approx \frac{1}{2 M + 1} \sum_{j = 1}^{2 M + 1} f \qty(\frac{2 \pi j}{2 M + 1}) \, \exp \qty(- \ii k \frac{2 \pi j}{2 M + 1})
	\end{equation}
	Se, poi, definiamo per \(j \in \Set{1, \dots, 2 M + 1}\)
	\begin{equation*}
		\mathcal{X}_j = \frac{1}{2 M + 1} f \qty(\frac{2 \pi j}{2 M + 1})
	\end{equation*}
	e a partire da essi
	\begin{equation}\label{eq:coeff-fourier-approx}
		\gammanum{k} = \sum_{j = 1}^{2 M + 1} \mathcal{X}_j \exp \qty(- \ii k \frac{2 \pi j}{2 M + 1})
	\end{equation}
	si vede che \(\gamma_k \approx \gammanum{k}\) per ogni \(k \in \Set{-M, \dots, M}\). Sotto opportune condizioni, si può ricorrere all'algoritmo della trasformata rapida di Fourier (\textsc{fft}) per calcolare i \(2 M + 1\) coefficienti \(\gammanum{-M}, \dots, \gammanum{M}\) in \(\mathcal{O} (M \log M)\) operazioni anziché in \(\mathcal{O} (M^2)\), come si riuscirebbe con altri algoritmi.
	
	\begin{teorema}[Polinomio trigonometrico interpolante]\label{th:polin-trig-interp}
		Se una funzione \(f \colon [\, 0, 2 \pi \,] \to \R\) è continua e periodica, allora il polinomio trigonometrico
		\begin{equation}\label{eq:polin-trig-interp}
			p_M (x) = \sum_{k = - M}^M \gammanum{k} \exp(\ii k x)
		\end{equation}
		con \(\gammanum{k}\) definito come nella \eqref{eq:coeff-fourier-approx}, interpola \(f\) nei nodi equispaziati
		\begin{equation*}
			x_j = \frac{2 \pi j}{2 M + 1}
		\end{equation*}
		definiti per \(j \in \Set{0, \dots, 2 M + 1}\).
	\end{teorema}

	Con i risultati seguenti ci assicuriamo del margine di errore in norma euclidea della funzione \(f_M\) definita nella \eqref{eq:funz-fourier-approx} e dell'interpolante \(p_M\) definito nella \eqref{eq:polin-trig-interp} e, di conseguenza, della natura dell'errore commesso nell'approssimare numericamente i \(\gamma_k\) coi \(\gammanum{k}\).
	
	\begin{definizione}\label{def:variaz-limit}
		Una funzione \(f \colon [\, a, b \,] \to \R\) si dice \emph{a variazione limitata} se
		\begin{equation}
			T_a^b (f) \coloneqq \sup \Set{\sum_{i = 1}^n \abs{f (t_i) - f (t_{i - 1})} : a = t_0 < \dots < t_n = b} < + \infty
		\end{equation}
		La quantità \(T_a^b (f)\) prende il nome di \emph{variazione} di \(f\).
	\end{definizione}

	\begin{esempio}
		Le funzioni lipschitziane su \([\, a, b \,]\) sono a variazione limitata, perché esiste \(L \in \R_{\ge 0}\) tale che \(T_a^b (f) \le L (b - a)\). Per lo stesso motivo le funzioni di classe \(\cont^1 ([\, a, b \,])\) sono anch'esse a variazione limitata.
	\end{esempio}
	
	Nei seguenti teoremi usiamo la notazione
	\begin{equation}
		S (\alpha) \coloneqq \Set{x + i y \in \C : -\alpha < y < \alpha}
	\end{equation}
	definita per ogni \(\alpha \in \R_{\ge 0}\).
	
	\begin{teorema}
		Se \(f \colon [\, 0, 2 \pi \,] \to \R\) è una funzione differenziabile \(\eta \in \N^*\) volte, periodica e tale che \(f^{(\eta)}\) sia periodica e a variazione limitata \(V\) in \([\, 0, 2 \pi \,]\), definiti
		\begin{gather*}
			I (f) = \int_0^{2 \pi} f (x) \dd{x} \\
			I_N (f) = \frac{\pi}{N} (f (0) + f (2 \pi)) + \frac{2 \pi}{N} \sum_{j = 2}^{N - 1} f \qty(\frac{2 \pi j}{N})
		\end{gather*}
		allora
		\begin{equation}
			\abs{I_N (f) - I (f)} \le \frac{4 V}{N^{\eta + 1}}
		\end{equation}
		Se, poi, \(f\) è analitica in \(S (\alpha)\) e ivi \(\abs{f (t)} \le \tilde{M}\), allora
		\begin{equation}
			\abs{I_N (f) - I (f)} \le \frac{4 \pi \tilde{M}}{\ee^{\alpha N} - 1}
		\end{equation}
	\end{teorema}

	\begin{teorema}
		Se \(f \colon [\, 0, 2 \pi \,] \to \R\) è una funzione differenziabile \(\eta \in \N^*\) volte, periodica e tale che \(f^{(\eta)}\) sia periodica e a variazione limitata \(V\) in \([\, 0, 2 \pi \,]\), allora
		\begin{equation}
			\abs{\gamma_k} \le \frac{V}{2 \pi \abs{k}^{\eta + 1}}
		\end{equation}
		Se, poi, \(f\) è analitica in \(S (\alpha)\) e ivi \(\abs{f (t)} \le \tilde{M}\), allora
		\begin{equation}
			\abs{\gamma_k} \le \tilde{M} \ee^{- \alpha \abs{k}}
		\end{equation}
	\end{teorema}

	\begin{teorema}
		Se \(f \colon [\, 0, 2 \pi \,] \to \R\) è una funzione differenziabile \(\eta \in \N^*\) volte, periodica e tale che \(f^{(\eta)}\) sia periodica e a variazione limitata \(V\) in \([\, 0, 2 \pi \,]\), allora \(f_M\) e \(p_M\) definiti nella \eqref{eq:funz-fourier-approx} e nella \eqref{eq:polin-trig-interp} rispettivamente verificano
		\begin{align}
			\norm{f - f_M}_\infty &\le \frac{V}{\pi \eta M^\eta} &
			\norm{f - p_M}_\infty &\le \frac{2 V}{\pi \eta M^\eta}
		\end{align}
	\end{teorema}
	
	\chapter{Polinomi ortogonali}

	\begin{definizione}
		Si dice \emph{spazio di Hilbert} uno spazio euclideo completo, separabile e di dimensione infinita.
	\end{definizione}

	\begin{esempio}
		Lo spazio \((L^2 ((\, a, b \,)), \norm{\cdot}_2)\) delle funzioni reali misurabili di modulo quadrato integrabile su un intervallo anche illimitato \((\, a, b \,)\) con la norma definita a partire dal prodotto scalare descritto nella \eqref{eq:prod-scal-l2r} è uno spazio di Hilbert.
	\end{esempio}

	\begin{esempio}
		Data una funzione misurabile positiva \(w \colon (\, a, b \,) \to \R\), lo spazio \((L^2_w ((\, a, b \,)), \norm{\cdot}_{2, w})\) delle funzioni misurabili \(f\) tali che
		\begin{equation*}
			\int_a^b \abs{f (x)}^2 w (x) \dd{x} < + \infty
		\end{equation*}
		è uno spazio di Hilbert dotato del prodotto scalare
		\begin{equation}\label{eq:prod-scal-l2w}
			(f, g)_{2, w} \coloneqq \int_a^b f (x) g (x) w (x) \dd{x}
		\end{equation}
	\end{esempio}

	\begin{definizione}
		Si dice \emph{funzione peso} una funzione \(w \colon (\, a, b \,) \to \R\) non negativa tale che per ogni \(n \in \N\)
		\begin{subequations}
			\begin{equation}
				\int_a^b \abs{x}^n w (x) \dd{x} < + \infty
			\end{equation}
			e che per ogni \(g\) continua e non negativa si abbia
			\begin{equation}
				\int_a^b g (x) w (x) \dd{x} = 0 \implies g \equiv 0
			\end{equation}
		\end{subequations}
		
	\end{definizione}

	\begin{esempio}[Funzioni peso classiche]
		Riportiamo alcuni esempi di funzioni peso comuni.
			\begin{itemize}
				\item Peso di Legendre: \(w (x) = 1\) per \(x \in [\, -1, 1\,]\).
				\item Peso di Chebyshev: \(w (x) = 1 / \sqrt{1 - x^2}\) per \(x \in (\, -1, 1 \,)\).
				\item Peso di Gegenbauer: \(w (x) = \qty(1 - x^2)^{\gamma - 1 / 2}\) per \(x \in (\, -1, 1 \,)\), con \(\gamma > (- 1 / 2)\).
				\item Peso di Jacobi: \(w (x) = (1 - x)^\alpha (1 + x)^\beta\) per \(x \in (\, -1, 1 \,)\), con \(\alpha, \beta > - 1\).
				\item Peso di Laguerre: \(w (x) = \ee^{-x}\) per \(x \in (\, 0, + \infty \,)\).
				\item Peso di Hermite: \(w (x) = \ee^{- x^2}\) per \(x \in \R\).
			\end{itemize}
	\end{esempio}

	\begin{teorema}
		Per ogni \(n \in \N\) si ha \(\P_n \subseteq L_w^2 ((\, a, b \,))\).
	\end{teorema}

	\begin{proof}
		Scelto un qualunque polinomio \(p_n (x) = \sum_{k = 0}^n a_k x^k\) di grado \(n\), per la disuguaglianza triangolare e per il fatto che
		\begin{equation*}
			\norm{x^k}_{2, w}^2 = \int_a^b \abs{x}^{2 k} w (x) \dd{x} < + \infty
		\end{equation*}
		si ottiene
		\begin{equation*}
			\norm{p_n}_{2, w} = \norm{\sum_{k = 0}^n a_k x^k}_{2, w} \le \sum_{k = 0}^n \abs{a_k} \, \norm{x^k}_{2, w} < + \infty
		\end{equation*}
		da cui si conclude che \(p_n \in L_w^2 ((\, a, b \,))\).
	\end{proof}

	\begin{osservazione}
		Se \(a\) e \(b\) sono finiti, per il teorema di Weierstrass
		\begin{equation*}
			\norm{x^n}_\infty = \max_{x \in [\, a, b \,]} \abs{x}^n < + \infty
		\end{equation*}
		e quindi, poiché \(w (x) \ge 0\) per ogni \(x \in (\, a, b \,)\),
		\begin{equation*}
			\int_a^b \abs{x}^n w (x) \dd{x} \le \norm{x^n}_\infty \int_a^b w (x) \dd{x}
		\end{equation*}
		In base a ciò, si conclude che per ogni \(n \in \N\)
		\begin{equation}
			\int_a^b w (x) \dd{x} < + \infty \implies \int_a^b \abs{x}^n w (x) \dd{x} < + \infty
		\end{equation}
	\end{osservazione}

	Fissati \(f \in L_w^2 ((\, a, b \,))\) e \(n \in \N\), vogliamo risolvere il \emph{problema dei minimi quadrati} nel continuo, ovvero determinare il polinomio \(p_n^* \in \P_n\), se esiste, che minimizzi al variare di \(p_n \in \P_n\) la quantità
	\begin{equation*}
		\norm{f - p_n}_{2, w} = \int_a^b \abs{f (x) - p_n (x)}^2 w (x) \dd{x}
	\end{equation*}
	
	Si può dimostrare che, nel caso in cui \(a, b\) siano finiti e \(f \in \cont ([\, a, b \,])\), si ha \(\norm{f - p_n^*}_{2, w} \to 0\) per \(n \to + \infty\).
	
	Dal momento che \(L_w^2 ([\, a, b \,])\) è uno spazio euclideo, individuata una base \(\varphi_0, \dots, \varphi_n\) di \(\P_n\), abbiamo già visto che la soluzione del problema di trovare
	\begin{equation*}
		\norm{f - f^*}_{2, w} = \; \min_{g \in \Braket{\varphi_0, \dots, \varphi_n}} \norm{f - g}_{2, w}
	\end{equation*}
	è data da \(f^* = \sum_{j = 0}^n \gamma_j^* \varphi_j\), ove i \(\gamma_j^*\) verificano le equazioni normali definite nella \eqref{eq:eq-normali}. Abbiamo anche visto che \(f^*\) è tale che \((f, \varphi_j)_{2, w} = (f^*, \varphi_j)_{2, w}\) per ogni \(j \in \Set{0, \dots, n}\).
	
	\begin{definizione}
		Una sequenza di polinomi \(\varphi_0, \dots, \varphi_n\) triangolare, ovvero tale che \(\deg (\varphi_k) = k\) per ogni \(k \in \Set{0, \dots, n}\), si dice \emph{ortogonale} rispetto alla funzione peso \(w\) se verifica per ogni \(i, j \in \Set{0, \dots, n}\)
		\begin{equation}
			(\varphi_i, \varphi_j)_{2, w} = c_i \delta_{i, j}
		\end{equation}
		ove \(\delta_{i, j}\) è il delta di Kronecker e \(c_i > 0\) per ogni \(i \in \Set{0, \dots, n}\).
	\end{definizione}

	\begin{osservazione}
		Attraverso il procedimento di Gram-Schmidt, si può dimostrare che esiste una tale famiglia di polinomi e costruirla direttamente. Si può osservare, inoltre, che ogni polinomio di grado \(n\) si può scrivere univocamente come combinazione lineare di \(\varphi_0\, dots, \varphi_n\). In base a queste osservazioni, per ogni \(p_m = \sum_{j = 0}^m a_j x^j\) con \(m < n\) dalla bilinearità del prodotto scalare segue che
		\begin{equation}
			\qty(\varphi_{m + 1}, p_m)_{2, w} = \qty(\varphi_{m + 1}, \sum_{j = 0}^m a_j x^j)_{\! 2, w} \hspace{-.7em} = \sum_{j = 0}^m a_j \qty(\varphi_{m + 1}, \varphi_j)_{2, w} = 0
		\end{equation}
	\end{osservazione}
	
	\begin{teorema}[Christoffel]
		Se una famiglia triangolare di polinomi \(\varphi_0, \dots, \varphi_n\) è ortogonale in \((\, a, b \,)\) rispetto a una funzione peso \(w\), allora il polinomio \(\varphi_n\) ha esattamente \(n\) zeri, i quali hanno tutti molteplicità \(1\) ed appartengono all'intervallo \((\, a, b \,)\).
	\end{teorema}

	\begin{proof}
		Chiamati \(x_1, \dots, x_m\) con \(m \le n\) gli zeri di \(\varphi_n\) interni ad \((\, a, b \,)\) e chiamate \(\alpha_1, \dots, \alpha_m\) le molteplicità rispettive, esiste \(a_n\) tale che
		\begin{equation*}
			\varphi_n (x) = a_n \qty[\prod_{k = 1}^m \qty(x - x_k)^{\alpha_k}] r (x)
		\end{equation*}
		ove si suppone che \(\prod_{k = 1}^m (x - x_k)^{\alpha_k} \equiv 1\) se \(\varphi_n\) non ammette zeri interni ad \((\, a, b \,)\). Per costruzione, \(r\) non ammette zeri in \((\, a, b \,)\) e, dato che è una funzione continua, è di segno costante.
		
		Consideriamo ora il polinomio di grado al piú \(n\)
		\begin{equation*}
			q (x) = \prod_{k = 1}^m \qty(x - x_k)^{\varmod_2 \alpha_k}
		\end{equation*}
		Si osserva che, se uno zero di \(\varphi_n\) interno ad \((\, a, b \,)\) ha molteplicità dispari ma maggiore di \(1\), allora \(\deg q < n\); se uno zero di \(\varphi_n\) interno ad \((\, a, b \,)\) ha molteplicità pari, allora \(\deg q < n\); se uno zero di \(\varphi_n\) è esterno ad \((\, a, b \,)\), allora \(\deg q < n\), perché \(r\) sarebbe di grado almeno \(1\). Ricordiamo, poi, che \(\alpha_k + \varmod_2 \alpha_k\) è pari per ogni \(\alpha_k \in \N\).
		
		Per assurdo sia \(q \in \P_{n - 1}\). Per come sono definiti \(\varphi_n\) e \(q\), si avrebbe
		\begin{equation*}
			\varphi_n (x) \, q(x) = a_n \qty[\prod_{k = 1}^m \qty(x - x_k)^{\alpha_k + \varmod_2 \alpha_k}] r (x)
		\end{equation*}
		il quale avrebbe segno costante e grado almeno \(n\), non coincidendo col polinomio. Dato che, però,
		\begin{multline*}
			0 = \qty(\varphi_n, q)_{2, w} = \int_a^b \varphi_n (x) q (x) w (x) \dd{x} \\
			= \int_a^b a_n \qty[\prod_{k = 1}^m \qty(x - x_k)^{\alpha_k + \varmod_2 \alpha_k}] r (x) w (x) \dd{x} \ne 0
		\end{multline*}
		si è trovato un assurdo. In virtú di ciò, si può concludere.
	\end{proof}

	Non è possibile che \(\varphi_n\) abbia come zeri \(a\) oppure \(b\): nella dimostrazione, infatti, basterebbe mettere in risalto che \(r (x)\) si annulla in \(a\) oppure \(b\), rimanendo comunque di segno costante in \((\, a, b \,)\).
	
	\begin{definizione}
		Un polinomio \(p (x) = \sum_{j = 0}^n a_j x^j\) di grado \(n\) si dice \emph{monico} se \(a_n = 1\).
	\end{definizione}

	\begin{teorema}[Ricorsione a tre termini]
		Se una famiglia triangolare \(\varphi_0, \dots, \varphi_n\) di polinomi monici in \((\, a, b \,)\) è ortogonale rispetto a una funzione peso \(w\), allora, supposti \(\varphi_{-1} (x) = 0\) e \(\varphi_0 (x) = 1\), vale per ogni \(n \in \N^*\)
		\begin{equation}\label{eq:ricorsione-tre-termini}
			\varphi_{n + 1} (x) = (x - \beta_n) \varphi_n (x) - \gamma_n \varphi_{n - 1} (x)
		\end{equation} 
		con
		\begin{align}
			\beta_n &= \frac{(x \varphi_n, \varphi_n)_{2, w}}{(\varphi_n, \varphi_n)_{2, w}} &
			\gamma_n &= \frac{(x \varphi_{n - 1}, \varphi_n)_{2, w}}{(\varphi_{n - 1}, \varphi_{n - 1})_{2, w}}
		\end{align}
	\end{teorema}

	Osserviamo che, scelti i polinomi ortogonali \(\varphi_0\) e \(\varphi_1\), la procedura descritta tramite la \eqref{eq:ricorsione-tre-termini} determina la famiglia triangolare di polinomi ortogonali di grado superiore, dopo aver calcolato \(\beta_k\) e \(\gamma_k\). Se, poi, \(\varphi_n\) verifica \((\varphi_n, \varphi_k)_{2, w} = 0\) per ogni \(k \in \Set{0, \dots, n - 1}\), allora tale condizione è soddisfatta anche da \(\tilde{\varphi}_n = \tau \varphi_n\).
	
	Qualora sia richiesta una famiglia di polinomi \(\hat{\varphi}_0, \dots, \hat{\varphi}_n\) ortonormale, si può partire da una famiglia ortogonale di polinomi monici e ragionare come segue. Dal momento che, infatti, \(\gamma_0 = \int_a^b w (x) \dd{x}\), dalla \eqref{eq:ricorsione-tre-termini} segue che
	\begin{equation*}
		\norm{\varphi_{n + 1}}_{2, w}^2 = \prod_{i = 0}^n \gamma_i \implies \hat{\varphi}_{n + 1} = \frac{\varphi_{n + 1}}{\norm{\varphi_{n + 1}}_{2, w}}
	\end{equation*}
	
	\chapter{Quadratura numerica}

	\noindent Un problema classico dell'analisi numerica consiste nell'approssimare integrali definiti di una funzione reale \(f\), definita in un intervallo \((\, a, b \,)\) non per forza limitato, del tipo
	\begin{equation}\label{eq:integrale-con-peso}
		I_w (f) = I_w (f, a, b) = \int_a^b f (x) w (x) \dd{x}
	\end{equation}
	ove \(w \colon (\, a, b \,) \to \R\) è una funzione peso. Questi integrali possono essere approssimati con formule del tipo
	\begin{equation}\label{eq:formula-quadratura}
		I_w (f) \approx S_N (f) = \sum_{i = 1}^N w_i f (x_i)
	\end{equation}
	ove \(w_1, \dots, w_N \in \R\) sono detti \emph{pesi} e \(x_0, \dots, x_N \in \mathcal{I} \supseteq (\, a, b \,)\) sono detti \emph{nodi}. D'ora in avanti supporremo che i nodi \(x_1, \dots, x_N\) siano distinti tra loro e che \(f \in \cont ((\, a, b \,))\) ammetta finito \(I_w (f)\), ovvero sia \(w\)-integrabile.
	
	Se \((\, a, b \,)\) è limitato e \(f \in \cont ([\, a, b \,])\), per l'integrabilità della funzione peso e per il teorema di Weierstrass si ha
	\begin{equation*}
		\abs{\int_a^b f (x) w (x) \dd{x}} \le \int_a^b \abs{f (x)} w (x) \dd{x} \le \norm{f}_\infty \norm{w}_1 < + \infty
	\end{equation*}
	ove si è usata la notazione \(\norm{w}_1 = \int_a^b \abs{w (x)} \dd{x} = \int_a^b w (x) \dd{x}\). Ciò mostra che, in tal caso, \(f\) è \(w\)-integrabile in \((\, a, b \,)\).
	
\section{Formule di Newton-Cotes}
	
	\noindent Indicato con
	\begin{equation*}
		p_{N - 1} = \sum_{i = 1}^N f (x_i) L_i (x)
	\end{equation*}
	il polinomio che interpola le coppie \((x_i, f (x_i))\) per \(i \in \Set{1, \dots, N}\) ove \(L_i\) indica l'\(i\)-esimo polinomio di Lagrange definito nella \eqref{eq:polin-lagrange}, possiamo scrivere
	\begin{equation*}
		\begin{split}
			\int_a^b f (x) w (x) \dd{x} &\approx \int_a^b p_{N - 1} (x) w (x) \dd{x} \\
			&= \int_a^b \sum_{i = 1}^N f (x_i) L_i (x) w (x) \dd{x} \\
			&= \sum_{i = 1}^N \qty(\int_a^b L_i (x) w (x) \dd{x}) f (x_i)
		\end{split}
	\end{equation*}
	Per com'è scritta l'ultima somma, ha senso definire
	\begin{equation*}
		w_i = \int_a^b L_i (x) w (x) \dd{x}
	\end{equation*}
	quali pesi per ogni \(i \in \Set{1, \dots, N}\).
	
	\begin{definizione}[Formula interpolatoria]\label{def:formula-interp}
		Si dice \emph{formula interpolatoria} una formula di quadratura di forma come nella \eqref{eq:formula-quadratura} tale che per ogni \(k \in \Set{1, \dots, N}\) si abbia
		\begin{equation}\label{eq:nodo-formula-interp}
			w_k = \int_a^b L_k (x) w (x) \dd{x}
		\end{equation}
	\end{definizione}

	\begin{definizione}[Grado di precisione]\label{def:grado-precisione}
		Si dice che una formula di quadratura di forma come nella \eqref{eq:formula-quadratura} ha \emph{grado di precisione almeno \(M\)} se è esatta per tutti i polinomi \(p \in \P_M\), ovvero se per ogni \(p \in \P_M\) si verifica
		\begin{equation}\label{eq:grado-precisione}
			\int_a^b p (x) \, w (x) \dd{x} = \sum_{i = 1}^N w_i \, p (x_i)
		\end{equation}
		Si dice, poi, che essa ha \emph{grado di precisione esattamente \(M\)} se ha grado di precisione almeno \(M\) ed esiste \(q \in \P_{M + 1}\) per cui la \eqref{eq:grado-precisione} non sia verificata.
	\end{definizione}

	\begin{teorema}\label{th:interp-grado-precisione}
		Una formula di quadratura definita come nella \eqref{eq:formula-quadratura} è interpolatoria se e solo se ha grado precisione almeno \(N - 1\).
	\end{teorema}

	\begin{proof}
		Mostriamo entrambe le implicazioni.
		
		\begin{description}
			\item[(\(\Rightarrow\))] Per ipotesi sia vera la \eqref{eq:nodo-formula-interp}; se \(f = p_{N - 1} \in \P_{N - 1}\), allora
			\begin{equation*}
				p_{N - 1} (x) = \sum_{i = 1}^N p_{N - 1} (x_i) L_i (x)
			\end{equation*}
			per ogni \(x \in (\, a, b \,)\) e, quindi,
			\begin{equation*}
				\begin{split}
					\int_a^b p_{N - 1} (x) w (x) \dd{x} &= \int_a^b \sum_{i = 1}^N p_{N - 1} (x_i) L_i (x) w (x) \dd{x} \\
					&= \sum_{i = 1}^N p_{N - 1} (x_i) \int_a^b L_i (x) w (x) \dd{x} \\
					&= \sum_{i = 1}^N w_i \, p_{N - 1} (x_i)
				\end{split}
			\end{equation*}
			il che mostra che la formula ha grado di precisione almeno \(N - 1\).
			\item[(\(\Leftarrow\))] Se la formula di quadratura è esatta per tutti i polinomi di grado \(N - 1\), lo è anche per i polinomi di Lagrange di tale grado. Dato che \(L_k (x_i) = \delta_{k, i}\), si ha per ogni \(k \in \Set{1, \dots, N}\)
			\begin{equation*}
				\int_a^b L_k (x) w (x) \dd{x} = \sum_{i = 1}^N w_i L_k (x_i) = \sum_{i = 1}^N w_i \delta_{k, i} = w_k
			\end{equation*}
			Ciò prova che la formula è interpolatoria.\qedhere
		\end{description}
	\end{proof}

	\begin{definizione}[Formule di Newton-Cotes chiuse]
		Dato un intervallo chiuso e limitato \([\, a, b \,] \subseteq \R\), una formula di quadratura definita come nella \eqref{eq:formula-quadratura} si dice \emph{di Newton-Cotes chiusa} se i nodi sono equispaziati, ovvero
		\begin{subequations}
			\begin{equation}
				x_i = a + \frac{(i - 1) (b - a)}{N - 1}
			\end{equation}
		al variare di \(i \in \Set{1, \dots, N}\), e i pesi sono definiti come
			\begin{equation}
				w_i = \int_a^b L_i (x) \dd{x}
			\end{equation}
		al variare di \(i \in \Set{1, \dots, N}\). Una tale formula è interpolatoria e ha grado di precisione almeno \(N - 1\).
		\end{subequations}
	\end{definizione}

	\begin{figure}[tpb]
		\centering
		
		\begin{tikzpicture}
			\begin{axis}[no marks, domain = 0:3, xmax = 3.14, ymax = 1.15, axis lines = center]
				\addplot+[smooth, black] {sin(deg(x))};
				\filldraw[fill = red, draw = red, fill opacity = 0.2] (0.5,0) -- (0.5, \drawsinlua{0.5}) -- (2, \drawsinlua{2}) -- (2,0) -- cycle;
				\filldraw[fill = blue, draw = blue, fill opacity = 0.1] (0.5, 0) -- (0.5, \drawsinlua{0.5}) to[parabola through={(1.25, \drawsinlua{1.25})}] (2, \drawsinlua{2}) -- (2,0) -- cycle;
			\end{axis}
		\end{tikzpicture}
		\caption{Confronto tra la regola del trapezio (in rosso) e la regola di Cavalieri-Simpson (in blu) per l'approssimazione di \(\int_{1 / 2}^2 \sin x \dd{x}\).}
	\end{figure}

	Vediamo ora alcuni casi particolari di formule di Newton-Cotes chiuse.
	
	\begin{definizione}[Regola del trapezio]\label{def:regola-trapezio}
		Si chiama \emph{regola del trapezio} la formula di quadratura
		\begin{equation}\label{eq:regola-trapezio}
			S_2 (f) = S_2 (f, a, b) = \frac{b - a}{2} \qty(f (a) + f (b))
		\end{equation}
	\end{definizione}

	Si può dimostrare che, posto \(h = b - a\), l'errore compiuto dalla formula in \eqref{eq:regola-trapezio} è pari a
	\begin{equation}\label{eq:regola-trapezio-errore}
		\mathcal{E}_2 (f) = I (f) - S_2 (f) = - \frac{h^3}{12} f'' (\xi)
	\end{equation}
	per un certo \(\xi \in (\, a, b \,)\), se si suppone che \(f\) sia derivabile due volte in \((\, a, b \,)\). Dalla \eqref{eq:regola-trapezio-errore} risulta evidente che \(S_2\) ha grado di precisione esattamente \(1\), perché
	\begin{equation*}
		f \in \P_1 \implies \forall \xi \in (\, a, b \,) \colon f'' (\xi) = 0 \implies \mathcal{E}_2 (f) = 0
	\end{equation*}
	mentre
	\begin{multline*}
		f \in \P_2 \setminus \P_1 \implies
		\begin{cases*}
			f (x) = a x^2 + b x + c \\
			a \ne 0
		\end{cases*} \\
		\implies \forall \xi \in (\, a, b \,) \colon f'' (\xi) = 2 a \ne 0 \implies \mathcal{E}_2 (f) \ne 0
	\end{multline*}

	\begin{osservazione}
		Posti \(x_1 = a\) e \(x_2 = b\), si ha
		\begin{align*}
			L_1 (x) &= \frac{x - b}{a - b} &
			L_2 (x) &= \frac{x - a}{b - a}
		\end{align*}
		e, visto che \(w \equiv 1\), si ottiene
		\begin{equation*}
			\begin{split}
				w_1 &= \int_a^b L_1 (x) \dd{x} = \int_a^b \frac{x - b}{a - b} \dd{x} \\
				&= \frac{1}{a - b} \int_a^b (x - b) \dd{x} = \frac{1}{a - b} \eval[\frac{(x - b)^2}{2}|_a^b \\
				&= \frac{1}{\cancel{a - b}} \frac{- (a - b)^{\cancel{2}}}{2} = \frac{b - a}{2}
			\end{split}
		\end{equation*}
		e
		\begin{equation*}
			\begin{split}
				w_2 &= \int_a^b L_2 (x) \dd{x} = \int_a^b \frac{x - a}{b - a} \dd{x} \\
				&= \frac{1}{b - a} \int_a^b (x - a) \dd{x} = \frac{1}{b - a} \eval[\frac{(x - a)^2}{2}|_a^b \\
				&= \frac{1}{\cancel{b - a}} \frac{(b - a)^{\cancel{2}}}{2} = \frac{b - a}{2}
			\end{split}
		\end{equation*}
		ovvero che la regola del trapezio è davvero una formula interpolatoria.
	\end{osservazione}

	\begin{definizione}[Regola di Cavalieri-Simpson]\label{def:regola-cavalieri}
		Si chiama \emph{regola di Cavalieri-Simpson} la formula di quadratura
		\begin{equation}\label{eq:regola-cavalieri}
			S_3 (f) = S_3 (f, a, b) = \frac{b - a}{6} \qty[f (a) + 4 f \qty(\frac{a + b}{2}) + f (b)]
		\end{equation}
	\end{definizione}

	Si può dimostrare che l'errore compiuto dalla formula in \eqref{eq:regola-cavalieri}, posto \(h = (b - a) / 2\), è
	\begin{equation}\label{eq:regola-cavalieri-errore}
		\mathcal{E}_3 (f) = I (f) - S_3 (f) = - \frac{h^5}{90} f^{(4)} (\xi)
	\end{equation}
	per un certo \(\xi \in (\, a, b \,)\), se si suppone che \(f\) sia derivabile quattro volte in \((\, a, b \,)\). Il grado di precisione di questa formula è esattamente \(3\), perché dalla \eqref{eq:regola-cavalieri-errore} segue che
	\begin{equation*}
		f \in \P_3 \implies \forall \xi \in (\, a, b \,) \colon f^{(4)} (\xi) = 0 \implies \mathcal{E}_3 (f) = 0
	\end{equation*}
	mentre
	\begin{multline*}
		f \in \P_4 \setminus \P_3 \implies
		\begin{cases*}
			f (x) = a x^4 + b x^3 + c x^2 + d x + e \\
			a \ne 0
		\end{cases*} \\
		\implies \forall \xi \in (\, a, b \,) \colon f^{(4)} (\xi) = 24 a \ne 0 \implies \mathcal{E}_3 (f) \ne 0
	\end{multline*}

	\begin{osservazione}
		Posti \(x_1 = a\), \(x_2 = (a + b) / 2\) e \(x_3 = b\), si ha
		\begin{align*}
			L_1 (x) &= \frac{(2 x - a - b) (x - b)}{(a - b)^2} \\
			L_2 (x) &= - 4 \frac{(x - a) (x - b)}{(a - b)^2} \\
			L_3 (x) &= \frac{(2 x - a - b)}{(b - a)^2}
		\end{align*}
		e, visto che \(w \equiv 1\), si ottiene
		\begin{equation*}
			\begin{split}
				w_1 &= \int_a^b L_1 (x) \dd{x} = \frac{1}{(a - b)^2} \int_a^b (2 x - a - b) (x - b) \dd{x} \\
				&= \frac{1}{(a - b)^2} \qty(\eval[(x^2 - a x - b x) (x - b)|_a^b - \eval[\frac{x^3}{3} - \frac{a x^2}{2} - \frac{b x^2}{2}|_a^b) \\
				&= \frac{1}{(a - b)^2} \qty(\frac{b^3}{6} - \frac{a b^2}{2} + \frac{a^2 b}{2} - \frac{a^3}{6}) \\
				&= \frac{(b - a)^3}{6 (a - b)^2} = \frac{b - a}{6}
			\end{split}
		\end{equation*}
		e
		\begin{equation*}
			\begin{split}
				w_2 &= \int_a^b L_2 (x) \dd{x} = - \frac{4}{(a - b)^2} \int_a^b (x - a) (x - b) \dd{x} \\
				&= - \frac{4}{(a - b)^2} \eval[\frac{x^3}{3} - \frac{a + b}{2} x^2 + a b x|_a^b \\
				&= - \frac{4}{(a - b)^2} \qty(\frac{a^3}{6} - \frac{a^2 b}{2} + \frac{a b^2}{2} - \frac{b^3}{6}) \\
				&= -\frac{4 (a - b)^3}{6 (a - b)^2} = \frac{2}{3} (b - a)
			\end{split}
		\end{equation*}
		e infine
		\begin{equation*}
			\begin{split}
				w_3 &= \int_a^b L_3 (x) \dd{x} = \frac{1}{(b - a)^2} \int_a^b (2 x - a - b) (x - a) \dd{x} \\
				&= \frac{1}{(b - a)^2} \qty(\eval[(x^2 - a x - b x) (x - a)|_a^b - \eval[\frac{x^3}{3} - \frac{a x^2}{2} - \frac{b x^2}{2}|_a^b) \\
				&= \frac{1}{(b - a)^2} \qty(\frac{a^3}{6} - \frac{a^2 b}{2} + \frac{a b^2}{2} - \frac{b^3}{6}) \\
				&= \frac{(b - a)^3}{6 (b - a)^2} = \frac{b - a}{6}
			\end{split}
		\end{equation*}
		ovvero che la regola di Cavalieri-Simpson risulta essere davvero una formula interpolatoria.
	\end{osservazione}

	Sia per la regola del trapezio che per quella di Cavalieri-Simpson si ricorre a stime che fanno uso del nucleo di Peano qualora le funzioni in esame non siano derivabili un numero sufficiente di volte a rendere ben definite la \eqref{eq:regola-trapezio-errore} e la \eqref{eq:regola-cavalieri-errore}.
	
	Ricordiamo, poi, che per \(N \ge 8\) le formule di Newton-Cotes chiuse hanno pesi di segno discorde; ciò le rende instabili per quanto riguarda la propagazione degli errori.
	
	Riportiamo altre formule di Newton-Cotes di grado superiore a \(2\), usando la notazione \(f_k = f (x_k)\) e ponendo \(h = (b - a) / (N - 1)\), ove \(N\) è il numero dei nodi:
		\begin{subequations}
			\begin{itemize}
				\item la formula dei tre ottavi
				\begin{equation}
					\frac{3h}{8} (f_1 + 3 f_2 + 3 f_3 + f_4)
				\end{equation}
				\item la regola di Milne-Boole
				\begin{equation}
					\frac{2 h}{45} (7 f_1 + 32 f_2 + 12 f_3 + 32 f_4 + 7 f_5)
				\end{equation}
				\item la formula a sei punti
				\begin{equation}
					\frac{5 h}{288} (19 f_1 + 75 f_2 + 50 f_3 + 50 f_4 + 75 f_5 + 19 f_6)
				\end{equation}
				\item la regola di Weedle-Hardy
				\begin{equation}
					\frac{h}{140} (41 f_1 + 216 f_2 + 27 f_3 + 272 f_4 + 27 f_5 + 216 f_6 + 41 f_7)
				\end{equation}
				\item la formula a otto punti
				\begin{multline}
					\frac{7 h}{17280} (751 f_1 + 3577 f_2 + 1323 f_3 + 2989 f_4 \\
					+ 2989 f_5 + 1323 f_6 + 3577 f_7 + 751 f_8)
				\end{multline}
				\item la formula a nove punti
				\begin{multline}
					\frac{4 h}{14175} (989 f_1 + 5888 f_2 - 928 f_3 + 10496 f_4 \\
					- 4540 f_5 + 10496 f_6 - 928 f_7 + 5888 f_8 + 989 f_9)
				\end{multline}
				\item la formula a dieci punti
				\begin{multline}
					\frac{9 h}{89600} (2857 f_1 + 15741 f_2 + 1080 f_3 + 19344 f_4 + 5788 f_5 \\
					+ 5788 f_6 + 19344 f_7 + 1080 f_8 + 15741 f_9 + 2857 f_{10})
				\end{multline}
				\item la formula a undici punti
				\begin{multline}
					\frac{5 h}{\num{2999376}} (16067 f_1 + 106300 f_2 - 48525 f_3 \\
					+ 272400 f_4 - 260550 f_5 + 427368 f_6 - 260550 f_7 \\
					+ 272400 f_8 - 48525 f_9 + 106300 f_{10} + 16067 f_{11})
				\end{multline}
			\end{itemize}
		\end{subequations}
	
\section{Formule di Newton-Cotes composte}
	
	\noindent Dato che le formule di Newton-Cotes sono generalmente instabili per \(N \ge 8\), è necessario trovare formule di quadratura che facciano uso di altrettanti nodi ma che siano numericamente stabili.
	
	\begin{definizione}[Formule di Newton-Cotes composte]
		Suddiviso l'intervallo \([\, a, b \,]\) in \(N\) subintervalli \(T_j = [\, x_j, x_{j + 1} \,]\) tali che \(x_j = a + j h_N\), ove si è posto \(h_N = (b - a) / N\), si dice \emph{formula di Newton-Cotes composta} una formula di quadratura del tipo
		\begin{equation}\label{eq:formule-composte}
			I (f) \approx \sum_{j = 0}^{N - 1} S (f, x_j, x_{j + 1})
		\end{equation}
		ove \(S\) è una formula di Newton-Cotes chiusa.
	\end{definizione}

	Questa definizione segue dalle proprietà dell'integrale:
	\begin{equation*}
		\int_a^b f (x) \dd{x} = \sum_{j = 0}^{N - 1} \int_{x_j}^{x_{j + 1}} f (x) \dd{x} \approx \sum_{j = 0}^{N - 1} S (f, x_j, x_{j + 1})
	\end{equation*}

	\begin{figure}[tpb]
		\centering
		
		\begin{tikzpicture}
			\begin{axis}[no marks, domain = 0:3, xmax = 3.14, ymax = 1.15, axis lines = center]
				\addplot+[smooth, black] {sin(deg(x))};
				\filldraw[fill = red, draw = red, fill opacity = 0.2] (0.5,0) -- (0.5, \drawsinlua{0.5}) -- (1, \drawsinlua{1}) -- (1.5, \drawsinlua{1.5}) -- (2, \drawsinlua{2}) -- (2,0) -- cycle;
				\draw[red] (1, \drawsinlua{1}) -- (1,0) (1.5, \drawsinlua{1.5}) -- (1.5,0);
			\end{axis}
		\end{tikzpicture}
		\caption{Rappresentazione dell'approssimazione di \(\int_{1 / 2}^2 \sin x \dd{x}\) tramite la regola dei trapezi.}
	\end{figure}

	Vediamo ora delle formule di Newton-Cotes composte particolari.
	
	\begin{definizione}[Formula dei trapezi]
		Si chiama \emph{formula dei trapezi} o \emph{del trapezio composta} la formula di quadratura su \([\, a, b \,]\)
		\begin{equation}\label{eq:formula-trapezi}
			S_2^{(c)} (f, N) = \frac{b - a}{N} \qty[\frac{f (x_0)}{2} + f (x_1) + \dots + f (x_{N - 1}) + \frac{f (x_N)}{2}]
		\end{equation}
		ove \(x_i = a + i h_N\) per \(i \in \Set{0, \dots, N}\).
	\end{definizione}

	Si può dimostrare che, se \(f\) è derivabile due volte in \((\, a, b \,)\), l'errore commesso dalla \eqref{eq:formula-trapezi} è pari a
	\begin{equation}\label{eq:formula-trapezi-errore}
		\mathcal{E}_2^{(c)} (f) = I (f) - S_2^{(c)} (f, N) = - \frac{b - a}{12} h_N^2 f'' (\xi)
	\end{equation}
	per un certo \(\xi \in (\, a, b \,)\). Il grado di precisione di questa formula è ancora \(1\), ma per \(N \ge 1\) si ha \(h_N \le h\), da cui segue che \(\abs{\mathcal{E}_2^{(c)} (f)} \le \abs{\mathcal{E}_2 (f)}\).
	
	Sotto ipotesi piú stringenti è possibile migliorare la stima dell'andamento dell'errore \(\mathcal{E}_2^{(c)} (f)\), che in base alla \eqref{eq:formula-trapezi-errore} è \(\mathcal{O} (1 / N^2)\).
	
	\begin{teorema}[Formula di Eulero-Maclaurin]
		Se \(f \in \cont^{2 M + 2} ([\, a, b \,])\), allora esiste \(\xi \in (\, a, b \,)\) tale che
		\begin{multline}\label{eq:formula-eulero-maclaurin}
			\int_a^b f (x) \dd{x} = S_2^{(c)} (f, N) - \qty[\sum_{k = 1}^M \frac{B_{2 k}}{(2 k)!} h_N^{2 k} \qty(f^{(2 k - 1)} (b) - f^{(2 k - 1)}) (a)] \\
			- \frac{B_{2 M + 2}}{(2 M + 2)!} h_N^{2 M + 2} (b - a) f^{(2 M + 2)} (\xi)
		\end{multline}
		ove i \(B_k\) sono i numeri di Bernoulli.
	\end{teorema}

	Se ci poniamo nel caso particolare in cui \(f^{(2 k -1)} (a) = f^{(2 k - 1)} (b)\) per ogni \(k \in \Set{1, \dots, M}\), la sommatoria nella \eqref{eq:formula-eulero-maclaurin} si annulla e diviene evidente che \(\mathcal{E}_2^{(c)} (f)\) è in tal caso \(\mathcal{O} (1 / N^{2 M + 2})\). In casi ancor piú specifici, l'errore decresce esponenzialmente.
	
	\begin{teorema}
		Se \(f \colon [\, 0, 2 \pi \,] \to \R\) è una funzione analitica e periodica di periodo \(2 \pi\) tale che esistano \(a, M > 0\) tali che \(\abs{f (z)} \le M\) per ogni \(z \in \C\) tale che \(\Im z > - a\), allora per ogni \(N \in \N^*\) \(f\) verifica
		\begin{equation}
			\abs{S_2^{(c)} (f, N) - I (f)} \le \frac{2 \pi M}{\ee^{a N} - 1}
		\end{equation}
		e la costante \(2 \pi\) è la minima costante che renda vera tale diseguaglianza.
	\end{teorema}

	\begin{definizione}[Formula di Cavalieri-Simpson composta]
		Si chiama \emph{formula di Cavalieri-Simpson composta} la formula di quadratura su \([\, a, b \,]\)
		\begin{equation}\label{eq:formula-simpson-composta}
			S_3^{(c)} (f, N) = \frac{h_N}{6} \qty[f (x_0) + 2 \sum_{r = 1}^{N - 1} f (x_{2 r}) + 4 \sum_{s = 0}^{N - 1} f (x_{2 s + 1}) + f (x_{2 N})]
		\end{equation}
		ove \(x_k = a + k h_N / 2\) per \(k \in \Set{0, \dots, 2 N}\).
	\end{definizione}

	Si può dimostrare che, se \(f\) è derivabile quattro volte in \((\, a, b \,)\), l'errore compiuto dalla \eqref{eq:formula-simpson-composta} è
	\begin{equation}
		\mathcal{E}_3^{(c)} (f) = I (f) - S_3^{(c)} (f, N) = - \frac{b - a}{180} \qty(\frac{h_N}{2})^4 f^{(4)} (\xi)
	\end{equation}
	per un certo \(\xi \in (\, a, b \,)\). Il grado di precisione di questa formula è esattamente \(3\), ma, analogamente alla regola dei trapezi composta, si ha che \(\abs{\mathcal{E}_3^{(c)} (f)} \le \abs{\mathcal{E}_3 (f)}\) perché \(h_N \le h\).
	
	\begin{figure}[tpb]
		\centering
		
		\begin{tikzpicture}
			\begin{axis}[no marks, domain = 0:3, xmax = 3.14, ymax = 1.15, axis lines = center]
				\addplot+[smooth, black, samples = 350] {sin(deg(x))};
				\filldraw[fill = blue, draw = blue, fill opacity = 0.1] (0.5,0) -- (0.5, \drawsinlua{0.5}) to[parabola through={(0.875, \drawsinlua{0.875})}] (1.25, \drawsinlua{1.25}) to[parabola through={(1.625, \drawsinlua{1.625})}] (2,\drawsinlua{2}) -- (2,0) -- cycle;
				\draw[densely dashed, blue] (0.875, 0) -- (0.875, \drawsinlua{0.875}) (1.625, 0) -- (1.625, \drawsinlua{1.625});
				\draw[blue] (1.25, \drawsinlua{1.25}) -- (1.25,0);
			\end{axis}
		\end{tikzpicture}
		\caption{Rappresentazione dell'approssimazione di \(\int_{1 / 2}^2 \sin x \dd{x}\) tramite la regola di Cavalieri-Simpson composta.}
	\end{figure}
	
\section{Formule gaussiane}
	
	\noindent Le formule interpolatorie di Newton-Cotes presuppongono sempre che i nodi in esame siano equispaziati e hanno grado di precisione almeno \(N - 1\), ove \(N\) è il numero dei nodi, benché le formule con un numero dispari di nodi come quella di Cavalieri-Simpson abbiano grado di precisione esattamente \(N\).
	
	Vogliamo ora studiare formule che si possano applicare anche per integrali su intervalli \((\, a, b \,)\) non necessariamente limitati e valide per funzioni peso diverse da \(w \equiv 1\); vogliamo, poi, trovare formule che abbiano un grado di precisione piú alto a parità di nodi.
	
	Il problema di dover calcolare un integrale di una funzione \(f \colon (\, a, b \,) \to \R\) come nella \eqref{eq:integrale-con-peso} è di gran lunga piú generale rispetto all'integrazione di una funzione continua su un intervallo chiuso e limitato: l'integranda \(f w\), infatti, potrebbe non essere continua in \([\, a, b \,]\), come ad esempio accade usando la funzione peso di Chebyshev, discontinua in \(\pm 1\); l'intervallo, poi, potrebbe non essere nemmeno limitato, come accade quando si usa la funzione peso di Laguerre o quella di Hermite.
	
	\begin{teorema}\label{th:formula-gaussiana-esiste-unica}
		Per ogni \(n \in \N^*\) esistono un'unica \(n\)-upla \(x_1, \dots, x_n\) di nodi e un'unica \(n\)-upla di pesi \(w_1, \dots, w_n\) tali che il grado di precisione della formula di quadratura che ha questi come nodi e come pesi abbia grado di precisione almeno \(2 n - 1\). Tali nodi sono gli zeri del polinomio ortogonale rispetto a \(w\) di grado \(n\) \(\varphi_n (x) = A_n (x - x_1) \cdots (x - x_n)\), mentre i pesi sono individuati per ogni \(i \in \Set{1, \dots, n}\) da
		\begin{equation}\label{eq:formule-gauss-pesi}
			w_i = \int_a^b L_i (x) w (x) \dd{x} = \int_a^b L_i^2 (x) w (x) \dd{x}
		\end{equation}
	\end{teorema}

	\begin{proof}
		Dimostriamo innanzitutto che una formula di quadratura cosí definita ha grado di precisione almeno \(2 n - 1\). Siano dunque \(p_{2 n - 1} \in \P_{2 n - 1}\) e \(q_{n - 1}, r_{n - 1} \in \P_{n - 1}\) tali che
		\begin{equation*}
			p_{2 n - 1} = q_{n - 1} \varphi_n + r_{n - 1}
		\end{equation*}
		Poiché \(\varphi_n\) è il polinomio ortogonale rispetto a \(w\) di grado \(n\), si ha
		\begin{multline*}
			\int_a^b q_{n - 1} (x) \varphi_n (x) w (x) \dd{x} = (q_{n - 1}, \varphi_n)_{2, w} \\
			= \qty(\sum_{j = 0}^{n - 1} \gamma_j \varphi_j, \varphi_n)_{2, w} = \sum_{j = 0}^{n - 1} \gamma_j (\varphi_j, \varphi_n)_{2, w} = 0
		\end{multline*}
		Dalla \eqref{eq:formule-gauss-pesi} è evidente che la formula di quadratura definita è interpolatoria e, quindi, esatta su ogni polinomio di grado \(n - 1\), in quanto basata su \(n\) punti a due a due distinti. Dato che, per ogni zero \(x_k\) di \(\varphi_n\) si ha
		\begin{equation*}
			p_{2 n - 1} (x_k) = q_{n - 1} (x_k) \varphi_n (x_k) + r_{n - 1} (x_k) = r_{n - 1} (x_k)
		\end{equation*}
		segue che
		\begin{equation*}
			\begin{split}
				\int_a^b p_{2 n - 1} (x) w (x) \dd{x} &= \cancel{\int_a^b q_{n - 1} (x) \varphi_n (x) w (x) \dd{x}} + \int_a^b r_{n - 1} (x) w (x) \dd{x} \\
				&= \int_a^b r_{n - 1} (x) w (x) \dd{x} = \sum_{k = 1}^n w_k \, r_{n - 1} (x_k) \\
				&= \sum_{k = 1}^n w_k \, p_{2 n - 1} (x_k)
			\end{split}
		\end{equation*}
		ovvero che la formula di quadratura ha grado di precisione almeno \(2 n - 1\).
		
		Visto che \(\deg L_i^2 = 2 (n - 1) < 2 n - 1\) per ogni \(i \in \Set{1, \dots, n}\), la formula di quadratura in esame integra esattamente i quadrati dei polinomi di Lagrange; da ciò segue che per ogni \(i \in \Set{1, \dots, n}\)
		\begin{equation*}
			0 < \int_a^b L_i^2 (x) w (x) \dd{x} = \sum_{k = 1}^n w_k L_i^2 (x_k) = \sum_{k = 1}^n w_k \delta_{k, i} = w_i
		\end{equation*}
	
		Supponiamo ora che esista un'altra formula interpolatoria con grado di precisione almeno \(2 n - 1\) che abbia nodi \(\tilde{x}_1, \dots, \tilde{x}_n\) e pesi \(\tilde{w}_1, \dots, \tilde{w}_n\). Per quanto visto sopra, dal grado di precisione della formula segue che \(\tilde{w}_j > 0\) per ogni \(j \in \Set{1, \dots, n}\). A meno di cambiare gli indici, possiamo supporre che sia i nodi \(x_j\) sia i nodi \(\tilde{x}_j\) abbiano ordine crescente. Indicato con \(\tilde{L}_j\) il \(j\)-esimo polinomio di Lagrange per i nodi \(\tilde{x}_j\), che ha grado \(n - 1\), si ha
		\begin{multline*}
				0 = \qty(\tilde{L}_j, \varphi_n)_{2, w} = \int_a^b \tilde{L}_j (x) \varphi_n (x) w (x) \dd{x} \\
				= \sum_{k = 1}^n \tilde{w}_k \tilde{L}_j (\tilde{x}_k) \varphi_n (\tilde{x}_k) = \tilde{w}_j \varphi_n (\tilde{x}_j)
		\end{multline*}
		e, dato che \(\tilde{w}_j > 0\) per ogni \(j\), risulta che \(\tilde{x}_j\) è uno zero di \(\varphi_n\). Ciò prova che i \(\tilde{x}_j\) sono al piú \(n\), ma potrebbero essere meno. Per assurdo siano in numero \(m < n\): potremmo costruire il polinomio di grado \(m\)
		\begin{equation*}
			p_m (x) = \prod_{j = 1}^m \qty(x - \tilde{x}_j) \ne 0
		\end{equation*}
		che, elevato al quadrato, avrebbe grado \(2 m \le 2 (n - 1) = 2 n - 2\) e, quindi,
		\begin{equation*}
			0 < \int_a^b p_m^2 (x) w (x) \dd{x} = \sum_{j = 1}^m \tilde{w}_j \prod_{k = 1}^m \qty(\tilde{x}_j - \tilde{x}_k)^2 = 0
		\end{equation*}
		che è assurdo. Da ciò si ottiene che i nodi \(\tilde{x}_j\) sono esattamente \(n\), ovvero \(x_j = \tilde{x}_j\) per ogni \(j \in \Set{1, \dots, n}\) e, dato che \(L_j = \tilde{L}_j\), si conclude anche che \(w_j = \tilde{w}_j\) per ogni \(j\).
	\end{proof}

	\begin{osservazione}
		Una formula gaussiana a \(n\) nodi non può avere grado di precisione superiore a \(2 n - 1\). Posto, infatti, \(p_n (x) = \prod_{j = 1}^n (x - x_j) \ne 0\), il polinomio \(p_n^2 (x)\) avrebbe grado \(2 n\) e, quindi,
		\begin{equation*}
			0 < \int_a^b p_n^2 (x) w (x) \dd{x} = \sum_{j = 1}^n w_j \prod_{k = 1}^n (x_j - x_k)^2 = 0
		\end{equation*}
		che è assurdo.
	\end{osservazione}

\section[Stima dell'errore]{Stima dell'errore delle formule di quadratura}
	
	\begin{teorema}
		Indichiamo con \(I_n (f) = \sum_{i = 1}^n w_{i, n} f (x_{i, n})\) la regola di Newton-Cotes a \(n\) nodi su \([\, a, b \,]\) e poniamo \(h = (b - a) / (n - 1)\). Se \(n\) è dispari e \(f \in \cont^{n + 1} ([\, a, b \,])\), allora esiste \(\eta \in (\, a, b \,)\) tale che
		\begin{subequations}
			\begin{equation}
				I (f) - I_n (f) = C_n h^{n + 2} f^{(n + 1)} (\eta)
			\end{equation}
			ove
			\begin{equation}
				C_n = \frac{1}{(n + 1)!} \int_0^n \mu^2 (\mu - 1) \cdots (\mu - n - 1) \dd{\mu}
			\end{equation}
		\end{subequations}
		Se \(n\) è pari e \(f \in \cont^{n} ([\, a, b \,])\), allora esiste \(\eta \in (\, a, b \,)\) tale che
		\begin{subequations}
			\begin{equation}
				I (f) - I_n (f) = C_n h^{n + 1} f^{(n)} (\eta)
			\end{equation}
			ove
			\begin{equation}
				C_n = \frac{1}{n!} \int_0^{n - 1} \mu (\mu - 1) \cdots (\mu - n - 1) \dd{\mu}
			\end{equation}
		\end{subequations}
	\end{teorema}

	Questo risultato generale conferma quanto trovato nella \eqref{eq:regola-trapezio-errore} e nella \eqref{eq:regola-cavalieri-errore}.
	
	\begin{teorema}
		Chiamata \(I_n (f)\) la formula gaussiana a \(n\) nodi relativa all'integrale \(I_w (f)\) con funzione peso \(w\) sull'intervallo limitato \((\, a, b \,)\), se \(f \in \cont^{2 n} ((\, a, b \,))\), allora esiste \(\eta \in (\, a, b \,)\) tale che
		\begin{equation}
			I_w (f) - I_n (f) = \frac{\gamma_n}{A_n^2 (2 n)!} f^{(2 n)} (\eta)
		\end{equation}
		ove \(A_n\) è il coefficiente di grado \(n\) del polinomio ortogonale \(\varphi_n\) rispetto a \(w\) e \(\gamma_n = \int_a^b \varphi_n^2 (x) w (x) \dd{x}\).
	\end{teorema}

	\begin{figure}[tpb]
		\centering
		
		\begin{tikzpicture}
			\begin{axis}[ymode=log, only marks, legend entries = {Newton-Cotes, Gauss-Legendre}, legend pos = north east, xtick = {0, 5, ..., 35}]
				\addplot table {risorse/newcot.dat};
				\addplot table {risorse/errgauss.dat};
			\end{axis}
		\end{tikzpicture}
		
		\caption{Confronto tra i fattori di errore indipendenti da \(f\) relativi alle formule di Newton-Cotes e alle formule di Gauss-Legendre.}\label{fig:legendre-newton}
	\end{figure}

	In particolare, se \(w \equiv 1\) sull'intervallo \((\, -1, 1 \,)\), allora l'errore commesso dalla formula di Gauss-Legendre è pari a
	\begin{equation*}
		\mathcal{E}_n (f) = I (f) - I_n (f) = \frac{2^{2 n + 1} (n!)^4}{(2 n + 1) [(2 n)!]^3} f^{(2 n)} (\eta) \qquad \exists \eta \in (\, -1, 1 \,)
	\end{equation*}
	Nella Figura~\ref{fig:legendre-newton} è evidente che il fattore di \(\mathcal{E}_n (f)\) indipendente da \(f\) abbia una decrescita di gran lunga piú veloce rispetto all'errore commesso con le formule di Newton-Cotes chiuse.

\section[Stabilità delle formule]{Stabilità di una formula di quadratura}
	
	\noindent Considerata una funzione peso \(w\) sull'intervallo generalizzato \((\, a, b \,)\), supponiamo di avere una formula di quadratura \(S_n\) tale che
	\begin{equation*}
		I_w (f) \approx S_n (f) = \sum_{j = 1}^n w_j f_j
	\end{equation*}
	ove \(f_j = f (x_j)\) con gli \(x_j\) scelti secondo quanto richiesto da \(f\). In questa sezione vogliamo capire quanto errore si commetta sostituendo gli \(f_j\) con dei valori perturbati \(\tilde{f}_j\), ovvero calcolando
	\begin{equation*}
		I_w (f) \approx \tilde{S}_n (f) = \sum_{j = 1}^n w_j \tilde{f}_j
	\end{equation*}

	Per la disuguaglianza triangolare si ha
	\begin{multline*}
		\abs{S_n (f) - \tilde{S}_n (f)} = \abs{\sum_{j = 1}^n w_i \qty(f_j - \tilde{f}_j)} \\
		\le \sum_{j = 1}^n \abs{w_j} \abs{f_j - \tilde{f}_j} \le \qty(\sum_{j = 1}^n \abs{w_j}) \max_j \abs{f_j - \tilde{f}_j}
	\end{multline*}
	e il termine \(\sum_{j = 1}^n \abs{w_j}\) è l'\emph{indice di stabilità} della formula di quadratura \(S_n\).
	
	Se supponiamo che \(S_n\) abbia grado di precisione almeno \(0\) e individuiamo i pesi positivi \(w_1^+, \dots, w_{n^+}^+\) e negativi \(w_1^-, \dots, w_{n^-}^-\), allora
	\begin{equation*}
		\begin{split}
			\int_a^b w (x) \dd{x} &= \int_a^b 1 \cdot w (x) \dd{x} = \sum_{j = 1}^n w_j = \sum_{j = 1}^{n^+} w_j^+ + \sum_{j = 1}^{n^-} w_j^- \\
			&= \sum_{j = 1}^{n^+} \abs{w_j^+} + \qty(\sum_{j = 1}^{n^-} \abs{w_j^-} - \sum_{j = 1}^{n^+} \abs{w_j^+}) - \sum_{j = 1}^{n^-} \abs{w_j^-} \\
			&= \sum_{j = 1}^n \abs{w_j} - 2 \sum_{k = 1}^{n^-} \abs{w_k^-}
		\end{split}
	\end{equation*}
	da cui segue che
	\begin{equation}
		\sum_{j = 1}^n \abs{w_j} = \int_a^b w (x) \dd{x} + 2 \sum_{k = 1}^{n^-} \abs{w_k^-}
	\end{equation}
	In base a ciò, si può affermare che la presenza di pesi negativi peggiora l'indice di stabilità della formula \(S_n\). Se, invece, \(S_n\) non prevede pesi negativi, allora \(\sum_{j = 1}^n \abs{w_j} = \int_a^b w (x) \dd{x}\).
	
	La prossima Proposizione mostra che l'indice di stabilità di una formula \(S_n\) coincide di fatto con \(\norm{S}_\infty\) nel caso di intervalli di integrazione limitati.
	
	\begin{proposizione}\label{prop:norma-formula-quadratura}
		Se \([\, a, b \,]\) è un intervallo limitato, allora l'operatore
		\begin{equation*}
			\begin{array}{rccl}
				S \colon & \cont ([\, a, b \,]) & \to & \R \\
				& f & \mapsto & \sum_{j = 1}^n w_j f_j
			\end{array}
		\end{equation*}
		è lineare e continuo in norma \(\norm{\cdot}_\infty\) ed ha norma pari a \(\sum_{j = 1}^{n} \abs{w_j}\).
	\end{proposizione}

	\begin{proof}
		Per il teorema di Weierstrass è ben definita la scrittura \(\norm{f}_\infty\) per ogni \(f \in \cont ([\, a, b \,])\). Si ha
		\begin{equation*}
			\abs{S (f)} = \abs{\sum_{j = 1}^n w_j f_j} \le \sum_{j = 1}^n \abs{w_j} \, \abs{f_j} \le \norm{f}_\infty \sum_{j = 1}^n \abs{w_j}
		\end{equation*}
		ovvero \(S\) è un operatore lineare e continuo e con norma limitata superiormente da \(\sum_{j = 1}^{n} \abs{w_j}\). Scegliendo una \(f\) opportuna, è possibile mostrare che questa limitazione superiore è effettivamente un'uguaglianza.
	\end{proof}

	\begin{proposizione}\label{prop:norma-integrale}
		Se \([\, a, b \,]\) è un intervallo limitato, allora l'operatore
		\begin{equation*}
			\begin{array}{rccl}
				I_w \colon & \cont ([\, a, b \,]) & \to & \R \\
				& f & \mapsto & \int_a^b f (x) w(x) \dd{x}
			\end{array}
		\end{equation*}
		verifica la proprietà \(\norm{I_w}_\infty = \int_a^b w (x) \dd{x} = \norm{w}_1\).
	\end{proposizione}
	
	\begin{proof}
		Poiché
		\begin{multline*}
			\abs{I_w (f)} = \abs{\int_a^b f (x) w (x) \dd{x}} \le \int_a^b \abs{f (x)} \abs{w (x)} \dd{x} \\
			\le \norm{f}_\infty \int_a^b \abs{w (x)} \dd{x} = \norm{f}_\infty \norm{w}_1
		\end{multline*}
		si ottiene la disuguaglianza \(\norm{I_w}_\infty \le \norm{w}_1\). Il fatto che \(\abs{I_w (1)} = \norm{w}_1\) permette di concludere.
	\end{proof}

	\begin{teorema}[Stieltjes]\label{th:stieltjes}
		Dati \(a, b \in \R\) con \(a < b\), una funzione \(f \in \cont ([\, a, b \,])\) e una funzione peso \(w \colon (\, a, b \,) \to \R\), se la formula di quadratura \(I_n (f) = \sum_{j = 1}^{\eta_n} w_j f_j\) ha grado di precisione almeno \(n\), allora
		\begin{equation}\label{eq:stieltjes}
			\abs{\mathcal{E}_n (f)} \coloneqq \abs{I_w (f) - I_n (f)} \le \qty(\norm{w}_1 + \sum_{j = 1}^{\eta_n} \abs{w_j}) \min_{q \in \P_n} \norm{f - q_n}_\infty
		\end{equation}
	\end{teorema}

	\begin{proof}
		Dalle ipotesi segue subito che \(I_w (q_n) = I_n (q_n)\) per ogni polinomio \(q_n \in \P_n\). Per la linearità di \(I_w\) e \(I_n\), si ha per ogni \(q_n \in \P_n\)
		\begin{align*}
			I_n (f - q_n) &= I_n (f) - I_n (q_n) &
			I_w (f - q_n) &= I_w (f) - I_w (q_n)
		\end{align*}
		e dalle Proposizioni~\ref{prop:norma-formula-quadratura} e \ref{prop:norma-integrale} è noto che
		\begin{align*}
			\abs{I_n (f)} &\le \sum_{j = 1}^{\eta_n} \abs{w_j} \norm{f}_\infty &
			\abs{I_w (f)} &\le \norm{w}_1 \norm{f}_\infty
		\end{align*}
		
		Sia ora \(q_n^* \in \P_n\) il polinomio di miglior approssimazione per \(f\) relativamente alla norma \(\norm{\cdot}_\infty\). Per quanto visto sopra, si ha
		\begin{equation*}
			\begin{split}
				\abs{\mathcal{E}_n (f)} &= \abs{I_w (f) - I_n (f)} = \abs{I_w (f) - I_n (q_n^*) + I_n (q_n^*) - I_n (f)} \\
				&\le \abs{I_w (f) - I_n (q_n^*)} + \abs{I_n (f) - I_n (q_n^*)} \\
				&= \abs{I_w (f - q_n^*)} + \abs{I_n (f - q_n^*)} \\
				&\le \norm{w}_1 \norm{f - q_n^*}_\infty + \sum_{j = 1}^{\eta_n} \abs{w_j} \, \norm{f - q_n^*}_\infty \\
				&= \qty(\norm{w}_1 + \sum_{j = 1}^{\eta_n} \abs{w_j}) \min_{q \in \P_n} \norm{f - q_n}_\infty \qedhere
			\end{split}
		\end{equation*}
	\end{proof}

	\begin{esempio}
		Consideriamo una formula di quadratura a pesi positivi e con grado di precisione almeno \(n \ge 0\); si ha \(\norm{I_n}_\infty = \norm{w}_1 = \norm{I_w}_\infty\), in quanto la formula \(I_n\) integra esattamente la funzione costantemente uguale a \(1\). Dal Teorema~\ref{th:stieltjes} si ricava
		\begin{equation}
			\abs{I_w (f) - I_n (f)} \le 2 \norm{w}_1 E_n (f)
		\end{equation}
		con \(E_n (f)\) definito come nella \eqref{eq:errore-approx}.
		
		Se, ad esempio, \(w \equiv 1\) sull'intervallo \((\, -1, 1 \,)\), allora \(\abs{\mathcal{E}_n (f)} \le 4 E_n (f)\).
	\end{esempio}

	Il Teorema~\ref{th:stieltjes} gode di particolare importanza per via del rapporto tra errore d'integrazione e polinomio di miglior approssimazione. Osservando i termini nel membro di destra della \eqref{eq:stieltjes}, si può vedere che il primo dipende solo dalla stabilità della funzione di quadratura e dalla scelta della funzione peso, mentre l'altro dipende soltanto dalla miglior approssimazione dell'integranda \(f\), e non dipende dal prodotto \(f w\). Di conseguenza, usare un'appropriata formula gaussiana rispetto alla funzione peso \(w\) garantisce risultati migliori anche quando \(f w\) non è sufficientemente regolare, se \(f\) invece lo è; ciò è in accordo coi teoremi di Jackson, che offrono stime per \(E_n (f)\) proprio in funzione della regolarità di \(f\), che però può anche essere soltanto continua.
	
	\begin{esempio}
		Si supponga di dover approssimare numericamente l'integrale
		\begin{equation*}
			\int_{-1}^1 \ee^x \sqrt{1 - x^2}
		\end{equation*}
		con la formula di Gauss-Legendre e con la formula di Gauss-Jacobi avente come esponenti \(\alpha = 1 / 2\) e \(\beta = 0\). Nel primo caso, si sta considerando una funzione che non è derivabile in \(\pm 1\), mentre nel secondo caso la funzione da considerare non peso è non derivabile solo in \(-1\). Se, invece, si usa la formula di quadratura di Gauss-Jacobi con \(\alpha = \beta = 1/2\), la funzione integranda è analitica, e quindi l'errore è di gran lunga inferiore. La Figura~\ref{fig:legendre-jacobi} mostra a confronto questi tre approcci.
		
		\begin{figure}[tpb]
			\centering
			
			\begin{tikzpicture}
				\begin{axis}[scale=1.1, ymode=log, only marks, mark size = 0.8pt, legend entries = {G.-Legendre, {G.-Jacobi \(\alpha = \frac{1}{2}\), \(\beta = 0\)}, {G.-Jacobi \(\alpha = \frac{1}{2}\), \(\beta = \frac{1}{2}\)}}, legend pos = north east]
					\addplot table {risorse/gaussleg.dat};
					\addplot table {risorse/gaussjac.dat};
					\addplot table {risorse/gaussjac2.dat};
				\end{axis}
			\end{tikzpicture}
			
			\caption{Confronto tra gli errori relativi commessi dalle formule di Gauss-Legendre e di Gauss-Jacobi per calcolare \(\int_{-1}^1 \ee^x \sqrt{1 - x^2} \dd{x}\). La stabilizzazione dell'errore relativo della terza formula è dovuto ad approssimazioni in aritmetica di macchina.}\label{fig:legendre-jacobi}
		\end{figure}
	\end{esempio}

\section[Convergenza delle formule]{Convergenza di una formula di quadratura}
	
	\noindent Se una funzione peso \(w\) è definita su un intervallo \((\, a, b \,)\) limitato e si ha \(f \in \cont([\, a, b \,])\), allora \(f w \in L^1 ((\, a, b \,))\). Definita una famiglia di formule \(\Set{S_n : n \in \N}\), non necessariamente con grado di precisione \(n\), tali che
	\begin{equation}\label{eq:famiglia-formule}
		I_w (f) \approx S_n (f) = \sum_{i = 1}^{\eta_n} w_{i, n} f (x_{i, n})
	\end{equation}
	studiamo quando l'errore della formula \(n\)-esima \(\mathcal{E}_n (f) = I_w (f) - S_n (f)\) tenda a \(0\) per \(n \to + \infty\).
	
	\begin{teorema}[Polya-Steklov]\label{th:polya-steklov}
		Dato un intervallo compatto \([\, a, b \,]\), sia \(\Set{S_n : n \in \N}\) una successione di formule di quadratura come nella \eqref{eq:famiglia-formule}. Si verifica per ogni \(f \in \cont ([\, a, b \,])\) che \(\lim_{n \to \infty} \mathcal{E}_n (f) = 0\) se e solo se esiste \(M \in \R\) tale che per ogni \(n \in \N\) si abbia \(\sum_{i = 1}^{\eta_n} \abs{w_{i, n}} \le M\) e per ogni \(k \in \N\) si abbia \(\lim_{n \to \infty} \mathcal{E}_n (x^k) = 0\). 
	\end{teorema}

	\begin{proof}
		Mostriamo entrambe le implicazioni.
		
		\begin{description}
			\item[(\(\Leftarrow\), Steklov)] In base al teorema di densità di Weierstrass, per ogni \(\varepsilon > 0\) esiste un polinomio \(p\) tale che \(\norm{f - p}_\infty \le \varepsilon\). Fissato \(n \in \N\), dalle Proposizioni~\ref{prop:norma-formula-quadratura} e \ref{prop:norma-integrale} si ha per ogni \(g \in \cont ([\, a, b \,])\)
			\begin{align*}
				\abs{I_w (g)} & \le \norm{I_w}_\infty \norm{g}_\infty = \norm{w}_1 \norm{g}_\infty \\
				\abs{S_n (g)} & \le \norm{S_n}_\infty \norm{g}_\infty = \sum_{i = 1}^{\eta_n} \abs{w_{i, n}} \, \norm{g}_\infty
			\end{align*}
			Da ciò segue che, posto \(g = f - p\),
			\begin{equation*}
				\begin{split}
					\abs{\mathcal{E}_n (f - p)} &= \abs{I_w (f - p) - S_n (f - p)} \le \abs{I_w (f - p)} + \abs{S_n (f - p)} \\
					& \le \norm{w}_1 \norm{f - p}_\infty + \sum_{i = 1}^{\eta_n} \abs{w_{i, n}} \, \norm{f - p}_\infty \\
					&= \qty(\norm{w}_1 + \sum_{i = 1}^{\eta_n} \abs{w_{i, n}}) \norm{f - p}_\infty \\
					& \le \qty(\norm{w}_1 + M) \varepsilon
				\end{split}
			\end{equation*}
			e questa stima è indipendente da \(n\). Dato che, per ipotesi, \(\abs{\mathcal{E}_n (p)} \to 0\) per le proprietà dei limiti, si deduce che
			\begin{equation*}
				\abs{\mathcal{E}_n (f)} \le \abs{\mathcal{E}_n (f - p)} + \abs{\mathcal{E}_n (p)} \le \qty(\norm{w}_1 + M) \varepsilon + \abs{\mathcal{E}_n (p)}
			\end{equation*}
			e, poiché la convergenza di \(\abs{\mathcal{E}_n (p)}\) garantisce che esista \(n \in \N\) tale che \(\abs{\mathcal{E}_n (p)} \le \varepsilon\), si ottiene
			\begin{equation*}
				\abs{\mathcal{E}_n (f)} \le \qty(\norm{w}_1 + M + 1) \varepsilon
			\end{equation*}
			da cui si conclude per l'arbitrarietà di \(\varepsilon\).
			\item[(\(\Rightarrow\), Polya)] Poiché \(\mathcal{E}_n (f) = I_w (f) - S_n (f)\), si ha \(S_n (f) = I_w (f) - \mathcal{E}_n (f)\) e, per la disuguaglianza triangolare e per la Proposizione~\ref{prop:norma-integrale},
			\begin{equation*}
				\abs{S_n (f)} \le \abs{I_w (f)} + \abs{\mathcal{E}_n (f)} \le \norm{w}_1 \norm{f}_\infty + \abs{\mathcal{E}_n (f)}
			\end{equation*}
			 Per ipotesi \(\mathcal{E}_n (f) \to 0\), quindi anche \(\abs{\mathcal{E}_n (f)} \to 0\); per la definizione di limite, esiste \(M_f \in \R\) dipendente solo da \(f\) tale che
			 \begin{equation*}
			 	\abs{S_n (f)} \le \norm{w}_1 \norm{f}_\infty + \abs{\mathcal{E}_n (f)} \le M_f < + \infty
			 \end{equation*}
		 	In base al teorema di Banach-Steinhaus,\footnote{Il teorema di Banach-Steinhaus asserisce che, se \((L_n)_{n \in \N}\) è una successione di operatori lineari limitati tra due spazi di Banach \(V \!\) e \(W \!\) tale che per ogni \(v \in V\) la successione \((L_n (v))_{n \in \N}\) è limitata, allora \(\sup_n \norm{L_n} < + \infty\). Nel nostro caso, \(V = \cont ([\, a, b \,])\) dotato della norma \(\norm{\cdot}_\infty\) e \(W = \R\) dotato della norma euclidea.} ciò implica che esiste finito \(\sup_n \norm{S_n}_\infty < + \infty\) e, quindi, esiste \(M \in \R\) tale che \(\sum_{i = 1}^{\eta_n} \abs{w_{i, n}} \le M < + \infty\). Il fatto che i monomi siano tutti funzioni continue permette di concludere.\qedhere
		\end{description}
	\end{proof}

	\begin{osservazione}
		La richiesta nelle ipotesi del Teorema~\ref{th:polya-steklov} che l'intervallo di integrazione \([\, a, b \,]\) sia limitato rende inapplicabile tale Teorema se si fa uso di funzioni peso come quella di Gauss-Laguerre o quella di Gauss-Hermite.
		
		Si osservi, poi, che il Teorema~\ref{th:polya-steklov} richiede soltanto che la funzione integranda sia continua (e non per forza derivabile un certo numero di volte, come in altri Teoremi visti in precedenza), però non offre alcuna stima esplicita dell'errore commesso al variare di \(n \in \N\).
	\end{osservazione}

	\begin{nota}
		La dimostrazione originale del teorema di Polya non ricorre al teorema di Banach-Steinhaus, ma mostra l'esistenza di una certa funzione \(f\) tale che \(\mathcal{E}_n (f)\) non converga a \(0\), se non esiste \(M \in \R\) tale che per ogni \(n \in \N\) si abbia \(\sum_{i = 1}^{\eta_n} \abs{w_{i, n}} \le M\).
	\end{nota}

	\begin{teorema}\label{th:quadratura-pesi-pos-converge}
		Data una successione di formule di quadratura \(S_n\) su un intervallo limitato \((\, a, b \,)\) che abbiano tutte pesi positivi, si ha \(\mathcal{E}_n (f) \to 0\) per ogni \(f \in \cont ([\, a, b \,])\) se e solo se \(\mathcal{E}_n (p) \to 0\) per ogni polinomio \(p\) di grado qualunque.
	\end{teorema}

	\begin{proof}
		Mostriamo entrambe le implicazioni.
		
		\begin{description}
			\item[(\(\Rightarrow\))] Se \(\mathcal{E}_n (f)\) è convergente per ogni funzione continua \(f\), lo è anche per ogni polinomio \(p\).
			\item[(\(\Leftarrow\))] In virtú del Teorema~\ref{th:polya-steklov}, è sufficiente mostrare che esiste \(M \in \R\) tale che \(\sum_{i = 1}^{\eta_n} \abs{w_{i, n}} \le M\). Poiché la successione di formule di quadratura converge sui polinomi, è convergente anche per il polinomio \(1\); da ciò segue che
			\begin{equation*}
				\sum_{i = 1}^{\eta_n} \abs{w_{i, n}} = \sum_{i = 1}^{\eta_n} w_{i, n} = S_n (1) \xrightarrow{n \to + \infty} I_w (1) = \int_a^b w (x) \dd{x}
			\end{equation*}
			e quindi esiste \(M\) indipendente da \(n\) tale che \(\sup_n \sum_{i = 1}^{\eta_n} \abs{w_{i, n}} \le M\).\qedhere
		\end{description}
	\end{proof}

	\begin{teorema}
		Data una successione di formule gaussiane \(S_n\) a \(n\) punti relative alla funzione peso \(w \colon (\, a, b \,) \to \R\) con \((\, a, b \,)\) limitato, si ha \(\mathcal{E}_n (f) \to 0\) per ogni \(f \in \cont ([\, a, b \,])\).
	\end{teorema}

	\begin{proof}
		In base al Teorema~\ref{th:formula-gaussiana-esiste-unica}, i pesi \(w_1, \dots, w_n\) della formula \(S_n\) sono positivi e, per quanto visto nel Teorema~\ref{th:quadratura-pesi-pos-converge}, si può giungere alla tesi se la successione di formule converge per ogni monomio \(x^k\). Sempre per il Teorema~\ref{th:formula-gaussiana-esiste-unica}, ciò è verificato, perché \(S_n\) ha grado di precisione almeno \(2 n - 1\), per cui
		\begin{equation*}
			n \ge \left\lceil \frac{k + 1}{2} \right\rceil \implies \mathcal{E}_n (x^k) = 0 % Probabilmente c'è una soluzione piú pulita per mettere i delimitatori del soffitto, ma non ho voglia di cercare.
		\end{equation*}
		e per il Teorema~\ref{th:polya-steklov} si può concludere.
	\end{proof}
	
	\chapter{Metodi iterativi per sistemi lineari}

	\noindent Dati una matrice \(A \in M_n (\C)\) e un vettore colonna \(b \in \C^n\), si vogliono trovare le soluzioni del sistema lineare \(A x = b\). D'ora in avanti supporremo che la soluzione \(x^*\) del sistema sia unica, ovvero che \(A\) sia non singolare.
	
	Questo problema si può risolvere mediante la fattorizzazione \textsc{lu} con \emph{pivoting}, che però ha costo computazionale pari a \(\order{n^3 / 3}\) moltiplicazioni; un tale costo è eccessivo se \(n\) è particolarmente elevato. La fattorizzazione \textsc{lu} decompone una matrice \(A\) invertibile secondo la regola \(P \! A = L U\), ove
		\begin{itemize}
			\item \(P\) è una matrice di permutazione;
			\item \(L\) è una matrice triangolare inferiore e tale che \(L_{i, i} = 1\) per ogni \(i \in \Set{1, \dots, n}\);
			\item \(U\) è una matrice triangolare superiore.
		\end{itemize}
	Osservando, poi, che
	\begin{equation*}
		A x = b \iff P \! A x = P b \iff L U x = P b
	\end{equation*}
	si ottiene la soluzione trovando la soluzione \(y^*\) del sistema \(L y = P b\) e poi determinando \(x^*\) tale che \(U x^* = y^*\).
	
	Un approccio alternativo è quello dei cosiddetti \emph{metodi iterativi}, coi quali si mira ad ottenere una successione di vettori \(x^{(k)}\) che converga alla soluzione \(x^*\) e tale che per \(\bar{k} \ll n\) si abbia \(x^{(\bar{k})} \approx x^*\). In generale, questi metodi non restituiscono la soluzione in modo esatto dopo un numero finito di operazioni, come invece accade per i metodi diretti: la soluzione proposta da tali metodi, infatti, è un limite -- che nella realtà sarà approssimato con poche iterazioni ciascuna di costo quadratico, di solito un prodotto matrice-vettore. Il costo totale di questi metodi sarà \(\order{\bar{k} n^2}\), che si vorrà minore del costo cubico della fattorizzazione \textsc{lu}.
	
\section{Metodi iterativi stazionari}
	
	\noindent Supponiamo che la matrice non singolare \(A \in M_n (\C)\) sia tale che \(A = M - N\), con \(M\) non singolare. In base a ciò, si vede che
	\begin{multline*}
		A x = b \iff M x - N x = b \iff M x = N x + b \\
		\iff x = M^{-1} N x + M^{-1} b
	\end{multline*}
	ovvero che la risoluzione del sistema lineare è equivalente alla risoluzione di un'equazione di punto fisso \(x = \varphi (x)\), con \(\varphi (x) = M^{-1} N x + M^{-1} b\). Risulta, quindi, naturale usare la successione del metodo di punto fisso
	\begin{equation}\label{eq:metodo-punto-fisso}
		x^{(k + 1)} = \varphi \qty(x^{(k)}) = M^{-1} N x^{(k)} + M^{-1} b
	\end{equation}
	La matrice \(P = M^{-1} N\) si dice \emph{di iterazione} e non dipende dall'indice di iterazione \(k\). Metodi di questo tipo si chiamano \emph{iterativi stazionari} proprio perché \(P\) non varia al variare di \(k\).
	
	Risulta utile scrivere la matrice \(A\) nella forma \(A = D - E - F\), ove
		\begin{itemize}
			\item \(D\) è una matrice diagonale;
			\item \(E\) è una matrice triangolare inferiore con elementi diagonali nulli;
			\item \(F\) è una matrice triangolare superiore con elementi diagonali nulli.
		\end{itemize}
	Questa scrittura è ben definita ed evidentemente unica, e serve per descrivere con maggiore agilità alcuni metodi iterativi stazionari noti e implementarli nel calcolatore.
	
	\begin{lstlisting}[style=Matlab-editor, float=tp, frame=lines, caption={Implementazione della decomposizione \(A = D - E - F\).}, label=cod:decomp-def]
function [D, E, F] = decomp_met_iter(A)

D =   diag(diag(A));
E = - (tril(A) - D); 
F = - (triu(A) - D);
	\end{lstlisting}

	\begin{definizione}[Metodo di Jacobi]\label{def:metodo-jacobi}
		Data una matrice \(A \in M_n (\C)\), consideratane la decomposizione \(A = D - E - F\) come sopra e fissato \(x^{(0)} \in \C^n\), si dice \emph{metodo di Jacobi} il metodo iterativo stazionario con \(M = D\) e \(N = E + F\).
	\end{definizione}

	La matrice di iterazione del metodo di Jacobi è
	\begin{equation*}
		P = M^{-1} N = D^{-1} (E + F) = D^{-1} (D - D + E + F) = \uno_n - D^{-1} A
	\end{equation*}

	\begin{osservazione}
		Se \(D\) è non singolare, allora il metodo di Jacobi può essere descritto come \emph{metodo delle sostituzioni simultanee}
		\begin{equation}\label{eq:metodo-sostituzioni-simultanee}
			x_i^{(k + 1)} = \frac{1}{a_{i, i}} \qty(b_i - \sum_{j = 1}^{i - 1} a_{i, j} x_j^{(k)} - \sum_{j = i + 1}^n a_{i, j} x_j^{(k)}) \qquad \forall i \in \Set{1, \dots, n}
		\end{equation}
		in quanto \(a_{i, i} \ne 0\) per ogni \(i \in \Set{1, \dots, n}\). La \eqref{eq:metodo-sostituzioni-simultanee} deriva dal fatto che, se \(A x = b\), allora per ogni \(i \in \Set{1, \dots, n}\)
		\begin{equation*}
			b_i = \sum_{j = 1}^n a_{i, j} x_j = \qty(\sum_{j = 1}^{i - 1} a_{i, j} x_j) + a_{i, i} x_i + \qty(\sum_{j = i + 1}^n a_{i, j} x_j)
		\end{equation*}
		e, supposto \(a_{i, i} \ne 0\), si ricava
		\begin{equation*}
			x_i = \frac{1}{a_{i, i}} \qty(b_i - \sum_{\substack{j = 1 \\ j \ne i}}^n a_{i, j} x_j) \qquad \forall i \in \Set{1, \dots, n}
		\end{equation*}
		da cui risulta naturale derivare il metodo di Jacobi.
	\end{osservazione}

	\begin{esempio}
		Applicando la \eqref{eq:metodo-sostituzioni-simultanee} per risolvere un problema \(A x = b\) con \(A \in M_3 (\C)\), si ottiene
		\begin{align*}
			x_1^{(k + 1)} &= \frac{1}{a_{1, 1}} \qty(b_1 - a_{1, 2} x_2^{(k)} - a_{1, 3} x_3^{(k)}) \\
			x_2^{(k + 1)} &= \frac{1}{a_{2, 2}} \qty(b_2 - a_{2, 1} x_1^{(k)} - a_{2, 3} x_3^{(k)}) \\
			x_3^{(k + 1)} &= \frac{1}{a_{3, 3}} \qty(b_3 - a_{3, 1} x_1^{(k)} - a_{3, 2} x_2^{(k)})
		\end{align*}
	\end{esempio}

	\begin{definizione}[Metodo di Gauss-Seidel]\label{def:metodo-gauss-seidel}
		Data una matrice \(A \in M_n (\C)\), consideratane la decomposizione \(A = D - E - F\) come sopra e fissato \(x^{(0)} \in \C^n\), si dice \emph{metodo di Gauss-Seidel} il metodo iterativo stazionario con \(M = D - E\) e \(N = F\).
	\end{definizione}

	La matrice di iterazione del metodo di Gauss-Seidel è
	\begin{equation*}
		P = M^{-1} N = (D - E)^{-1} F
	\end{equation*}

	\begin{osservazione}
		Se \(D\) è non singolare, allora il metodo di Gauss-Seidel può essere descritto come \emph{metodo delle sostituzioni successive}
		\begin{equation}\label{eq:metodo-sostituzioni-successive}
			x_i^{(k + 1)} = \frac{1}{a_{i, i}} \qty(b_i - \sum_{j = 1}^{i - 1} a_{i, j} x_j^{(k + 1)} - \sum_{j = i + 1}^n a_{i, j} x_j^{(k)}) \quad \forall i \in \Set{1, \dots, n}
		\end{equation}
		in quanto \(a_{i, i} \ne 0\) per ogni \(i \in \Set{1, \dots, n}\). La \eqref{eq:metodo-sostituzioni-successive} deriva dal fatto che, se \(A x = b\), allora per ogni \(i \in \Set{1, \dots, n}\)
		\begin{equation*}
			b_i = \sum_{j = 1}^n a_{i, j} x_j = \qty(\sum_{j = 1}^{i - 1} a_{i, j} x_j) + a_{i, i} x_i + \qty(\sum_{j = i + 1}^n a_{i, j} x_j)
		\end{equation*}
		e, supposto \(a_{i, i} \ne 0\), si ricava
		\begin{equation*}
		x_i = \frac{1}{a_{i, i}} \qty(b_i - \sum_{\substack{j = 1 \\ j \ne i}}^n a_{i, j} x_j) \qquad \forall i \in \Set{1, \dots, n}
		\end{equation*}
		Considerate le ultime \(n - i\) coordinate della soluzione approssimata \(k\)-esima \(x_{i + 1}^{(k)}, \dots, x_n^{(k)}\) e calcolate le prime \(i - 1\) coordinate dell'approssimazione successiva \(x_1^{(k + 1)}, \dots, x_{i - 1}^{(k + 1)}\), è naturale derivare il metodo di Gauss-Seidel.
	\end{osservazione}

	\begin{esempio}
		Applicando la \eqref{eq:metodo-sostituzioni-successive} per risolvere un problema \(A x = b\) con \(A \in M_3 (\C)\), si ottiene
		\begin{align*}
			x_1^{(k + 1)} &= \frac{1}{a_{1, 1}} \qty(b_1 - a_{1, 2} x_2^{(k)} - a_{1, 3} x_3^{(k)}) \\
			x_2^{(k + 1)} &= \frac{1}{a_{2, 2}} \qty(b_2 - a_{2, 1} x_1^{(k + 1)} - a_{2, 3} x_3^{(k)}) \\
			x_3^{(k + 1)} &= \frac{1}{a_{3, 3}} \qty(b_3 - a_{3, 1} x_1^{(k + 1)} - a_{3, 2} x_2^{(k + 1)})
		\end{align*}
	\end{esempio}

	\begin{definizione}[Metodo \textsc{sor}]\label{def:metodo-sor}
		Data una matrice \(A \in M_n (\C)\), consideratane la decomposizione \(A = D - E - F\) come sopra e fissati \(x^{(0)} \in \C^n\) e \(\omega \in \C^*\), si dice \emph{metodo di \inglese{successive over-relaxation}} il metodo iterativo stazionario con \(M = \omega^{-1} D - E\) e \(N = (\omega^{-1} - 1) D + F\).
	\end{definizione}

	\begin{osservazione}
		I metodi \textsc{sor} si ricavano dallo scrivere esplicitamente un'iterazione del metodo di Gauss-Seidel
		\begin{equation*}
			z_i^{(k + 1)} = \frac{1}{a_{i, i}} \qty(b_i - \sum_{j = 1}^{i - 1} a_{i, j} x_j^{(k + 1)} - \sum_{j = i + 1}^n a_{i, j} x_j^{(k)})
		\end{equation*}
		e dall'effettuare la sostituzione
		\begin{equation*}
			x_i^{(k + 1)} = \omega z_i^{(k + 1)} + (1 - \omega) x_i^{(k)}
		\end{equation*}
		sicché si ottiene
		\begin{equation}\label{eq:metodo-sor}
			x_i^{(k + 1)} = \frac{\omega}{a_{i, i}} \qty(b_i - \sum_{j = 1}^{i - 1} a_{i, j} x_j^{(k + 1)} - \sum_{j = i + 1}^n a_{i, j} x_j^{(k)}) - (1 - \omega) x_i^{(k)}
		\end{equation}
		per ogni \(i \in \Set{1, \dots, n}\). L'equivalenza della \eqref{eq:metodo-sor} al metodo \textsc{sor} si vede osservando che il sistema di tutte le equazioni del tipo \eqref{eq:metodo-sor} è equivalente all'equazione matriciale
		\begin{equation*}
			x^{(k + 1)} = \omega D^{-1} \qty(b + E x^{(k + 1)} + F x^{(k)}) + (1 - \omega) x^{(k)}
		\end{equation*}
		da cui segue che
		\begin{equation*}
			\underbrace{\omega^{-1} D (\uno_n - \omega D^{-1} E)}_{M} x^{(k + 1)} = \underbrace{\omega^{-1} D \qty[\omega D^{-1} F + (1 - \omega) \uno_n]}_{N} x^{(k)} + b
		\end{equation*}
		che si riconduce facilmente a un metodo di forma come nella \eqref{eq:metodo-punto-fisso}.
	\end{osservazione}

	\begin{nota}
		Il metodo di Gauss-Seidel è un metodo \textsc{sor} con \(\omega = 1\).
	\end{nota}

\section{Metodi di Richardson}
	
	\begin{definizione}[Residuo]\label{def:residuo}
		Dati \(A \in M_n (\C)\), \(x \in \C^n\) e \(b \in \C^n\), si dice \emph{residuo} relativo ad \(A\) e \(b\) la quantità
		\begin{equation}\label{eq:residuo}
			r = A x - b
		\end{equation}
	\end{definizione}

	\begin{definizione}[Metodo di Richardson stazionario]\label{def:metodo-richardson-stazionario}
		Fissato \(\alpha \in \C\), un metodo iterativo si dice \emph{di Richardson stazionario} con matrice di precondizionamento \(P\) se
		\begin{equation}\label{eq:metodo-richardson-stazionario}
			P \qty(x^{(k + 1)} - x^{(k)}) = \alpha r^{(k)}
		\end{equation}
		ove \(r^{(k)}\) è il residuo della \(k\)-esima iterazione del metodo.
	\end{definizione}
	
	\begin{definizione}[Metodo di Richardson non stazionario]\label{def:metodo-richardson-non-stazionario}
		Data una successione \(\Set{\alpha_k | k \in \N} \subseteq \C\), un metodo iterativo si dice \emph{di Richardson non stazionario} con matrice di precondizionamento \(P\) se
		\begin{equation}\label{eq:metodo-richardson-non-stazionario}
			P \qty(x^{(k + 1)} - x^{(k)}) = \alpha_k r^{(k)}
		\end{equation}
		ove \(r^{(k)}\) è il residuo della \(k\)-esima iterazione del metodo.
	\end{definizione}

	\begin{osservazione}
		Se \(\alpha_k = \alpha\) per ogni \(k \in \N\), allora un metodo di Richardson non stazionario è in effetti stazionario.
		
		La matrice \(P\), poi, è di solito scelta di modo che la soluzione della \eqref{eq:metodo-richardson-non-stazionario} non richieda un grande sforzo computazionale; ad esempio, si cerca \(P\) diagonale o triangolare.
	\end{osservazione}

	\begin{proposizione}\label{prop:punto-fisso-richardson}
		I metodi di tipo come nella \eqref{eq:metodo-punto-fisso} sono di Richardson stazionari con \(P = M\) e \(\alpha = 1\).
	\end{proposizione}
	
	\begin{proof}
		Per ipotesi è verificata la \eqref{eq:metodo-punto-fisso}, ossia \(M x^{(k + 1)} = N x^{(k)} + b\). Si ha, dunque,
		\begin{equation*}
			M \qty(x^{(k + 1)} - x^{(k)}) = N x^{(k)} + b - M x^{(k)} = b - A x^{(k)} = r^{(k)} \qedhere
		\end{equation*}
	\end{proof}

	In base alla Proposizione~\ref{prop:punto-fisso-richardson}, i metodi di Jacobi, Gauss-Seidel e \textsc{sor} sono metodi di Richardson stazionari con \(P = M\) e \(\alpha = 1\).
	
	Di solito si cerca un parametro \(\alpha\) che renda la convergenza dei metodi di Richardson stazionari molto veloce. Nel caso di una matrice simmetrica, ad esempio, tale parametro è \(2 / (\lambda_\textup{min} + \lambda_\textup{max})\).
	
\section{Richiami di teoria delle matrici}
	
	\begin{definizione}[Raggio spettrale]\label{def:raggio-spettrale}
		Di una matrice \(A \in M_n (\C)\) con autovalori \(\lambda_1, \dots, \lambda_n\) si dice \emph{raggio spettrale} la quantità
		\begin{equation}
			\raggio (A) = \max \Set{\abs{\lambda_i} : i \in \Set{1, \dots, n}}
		\end{equation}
	\end{definizione}

	\begin{definizione}[Norma naturale]\label{def:norma-indotta}
		Data una norma vettoriale \(\norm{\cdot} \colon \C^n \to \R_{\ge 0}\), si dice \emph{norma naturale} o \emph{indotta} la norma matriciale
		\begin{equation}
			\norm{A} \coloneqq \sup_{\substack{x \in \C^n \\ x \ne 0}} \frac{\norm{A x}}{\norm{x}}
		\end{equation}
	\end{definizione}
	
	Si può dimostrare che, se \(\norm{\cdot}\) è una norma indotta, allora \(\norm{A x} \le \norm{A} \, \norm{x}\) per ogni \(A \in M_n (\C)\) e per ogni \(x \in \C^n\).
	
	La Definizione~\ref{def:raggio-spettrale} coincide con quella di una norma di un operatore lineare e continuo tra spazi normati.
	
	\begin{esempio}
		Dati \(x \in \C^n\) e \(A \in M_n (\C)\), si possono definire le seguenti norme
		\begin{subequations}
		\begin{align}
			\norm{x}_1 &= \sum_{k = 1}^n \abs{x_k} & \norm{A}_1 &= \max_{j \in \Set{1, \dots, n}} \sum_{i = 1}^n \abs{a_{i, j}} \\
			\norm{x}_2 &= \sqrt{\sum_{k = 1}^n \abs{x_k}^2} & \norm{A}_2 &= \sqrt{\raggio \qty(\her{A} A)} \\
			\norm{x}_\infty &= \max_{k \in \Set{1, \dots, n}} \abs{x_k} & \norm{A}_\infty &= \max_{i \in \Set{1, \dots, n}} \sum_{j = 1}^n \abs{a_{i, j}}
		\end{align}
		\end{subequations}
	
		La norma matriciale \(1\) si calcola trovando la massima somma per colonna; la norma matriciale \(\infty\), invece, si calcola trovando la massima somma per riga.
	\end{esempio}

	\begin{teorema}
		Per ogni norma naturale \(\norm{\cdot}\) e per ogni matrice \(A \in M_n (\C)\) si ha \(\raggio (A) \le \norm{A}\). Per ogni matrice \(A \in M_n (\C)\) e per ogni \(\varepsilon > 0\), inoltre, esiste una norma naturale \(\norm{\cdot}\) tale che \(\raggio (A) \le \norm{A} \le \raggio (A) + \varepsilon\).
	\end{teorema}

	\begin{teorema}\label{th:matrice-infinitesima}
		Fissata una norma naturale \(\norm{\cdot}\), per ogni \(A \in M_n (\C)\) le seguenti affermazioni sono equivalenti:
		\begin{itemize}
			\item esiste una norma naturale \(\norm{\cdot}_*\) tale che \(\lim_{m \to \infty} \norm{A^m}_* = 0\);
			\item \(\lim_{m \to \infty} \norm{A^m} = 0\);
			\item \(\raggio (A) < 1\).
		\end{itemize}
	\end{teorema}

	Si osservi che nel Teorema~\ref{th:matrice-infinitesima} non è richiesto che \(\norm{A} < 1\).
	
	\begin{osservazione}
		Il raggio spettrale non può essere usato come norma matriciale. Si consideri, infatti, la matrice \(A = \begin{psmallmatrix} 0 & 1 \\ 0 & 0 \end{psmallmatrix}\); benché \(\raggio (A) = 0\), essa non è la matrice nulla.
		
		Sebbene in generale il raggio spettrale differisca dalle norme \(1\), \(2\) e \(\infty\), esistono casi particolari in cui \(\raggio (A) = \norm{A}_2\), come ad esempio quando \(A\) è una matrice diagonale -- dato che i suoi autovalori coincidono con i suoi elementi diagonali.
	\end{osservazione}

	\begin{definizione}
		Una matrice \(A \in M_n (\C)\) si dice \emph{diagonalizzabile} se \(\C^n\) ammette una base di autovettori per \(A\) o, equivalentemente, se esistono una matrice \(D\) diagonale e una matrice \(S\) invertibile tali che \(A = S^{-1} D S\).
	\end{definizione}

\section{Convergenza dei metodi iterativi stazionari}
	
	\begin{definizione}[Metodo consistente]\label{def:metodo-consistente}
		Dati una matrice \(A \in M_n (\C)\) non singolare e \(x^*, b \in \C^n\) tali che \(A x^* = b\), un metodo stazionario \(x^{(k + 1)} = P x^{(k)} + c\) si dice \emph{consistente} rispetto al problema \(A x = b\) se verifica
		\begin{equation}
			x^* = P x^* + c
		\end{equation}
	\end{definizione}

	\begin{lemma}\label{lem:errore-metodo-iter-staz}
		Se un metodo iterativo stazionario \(x^{(k + 1)} = P x^{(k)} + c\) è consistente rispetto al problema \(A x = b\), allora, posto \(e^{(k)} = x^{(k)} - x^*\), esso verifica per ogni \(m \in \N\)
		\begin{equation}\label{eq:errore-metodo-iter-staz}
			e^{(m)} = P^m e^{(0)}
		\end{equation}
	\end{lemma}

	\begin{proof}
		Osservato che, per ogni \(m \in \N^*\),
		\begin{equation*}
			\begin{split}
				e^{(m)} &= x^{(m)} - x^* = \qty(P x^{(m - 1)} + c) - \qty(P x^* + c) \\
				&= P \qty(x^{(m - 1)} - x^*) = P e^{(m - 1)}
			\end{split}
		\end{equation*}
		la tesi segue procedendo iterativamente.
	\end{proof}

	\begin{teorema}\label{th:conv-raggio-spettr}
		Un metodo iterativo stazionario consistente \(x^{(k + 1)} = P x^{(k)} + c\), con \(P \in M_n (\C)\), converge per ogni vettore iniziale \(x_0 \in \C^n\) se e solo se \(\raggio (P) < 1\).
	\end{teorema}

	\begin{proof}
		Mostriamo entrambe le implicazioni.
		
		\begin{description}
			\item[(\(\Leftarrow\))] In virtú del Teorema~\ref{th:matrice-infinitesima}, scelta una qualsiasi norma naturale \(\norm{\cdot}\), essa verifica \(\norm{P^m} \xrightarrow{m \to \infty} 0\). Per la \eqref{eq:errore-metodo-iter-staz} e per le proprietà della norma naturale si ha
			\begin{equation*}
				\bigl\| e^{(k)} \bigr\| = \bigl\| P^k e^{(0)} \bigr\| \le \bigl\| P^k \bigr\| \, \bigl\| e^{(0)} \bigr\|
			\end{equation*}
			e, in base a quanto visto sopra a proposito di \(P\), si ottiene \(\bigl\| e^{(k)} \bigr\| \to 0\), ovvero \(x^{(k)} \to x^*\).
			\item[(\(\Rightarrow\))] Per assurdo si abbia \(x^{(k)} \to x^*\) e \(\raggio (P) \ge 1\). Chiamato \(\lambda \in \C\) l'autovalore per \(P\) di massimo modulo, scegliamo \(x^{(0)}\) tale che \(e^{(0)}\) sia autovettore relativo a \(\lambda\); si ha
			\begin{equation*}
				e^{(k)} = P^k e^{(0)} = P^{k - 1} \qty(P e^{(0)}) = P^{k - 1} \qty(\lambda e^{(0)}) = \lambda P^{k - 1} e^{(0)} = \lambda e^{(k - 1)}
			\end{equation*}
			da cui segue che, per ogni \(k \in \N\), si ha \(e^{(k)} = \lambda^k e^{(0)}\). Scelta una qualunque norma naturale \(\norm{\cdot}\), quindi, si ha per ogni \(k \in \N\)
			\begin{equation*}
				\bigl\| e^{(k)} \bigr\| = \bigl| \lambda^k \bigr| \, \bigl\| e^{(0)} \bigr\| \ge \bigl\| e^{(0)} \bigr\|
			\end{equation*}
			il che è assurdo, perché dovrebbe esistere almeno un \(k \in \N\) che verifichi \(\norm*{e^{(k)}} < \norm*{e^{(0)}}\). Da ciò si conclude che \(\raggio (P) < 1\), come si voleva. \qedhere
		\end{description}
	\end{proof}

	\begin{osservazione}
		Il Teorema~\ref{th:conv-raggio-spettr} è valido per ogni vettore iniziale scelto, quindi è di fatto un teorema di convergenza globale. \emph{A priori}, però, potrebbe esistere \(\tilde{x} \ne x^*\) tale che \(\tilde{x} = P \tilde{x} + c\) e che, quindi, sarebbe un altro limite verso cui il metodo tende. Posto, però, \(x^{(0)} = \tilde{x}\), si vede che \(x^{(k)} = \tilde{x}\) per ogni \(k \in \N\) e, per il Teorema~\ref{th:conv-raggio-spettr}, \(\tilde{x} \to x^*\), da cui segue che \(\tilde{x} = x^*\).
		
		La matrice \(P\), poi, non dev'essere per forza diagonalizzabile. Se, però, si verifica \(P \in M_n (\R)\), è comunque possibile che \(\lambda \notin \R\), in quanto il polinomio caratteristico di \(P\) può non avere tutte le radici reali. Il metodo, in tal caso, può non funzionare se si esclude \emph{a priori} l'utilizzo di vettori \(x^{(k)}\) a coordinate non reali.
	\end{osservazione}

	Vogliamo ora cercare e stimare un legame tra il raggio spettrale della matrice di iterazione \(P\) e la riduzione dell'errore. Supponiamo di voler approssimare quell'\(x^*\) tale che \(A x^* = b\) attraverso il metodo iterativo stazionario \(x^{(k + 1)} = P x^{(k)} + c\), che per ipotesi ammettiamo consistente. È possibile mostrare che, scelta una norma naturale \(\norm{\cdot}\), ogni matrice \(B \in M_n (\C)\) verifica \(\norm{B^m}^{1 / m} \xrightarrow{m \to \infty} \raggio (B)\); da ciò segue che, scelto \(m \in \N\) abbastanza grande, la matrice \(P\) verifica \(\norm{P^m}^{1 / m} \approx \raggio (P)\), ossia \(\norm{P^m} \approx \raggio^m (P)\). In base a questa considerazione e alla \eqref{eq:errore-metodo-iter-staz}, si ha
	\begin{equation*}
		\bigl\| e^{(m)} \bigr\| = \bigl\| P^m e^{(0)} \bigr\| \le \norm{P^m} \, \bigl\| e^{(0)} \bigr\| \approx \raggio^m (P) \bigl\| e^{(0)} \bigr\|
	\end{equation*}
	e, quindi,
	\begin{equation}\label{eq:stima-circa-quasi}
		\frac{\bigl\| e^{(m)} \bigr\|}{\bigl\| e^{(0)} \bigr\|} \lessapprox \raggio^m (P)
	\end{equation}
	
	Per questo motivo, supposto \(\raggio (P) < 1\) e chiamato \(m^* \in \R\) il numero che verifica \(\raggio^{m^*} (P) = \varepsilon\), si trova che necessariamente il numero intero
	\begin{equation*}
		m = \left\lceil \frac{\log \varepsilon}{\log \raggio (P)} \right\rceil
	\end{equation*}
	verifica \(\norm*{e^{(m)}} / \norm*{e^{(0)}} \lessapprox \varepsilon\), in quanto per la \eqref{eq:stima-circa-quasi} si ha
	\begin{equation*}
		\frac{\bigl\| e^{(m)} \bigr\|}{\bigl\| e^{(0)} \bigr\|} \lessapprox \raggio^m (P) \le \raggio^{m^*} (P) = \varepsilon
	\end{equation*}
	
	Denotata con \(R (P) = - \log \raggio (P)\) la \emph{velocità di convergenza asintotica} del metodo iterativo stazionario relativo a \(P\), si osserva che
	\begin{equation*}
		m = \left\lceil - \frac{\log \varepsilon}{R (P)} \right\rceil
	\end{equation*}
	In virtú di ciò, se \(0 < \raggio (P) \ll 1\), allora \(\log \raggio (P) \ll 0\) e, quindi, \(R (P) \gg 0\), ovvero il numero \(m\) di iterazioni da compiere perché l'errore sia all'incirca \(\varepsilon\) è molto basso -- e il metodo si dimostra molto veloce. Ciò suggerisce di ricercare metodi iterativi stazionari che abbiano \(\raggio (P)\) quanto piú prossimo a \(0\) possibile.
	
	\begin{definizione}
		Una matrice \(A \in M_n (\C)\) si dice \emph{tridiagonale} se \(a_{i, j} = 0\) ogniqualvolta \(\abs{i - j} > 1\).
	\end{definizione}

	\begin{teorema}\label{th:conv-tridiag-jac-gs}
		Data una matrice \(A \in M_n (\C)\) tridiagonale e priva di elementi diagonali nulli, il metodo di Jacobi risulta convergente nell'approssimare la soluzione al sistema lineare \(A x = b\) se e solo se lo è il metodo di Gauss-Seidel.
		
		Nel caso in cui tali metodi convergano, la velocità di convergenza del metodo di Gauss-Seidel è doppia rispetto a quella del metodo di Jacobi.
	\end{teorema}

	La seconda parte del Teorema~\ref{th:conv-tridiag-jac-gs} si interpreta come segue: il metodo di Gauss-Seidel richiede asintoticamente la metà delle iterazioni necessarie al metodo di Jacobi perché fornisca un'approssimazione con un errore sotto una certa soglia fissata.
	
	\begin{definizione}
		Una matrice \(A \in M_n (\C)\) si dice \emph{a predominanza diagonale} per righe se verifica per ogni \(i \in \Set{1, \dots, n}\)
		\begin{equation*}
			\abs{a_{i, i}} \ge \sum_{\substack{j = 1 \\ j \ne i}}^n \abs{a_{i, j}}
		\end{equation*}
		ed esiste \(s \in \Set{1, \dots, n}\) per cui tale disuguaglianza sia stretta.
		
		La matrice \(A\) si dice a predominanza diagonale per colonne se \(\her{A}\) è a predominanza diagonale per righe.
	\end{definizione}

	\begin{definizione}
		Una matrice \(A \in M_n (\C)\) si dice \emph{a predominanza diagonale} per righe in senso stretto se verifica per ogni \(i \in \Set{1, \dots, n}\)
		\begin{equation*}
			\abs{a_{i, i}} > \sum_{\substack{j = 1 \\ j \ne i}}^n \abs{a_{i, j}}
		\end{equation*}
	
		La matrice \(A\) si dice a predominanza diagonale per colonne in senso stretto se \(\her{A}\) è a predominanza diagonale per righe in senso stretto.
	\end{definizione}

	\begin{esempio}
		La matrice
		\begin{equation*}
			A =
			\begin{pmatrix}
				4  & -4 & 0  \\
				-1 & 4  & -1 \\
				0  & -4 & 4
			\end{pmatrix}
		\end{equation*}
		è a predominanza diagonale per righe non stretta.
	\end{esempio}

	\begin{osservazione}
		Si può dimostrare il primo teorema di Gerschgorin, che afferma che ogni matrice a predominanza diagonale in senso stretto è non singolare.
	\end{osservazione}
	
	I seguenti Teoremi si basano sul teorema di convergenza dei metodi stazionari.
	
	\begin{teorema}
		Se \(A \in M_n (\C)\) è a predominanza diagonale per righe in senso stretto, allora il metodo di Jacobi converge alla soluzione del problema \(A x = b\) per qualunque vettore iniziale \(x^{(0)}\).
	\end{teorema}

	\begin{teorema}
		Se \(A \in M_n (\C)\) è a predominanza diagonale per righe in senso stretto, allora il metodo di Gauss-Seidel converge alla soluzione del problema \(A x = b\) per qualunque vettore iniziale \(x^{(0)}\).
	\end{teorema}

	\begin{definizione}
		Una matrice \(A \in M_n (\C)\) si dice \emph{hermitiana} se \(A = \her{A}\).
		
		Una matrice \(A \in M_n (\R)\) si dice \emph{simmetrica} se \(A = \tra{A}\).
	\end{definizione}

	\begin{definizione}
		Una matrice hermitiana \(A \in M_n (\C)\) si dice \emph{definita positiva} se ha tutti gli autovalori positivi o, equivalentemente, se per ogni \(x \in \C^n \setminus \Set{0}\) si ha \(\her{x} A x > 0\).
	\end{definizione}

	\begin{teorema}[Kahan]\label{th:kahan}
		Data una matrice quadrata \(A \in M_n (\C)\), se un metodo \textsc{sor} di parametro \(\omega\) converge alla soluzione del problema \(A x = b\), allora \(\abs{1 - \omega} < 1\). In particolare, se \(\omega \in \R\), allora \(\omega \in (\, 0, 2 \,)\).
	\end{teorema}

	\begin{teorema}[Ostrowski-Reich]\label{th:ostrowski-reich}
		Se \(A \in M_n (\C)\) è una matrice definita positiva e \(\omega \in (\, 0, 2 \,)\), allora il metodo \textsc{sor} di parametro \(\omega\) converge alla soluzione del problema \(A x = b\).
	\end{teorema}

\section{Test di arresto}
	
	\noindent Consideriamo il sistema lineare \(A x = b\) avente come unica soluzione \(x^*\) e supponiamo di risolverlo con un metodo iterativo stazionario del tipo \(x^{(k + 1)} = P x^{(k)} + c\) e che supponiamo consistente. Intendiamo introdurre un \emph{test di arresto}, ovvero un criterio in base al quale interrompere le iterazioni allorquando una certa quantità, dipendente dal problema in esame e dal numero di iterazioni, sia inferiore a una soglia fissata \(\varepsilon > 0\).
	
	\begin{proposizione}[Stima a posteriori]\label{prop:stima-posteriori-crit-step}
		Dati un sistema lineare \(A x = b\) con \(A\) non singolare e un metodo iterativo stazionario consistente \(x^{(k + 1)} = P x^{(k)} + c\), se, scelta una norma naturale \(\norm{\cdot}\), si verifica \(\norm{P} < 1\), allora
		\begin{equation}\label{eq:stima-posteriori-crit-step}
			\bigl\| x^* - x^{(k)} \bigr\| \le \frac{1}{1 - \norm{P}} \bigl\| x^{(k + 1)} - x^{(k)} \bigr\|
		\end{equation}
	\end{proposizione}

	\begin{proof}
		Definiamo \(\varDelta^{(k)} = x^{(k + 1)} - x^{(k)}\) e \(e^{(k)} = x^* - x^{(k)}\). In base alla \eqref{eq:errore-metodo-iter-staz}, si ha
		\begin{multline*}
			\bigl\| e^{(k)} \bigr\| = \bigl\| x^* - x^{(k + 1)} + x^{(k)} - x^{(k)} \bigr\| = \bigl\| e^{(k + 1)} - \varDelta^{(k)} \bigr\| \\
			= \bigl\| P e^{(k)} - \varDelta^{(k)} \bigr\| \le \norm{P} \, \bigl\| e^{(k)} \bigr\| + \bigl\| \varDelta^{(k)}\bigr\|
		\end{multline*}
		ovvero \((1 - \norm{P}) \norm*{e^{(k)}} \! \le \norm*{\varDelta^{(k)}}\), da cui segue la tesi.
	\end{proof}

	La Proposizione~\ref{prop:stima-posteriori-crit-step} ci spinge a dare la seguente definizione.
	
	\begin{definizione}[Criterio dello step]
		Fissato \(\varepsilon > 0\), il \emph{criterio dello step} consiste nell'interrompere un metodo iterativo che generi la successione \((x^{(k)})_{k \in \N}\) alla \(k\)-esima iterazione se \(\norm*{x^{(k + 1)} - x^{(k)}} \! \le \varepsilon\).
	\end{definizione}

	Dalla \eqref{eq:stima-posteriori-crit-step} segue che, qualora scatti il criterio dello step, se \(\norm{P} \ll 1\), allora la soluzione è probabilmente approssimata a meno della tolleranza specificata.
	
	Analizziamo ora il caso particolare in cui si è scelta la norma \(\norm{\cdot}_2\) e \(P\) è una matrice simmetrica.
	
	\begin{teorema}
		Supposto che \(A x^* = b\) con \(A\) matrice non singolare, se \(x^{(k + 1)} = P x^{(k)} + c\) è un metodo iterativo stazionario consistente e convergente e \(P\) è simmetrica, allora
		\begin{equation}\label{eq:stima-posteriori-euclidea-simmetrica}
			\bigl\| x^* - x^{(k)} \bigr\|_2 \le \frac{\bigl\| x^{(k + 1)} - x^{(k)} \bigr\|_2}{1 - \raggio (P)}
		\end{equation}
	\end{teorema}

	\begin{osservazione}
		Poiché \(\raggio (P) \le \norm{P}\) per qualunque norma naturale \(\norm{\cdot}\), la \eqref{eq:stima-posteriori-crit-step} non implica in generale la \eqref{eq:stima-posteriori-euclidea-simmetrica}, perché
		\begin{equation*}
			\frac{1}{1 - \raggio (P)} \le \frac{1}{1 - \norm{P}_2}
		\end{equation*}
		Se, però, \(P\) è simmetrica, allora \(\raggio (P) = \norm{P}_2\) e, quindi, la \eqref{eq:stima-posteriori-crit-step} è di fatto equivalente alla \eqref{eq:stima-posteriori-euclidea-simmetrica}.
	\end{osservazione}

	\begin{lemma}
		Se \(P\) è simmetrica, allora esistono una matrice ortogonale \(U \!\) e una matrice diagonale reale \(\varLambda\) tali che \(P = U \! \varLambda \tra{U}\) e gli autovalori di \(P\) siano quelli di \(\varLambda\), con le stesse molteplicità.
	\end{lemma}

	La dimostrazione di questo Lemma è omessa, in quanto facoltativa.
	
	\begin{definizione}[Criterio del residuo]
		Fissato \(\varepsilon > 0\), il \emph{criterio del residuo} consiste nell'interrompere un metodo iterativo che generi la successione \((x^{(k)})_{k \in \N}\) alla \(k\)-esima iterazione se \(\norm*{r^{(k)}} \le \varepsilon\), ove \(r^{(k)}\) è il residuo definito nella \eqref{eq:residuo}.
	\end{definizione}

	\begin{definizione}[Criterio del residuo relativo]
		Dovendo risolvere un sistema lineare \(A x = b\) con \(b \ne \vec{0}\), fissato \(\varepsilon > 0\), il \emph{criterio del residuo relativo} consiste nell'interrompere un metodo iterativo che generi la successione \((x^{(k)})_{k \in \N}\) alla \(k\)-esima iterazione se \(\norm*{r^{(k)}} \! / \norm{b} \le \varepsilon\).
	\end{definizione}
	
	\begin{teorema}\label{th:stima-criterio-residuo-relativo}
		Supposto che \(A x^* = b\) con \(A\) matrice non singolare e \(b \ne \vec{0}\) e fissata una norma naturale \(\norm{\cdot}\), se \((x^{(k)})_{k \in \N}\) è una successione generata da un metodo iterativo, allora
		\begin{equation}\label{eq:stima-criterio-residuo-relativo}
			\frac{\bigl\| x^{(k)} - x^* \bigr\|}{\norm{x^*}} \le \kappa (A) \, \frac{\bigl\| r^{(k)} \bigr\|}{\norm{b}}
		\end{equation}
		ove \(\kappa (A)\) indica l'indice di condizionamento della matrice \(A\).
	\end{teorema}

	\begin{proof}
		Poiché \(A\) è invertibile, si ha
		\begin{equation*}
			x^* - x^{(k)} = A^{-1} A \qty(x^* - x^{(k)}) = A^{-1} \qty(b - A x^{(k)}) = A^{-1} r^{(k)}
		\end{equation*}
		e, dato che \(\norm{\cdot}\) è una norma naturale,
		\begin{equation*}
			\bigl\| x^* - x^{(k)} \bigr\| = \bigl\| A^{-1} r^{(k)} \bigr\| \le \norm{A^{-1}} \, \bigl\| r^{(k)} \bigr\|
		\end{equation*}
		e
		\begin{equation*}
			\norm{b} = \norm{A x^*} \le \norm{A} \, \norm{x^*} \implies \frac{1}{\norm{x^*}} \le \frac{\norm{A}}{\norm{b}}
		\end{equation*}
		Ricordando che \(\kappa (A) = \norm{A} \, \norm{A^{-1}}\) e che \(x^* \ne \vec{0}\),\footnote{Non è possibile che \(x = \vec{0}\) perché \(\ker A = \Braket{\vec{0}}\) e \(b \ne \vec{0}\).} si ricava
		\begin{equation*}
			\frac{\bigl\| x^* - x^{(k)} \bigr\|}{\norm{x^*}} \le \frac{\norm{A}}{\norm{b}} \norm{A^{-1}} \, \bigl\| r^{(k)} \bigr\| = \kappa (A) \, \frac{\bigl\| r^{(k)} \bigr\|}{\norm{b}} \qedhere
		\end{equation*}
	\end{proof}

	\begin{osservazione}
		Dal Teorema~\ref{th:stima-criterio-residuo-relativo}, che vale anche per metodi che non siano iterativi stazionari, si evince che il criterio del residuo relativo non offre una misura di accuratezza relativa del metodo se \(\kappa (A) \gg 1\).
	\end{osservazione}

\section{Metodi di discesa}
	
	\noindent Consideriamo una matrice reale \(A\) simmetrica e definita positiva. Si osserva che, se \(x^*\) è la soluzione unica del sistema lineare \(A x = b\), allora è anche il minimo del \emph{funzionale dell'energia}
	\begin{equation}\label{eq:funzionale-energia}
		\begin{array}{rccl}
			\varphi \colon & \R^n & \to & \R \\
			& x & \mapsto & \frac{1}{2} \tra{x} A x + \tra{b} x
		\end{array}
	\end{equation}
	perché
	\begin{equation*}
		\grad{\varphi (x)} = A x - b = \vec{0} \iff A x = b
	\end{equation*}
	Vogliamo, quindi, trovare la soluzione del sistema lineare calcolando il minimo del funzionale \(\varphi\).
	
	Un \emph{metodo di discesa} consiste nel generare una successione
	\begin{equation}\label{eq:metodo-discesa}
		x^{(k + 1)} = x^{(k)} + \alpha_k p^{(k)}
	\end{equation}
	ove \(p^{(k)}\) è una direzione fissata secondo un certo criterio. Siamo interessati a scegliere tali \(p^{(k)}\) di modo che \(\varphi (x^{(k + 1)}) < \varphi (x^{(k)})\) e che il punto \(x^*\), ove si trova il minimo di \(\varphi\), sia calcolato rapidamente.
	
	\begin{definizione}[Metodo del gradiente classico]
		Il \emph{metodo del gradiente classico} è un metodo di discesa come nella \eqref{eq:metodo-discesa} con \(\alpha_k\) e \(p^{(k)}\) scelti in modo da ottenere la massima riduzione del funzionale dell'energia a partire dal punto \(x^{(k)}\).
	\end{definizione}

	Differenziando \(\varphi\), si può mostrare che il metodo del gradiente classico consiste nello scegliere
	\begin{align}
		p^{(k)} &= r^{(k)} &
		\alpha_k &= \frac{\bigl\| r^{(k)} \bigr\|_2^2}{\tra{\qty(r^{(k)})} A r^{(k)}}
	\end{align}
	
	Si può dimostrare che questo metodo è classificabile come metodo di Richardson non stazionario con \(P = \uno_n\) e parametro \(\alpha_k\).
	
	\begin{teorema}\label{th:errore-metodo-gradiente-classico}
		Data una matrice \(A\) simmetrica e definita positiva, sia \(\norm{x}_A = \sqrt{\tra{x} A x}\). Se \(x^{(k)}\) rappresenta la \(k\)-esima iterazione del metodo del gradiente classico nell'approssimazione della soluzione \(x^*\) al sistema \(A x = b\), definiti \(e^{(k)} = x^* - x^{(k)}\) e \(\kappa_2 (A) = \norm{A^{-1}}_2 \, \norm{A}_2\), allora
		\begin{equation}\label{eq:errore-metodo-gradiente-classico}
			\bigl\| e^{(k)} \bigr\|_A \le \qty(\frac{\kappa_2 (A) - 1}{\kappa_2 (A) + 1})^k \bigl\| e^{(0)} \bigr\|_A
		\end{equation}
	\end{teorema}

	\begin{osservazione}
		Il Teorema~\ref{th:errore-metodo-gradiente-classico} mostra che la convergenza del metodo del gradiente classico è lenta se la matrice \(A\) è malcondizionata, perché
		\begin{equation*}
			\kappa_2 (A) = \norm{A^{-1}}_2 \, \norm{A}_2 = \frac{\max \Set{\abs{\lambda_i} : i \in \Set{1, \dots, n}}}{\min \Set{\abs{\lambda_j} : j \in \Set{1, \dots, n}}}
		\end{equation*}
		ove \(\lambda_1, \dots, \lambda_n\) sono gli autovalori di \(A\). Il primo grafico della Figura~\ref{fig:condiz-grad-classico} mostra che la funzione \(\mu_\textup{g} (x) = (x - 1) / (x + 1)\) tende asintoticamente a \(1\); per questo motivo, affinché si abbia la condizione \(\qty[(\kappa_2 (A) - 1) / (\kappa_2 (A) + 1)]^k < \varepsilon\), bisogna scegliere \(k\) mediante la formula
		\begin{equation*}
			k (\kappa_2 (A)) = \left\lceil \frac{\log \varepsilon}{\log \mu_\textup{g} (\kappa_2 (A))} \right\rceil
		\end{equation*}
		illustrata nel secondo grafico della Figura~\ref{fig:condiz-grad-classico} nel caso in cui \(\varepsilon = 10^{-6}\).
		
		\begin{figure}[tpb]
			\centering
			\begin{tikzpicture}
				\begin{groupplot}[group style = {group size = 1 by 2, vertical sep = 2cm}]
					\nextgroupplot[xlabel = \(x\), ylabel = \(\mu_\textup{g} (x)\), xmode = log, domain=1:100000]
					\addplot+[only marks] {(x - 1) / (x + 1)};
					\nextgroupplot[xlabel = \(\kappa_2 (A)\), ylabel = \(k (\kappa_2 (A))\), domain=1:100000, ymode = log]
					\addplot+[only marks, mark options = {color = red, fill = black!20!red}, mark = square*] {ceil(ln(10^(-6)) / ln((x - 1) / (x + 1)))};
				\end{groupplot}
			\end{tikzpicture}
			\caption{Grafici delle funzioni \(\mu_\textup{g} (x)\) e \(k (\kappa_2 (A))\).}\label{fig:condiz-grad-classico}
		\end{figure}
	\end{osservazione}

	\begin{definizione}[Metodo del gradiente coniugato]
		Il \emph{metodo del gradiente coniugato} è un metodo di discesa come nella \eqref{eq:metodo-discesa} con
		\begin{subequations}
		\begin{align}
			\alpha_k &= \frac{\bigl\| r^{(k)} \bigr\|_2^2}{\tra{\qty(p^{(k)})} A p^{(k)}} &
			\beta_k &= \frac{\bigl\| r^{(k)} \bigr\|_2^2}{\bigl\| r^{(k - 1)} \bigr\|_2^2} \label{eq:metodo-grad-coniug-a-b} \\
			p^{(0)} &= r^{(0)} &
			p^{(k)} &= r^{(k)} + \beta_k p^{(k - 1)} \label{eq:metodo-grad-coniug-p}
		\end{align}
		\end{subequations}
	\end{definizione}

	Per come sono definiti nella \eqref{eq:metodo-grad-coniug-p}, i vettori \(p^{(k)}\) e \(p^{(k - 1)}\) sono \(A\)-coniugati, ossia
	\begin{equation}
		\tra{\qty(p^{(k)})} \! A p^{(k)} = 0
	\end{equation}

	\begin{teorema}\label{th:metodo-grad-coniug-niter}
		Se \(A\) è una matrice simmetrica definita positiva di ordine \(n\), allora il metodo del gradiente coniugato è convergente e fornisce in aritmetica esatta la soluzione del sistema lineare \(A x = b\) in al piú \(n\) iterazioni.
	\end{teorema}

	Il Teorema~\ref{th:metodo-grad-coniug-niter} risulta spesso insoddisfacente, perché non è detto che sia possibile eseguire i calcoli in aritmetica esatta e, ad ogni modo, nelle applicazioni \(n\) può anche essere molto grande.
	
	\begin{definizione}[Spazio di Krylov]
		Fissato \(k \in \N^*\), si dice \emph{spazio di Krylov}
		\begin{equation}\label{eq:spazio-krylov}
			\krylov_k = \Braket{A^i r^{(0)} : i \in \Set{0, \dots, k - 1}}
		\end{equation}
	\end{definizione}

	\begin{osservazione}
		È evidente che, denominata \(x^{(k)}\) la \(k\)-esima iterazione del metodo del gradiente coniugato, si ha \(x^{(k)} \in x^{(0)} + \krylov_k\), ove si è posto \(\krylov_0 = \Braket{\varnothing}\).
		
		Mostriamo innanzitutto che \(p^{(k)} \in \Braket{r^{(0)}, \dots, r^{(k)}}\): il caso \(k = 0\) è dovuto al fatto che \(p^{(0)} = r^{(0)}\); se \(k > 0\), allora
		\begin{equation*}
			p^{(k)} = r^{(k)} + \beta_k p^{(k -1)} = r^{(k)} + \sum_{i = 0}^{k - 1} \qty(\prod_{j = i + 1}^k \beta_j) r^{(i)} \in \Braket{r^{(0)}, \dots, r^{(k)}}
		\end{equation*}
		
		Mostriamo ora che \(r^{(k)} \in \krylov_{k + 1}\): si ha \(r^{(0)} \in \krylov_1\) per la \eqref{eq:spazio-krylov}; se \(k > 0\), allora
		\begin{equation*}
			r^{(k)} = b - A x^{(k)} = b - A x^{(k - 1)} - \alpha_{k - 1} A p^{(k - 1)} = r^{(k - 1)} - \alpha_{k - 1} A p^{(k - 1)}
		\end{equation*}
		e, dato che \(r^{(k - 1)} \in \krylov_k \le \krylov_{k + 1}\) per ipotesi induttiva e, in base a quanto visto sopra, \(A p^{(k - 1)} \in \Braket{A r^{(0)}, \dots, A r^{(k - 1)}} \le \krylov_{k + 1}\), si conclude che \(r^{(k)} \in \krylov_{k + 1}\).
		
		Mostriamo ora che \(x^{(k)} \in x^{(0)} + \krylov_k\). Per \(k = 0\), è evidente che \(x^{(0)} \in x^{(0)} + \krylov_0\). Per \(k > 0\), invece, si ha \(x^{(k)} = x^{(k - 1)} + \alpha_{k - 1} p^{(k - 1)}\); per quanto visto sopra e per ipotesi induttiva, esso è un elemento di \(x^{(0)} + \krylov_k\).
	\end{osservazione}

	\begin{teorema}\label{th:minimo-funz-energia-krylov}
		Scelto \(x^{(0)} \in \R^n\), la \(k\)-esima iterazione del metodo del gradiente coniugato minimizza il funzionale energia \(\varphi\) della \eqref{eq:funzionale-energia} nell'insieme \(x^{(0)} + \krylov_k\).
	\end{teorema}

	\begin{osservazione}
		Anche la \(k\)-esima iterazione del metodo del gradiente classico è contenuta in \(x^{(0)} + \krylov_k\); di conseguenza, essa in generale non minimizza \(\varphi\) in tale insieme.
	\end{osservazione}
	
	\begin{osservazione}
		Se \(\dim \krylov_n = n\), allora \(\krylov_n = \R^n\) e, quindi, la soluzione \(x^*\) del sistema lineare \(A x = b\) verifica \(x^* \in x^{(0)} + \krylov_n = \R^n\) ed è punto di minimo del funzionale \(\varphi\). Dato che, però, per il Teorema~\ref{th:minimo-funz-energia-krylov} il punto di minimo di \(\varphi\) in \(x^{(0)} + \krylov_n\) è \(x^{(n)}\), allora \(x^{(n)} = x^*\).
		
		Si può dimostrare che, se \(\dim \krylov_n < n\), allora il metodo coniugato ha ottenuto la soluzione al sistema lineare in meno di \(n\) iterazioni.
	\end{osservazione}

	\begin{teorema}\label{th:errore-metodo-gradiente-coniugato}
		Data una matrice \(A\) simmetrica e definita positiva, sia \(\norm{x}_A = \sqrt{\tra{x} A x}\). Se \(x^{(k)}\) rappresenta la \(k\)-esima iterazione del metodo del gradiente coniugato nell'approssimazione della soluzione \(x^*\) al sistema \(A x = b\), definiti \(e^{(k)} = x^* - x^{(k)}\) e \(\kappa_2 (A) = \norm{A^{-1}}_2 \, \norm{A}_2\), allora
		\begin{equation}
			\bigl\| e^{(k)} \bigr\|_A \le 2 \qty(\frac{\sqrt{\kappa_2 (A)} - 1}{\sqrt{\kappa_2 (A)} + 1})^k \bigl\| e^{(0)} \bigr\|_A
		\end{equation}
	\end{teorema}

	\begin{osservazione}
		Il Teorema~\ref{th:errore-metodo-gradiente-coniugato} mostra che la convergenza del metodo del gradiente classico è lenta se la matrice \(A\) è malcondizionata, perché
		\begin{equation*}
			\kappa_2 (A) = \norm{A^{-1}}_2 \, \norm{A}_2 = \frac{\max \Set{\abs{\lambda_i} : i \in \Set{1, \dots, n}}}{\min \Set{\abs{\lambda_j} : j \in \Set{1, \dots, n}}}
		\end{equation*}
		ove \(\lambda_1, \dots, \lambda_n\) sono gli autovalori di \(A\). Il primo grafico della Figura~\ref{fig:condiz-grad-coniugato} mostra che la funzione \(\mu_\textup{cg} (x) = (\sqrt{x} - 1) / (\sqrt{x} + 1)\) tende asintoticamente a \(1\); per questo motivo, affinché si abbia la condizione \(\qty[(\sqrt{\kappa_2 (A)} - 1) / (\sqrt{\kappa_2 (A)} + 1)]^k < \varepsilon\), bisogna scegliere \(k\) mediante la formula
		\begin{equation*}
			k (\kappa_2 (A)) = \left\lceil \frac{\log(\varepsilon / 2)}{\log \mu_\textup{cg} (\kappa_2 (A))} \right\rceil
		\end{equation*}
		illustrata nel secondo grafico della Figura~\ref{fig:condiz-grad-coniugato} nel caso in cui \(\varepsilon = 10^{-6}\).
		
		\begin{figure}[tpb]
			\centering
			\begin{tikzpicture}
				\begin{groupplot}[group style = {group size = 1 by 2, vertical sep = 2cm}]
					\nextgroupplot[xlabel = \(x\), ylabel = \(\mu_\textup{cg} (x)\), xmode = log, domain=1:100000]
					\addplot+[only marks] {(sqrt(x) - 1) / (sqrt(x) + 1)};
					\nextgroupplot[xlabel = \(\kappa_2 (A)\), ylabel = \(k (\kappa_2 (A))\), domain=1:100000, ymode = log]
					\addplot+[only marks, mark options = {color = red, fill = black!20!red}, mark = square*] {ceil(ln(10^(-6) / 2) / ln((sqrt(x) - 1) / (sqrt(x) + 1)))};
				\end{groupplot}
			\end{tikzpicture}
			\caption{Grafici delle funzioni \(\mu_\textup{cg} (x)\) e \(k (\kappa_2 (A))\).}\label{fig:condiz-grad-coniugato}
		\end{figure}
		
		\begin{figure}[tpb]
			\centering
			
			\begin{tikzpicture}
				\begin{axis}[xlabel = \(\kappa_2 (A)\), ylabel = \(k (\kappa_2 (A))\), only marks, domain = 1:100000, ymode = log, legend entries = {{Gradiente classico}, {Gradiente coniugato}}, legend pos = outer north east]
					\addplot {ceil(ln(10^(-6)) / ln((x - 1) / (x + 1)))};
					\addplot {ceil(ln(10^(-6) / 2) / ln((sqrt(x) - 1) / (sqrt(x) + 1)))};
				\end{axis}
			\end{tikzpicture}
			
			\caption{Confronto tra metodo del gradiente classico e metodo del gradiente coniugato relativamente al numero di iterazioni necessarie per ottenere un errore inferiore a \(10^{-6}\) al variare del condizionamento di \(A\).}\label{fig:confronto-grad}
		\end{figure}
	\end{osservazione}
	
	\chapter{Calcolo di autovalori e autovettori}
	
	\noindent Il calcolo degli autovalori di una data matrice quadrata \(A\) di ordine \(n\) consiste nel trovare gli \(n\) numeri \(\lambda_1, \dots, \lambda_n\) tali che esista \(x \ne \vec{0}\) tale che \(A x = \lambda_i x\) per un certo \(i \in \Set{1, \dots, n}\).
	
	È noto che gli autovalori di \(A\) sono anche gli zeri del polinomio caratteristico \(p_A (X) = \det (A - X \uno_n)\), perciò sono \(n\) numeri complessi, se contati con molteplicità.
	
	Nelle applicazioni talvolta si richiede di calcolare tutti gli autovalori di \(A\), talvolta solo alcuni -- ad esempio solo quelli di massimo modulo, in modo da determinare il raggio spettrale di \(A\).
	
\section{Teoremi di Gerschgorin}
	
	\begin{teorema}[Gerschgorin \textsc{i}]\label{th:gerschgorin-1}
		Data una matrice quadrata \(A\) di ordine \(n\) e di autovalori \(\lambda_1, \dots, \lambda_n\) e definiti per ogni \(i \in \Set{1, \dots, n}\)
		\begin{equation*}
			K_i = \Set{z \in \C : \abs{z - a_{i, i}} \le \sum_{\substack{j = 1 \\ j \ne i}}^n \abs{a_{i, j}}}
		\end{equation*}
		si verifica
		\begin{equation}\label{eq:gerschgorin-1}
			\Set{\lambda_1, \dots, \lambda_n} \subseteq \bigcup_{i = 1}^n K_i
		\end{equation}
	\end{teorema}

	\begin{esempio}\label{eg:gersch-1}
		Consideriamo la matrice
		\begin{equation*}
			A =
			\begin{pmatrix}
				15 & -2 & 2  \\
				1  & 10 & -3 \\
				-2 & 1  & 0
			\end{pmatrix}
		\end{equation*}
		I suoi cerchi di Gerschgorin sono
		\begin{align*}
			K_1 &= \Set{z \in \C : \abs{z - 15} \le 4} \\
			K_2 &= \Set{z \in \C : \abs{z - 10} \le 4} \\
			K_3 &= \Set{z \in \C : \abs{z} \le 3}
		\end{align*}
		Per il Teorema~\ref{th:gerschgorin-1}, gli autovalori di \(A\) appartengono all'unione di questi tre cerchi, come si può vedere nella Figura~\ref{fig:gersch-1}.
		
		\begin{figure}[tpb]
			\centering
			\begin{tikzpicture}[scale=0.4]
				\filldraw[color = red, fill opacity = 0.1] (15,0) circle[radius=4];
				\filldraw[color = red, fill opacity = 0.1] (10,0) circle[radius=4];
				\filldraw[color = red, fill opacity = 0.1] (0,0) circle[radius=3];
				\filldraw (14.1026,0) circle[radius=0.1] node [above right] {\(\lambda_1\)};
				\filldraw (10.3854,0) circle[radius=0.1] node [above] {\(\lambda_2\)};
				\filldraw (0.512085,0) circle[radius=0.1] node [above] {\(\lambda_3\)};
				\draw[->] (-4,0) -- (20,0) node [above] {\(\Re z\)};
				\draw[->] (0,-4) -- (0,4) node [left] {\(\Im z\)};
				\foreach \i in {4, 8, 12, 16} {
					\draw (\i, 0.2) -- (\i, -0.2) node [below] {\i};	
				}
			\end{tikzpicture}
			
			\caption{Rappresentazione dei cerchi di Gerschgorin e degli autovalori relativi alla matrice \(A\) dell'Esempio~\ref{eg:gersch-1}.}\label{fig:gersch-1}
		\end{figure}
	\end{esempio}

	\begin{teorema}[Gerschgorin \textsc{ii}]\label{th:gerschgorin-2}
		Se l'unione \(M_1\) di \(k\) cerchi di Gerschgorin è disgiunta dall'unione \(M_2\) dei rimanenti \(n - k\) cerchi, allora \(M_1\) contiene esattamente \(k\) autovalori di \(A\) contati con molteplicità e \(M_2\) contiene esattamente \(n - k\) autovalori di \(A\) contati con molteplicità.
	\end{teorema}

	Guardando la Figura~\ref{fig:gersch-1}, è evidente che la matrice \(A\) dell'Esempio~\ref{eg:gersch-1} abbia un solo autovalore in \(K_1\) e due autovalori in \(K_2 \cup K_3\).
	
	\begin{definizione}
		Una matrice quadrata \(A\) di ordine \(n \ge 2\) si dice \emph{riducibile} se esistono una matrice di permutazione \(P\) e un intero \(k \in \Set{1, \dots, n - 1}\) tali che
		\begin{equation}
			P \! A \tra{P} =
			\begin{pmatrix}
				A_{1, 1} & A_{1, 2} \\
				\zero    & A_{2, 2}
			\end{pmatrix}
		\end{equation}
		con \(A_{1, 1} \in M_k (\C)\) e \(A_{2, 2} \in M_{n - k} (\C)\).
		
		Una matrice quadrata si dice \emph{irriducibile} se non è riducibile.
	\end{definizione}

	Per verificare che una matrice sia irriducibile, ricordiamo che una matrice quadrata \(A\) è irriducibile se e solo se il suo grafo orientato associato è fortemente connesso, ovvero se e solo se per ogni coppia \((i, j)\) esiste un cammino da \(i\) verso \(j\).\footnote{A partire da una matrice quadrata \(A\) di ordine \(n\) se ne costruisce il grafo orientato associato ponendo \(1, \dots, n\) come nodi e tracciando il lato \((i, j)\) se e solo se \(a_{i, j} \ne 0\).} Nella Figura~\ref{fig:grafo-gersch-1} è rappresentato il grafo orientato associato alla matrice \(A\) dell'Esempio~\ref{eg:gersch-1}: in base ad esso, si può affermare che \(A\) è irriducibile.
	
	\begin{figure}[tpb]
		\centering
		
		\begin{tikzpicture}[scale = 3]
			\node (a) at (0,0) {1};
			\node (b) at (1,0) {2};
			\node (c) at (0.5,0.866) {3};
			
			\draw[->] (a) to[bend left = 20] (b);
			\draw[->] (a) to[bend left = 20] (c);
			\draw[->] (b) to[bend left = 20] (a);
			\draw[->] (b) to[bend left = 20] (c);
			\draw[->] (c) to[bend left = 20] (a);
			\draw[->] (c) to[bend left = 20] (b);
			
			\draw[->] (a.west) arc[start angle = 90, end angle = 370, radius = 1mm];
			\draw[->] (b.east) arc[start angle = 90, end angle = -190, radius = 1mm];
		\end{tikzpicture}
	
		\caption{Grafo orientato associato alla matrice \(A\) dell'Esempio~\ref{eg:gersch-1}.}\label{fig:grafo-gersch-1}
	\end{figure}

	\begin{teorema}[Gerschgorin \textsc{iii}]\label{th:gerschgorin-3}
		Se una matrice quadrata \(A\) di ordine \(n\) è irriducibile e un suo autovalore \(\lambda\) appartiene alla frontiera dell'unione dei cerchi di Gerschgorin di \(A\), allora \(\lambda\) appartiene alla frontiera di ogni cerchio di Gerschgorin di \(A\).
	\end{teorema}

	\begin{esempio}\label{eg:gersch-2}
		Consideriamo la matrice
		\begin{equation*}
			B =
			\begin{pmatrix}
				2  & -1 & 0  & 0  \\
				-1 & 2  & -1 & 0  \\
				0  & -1 & 2  & -1 \\
				0  & 0  & -1 & 2
			\end{pmatrix}
		\end{equation*}
		Essa è irriducibile, come mostra il suo grafo orientato nella Figura~\ref{fig:grafo-gersch-2}. Per il Teorema~\ref{th:gerschgorin-1} tutti gli autovalori di \(B\) appartengono alla palla chiusa di centro \(2\) e raggio \(2\), ma non possono appartenere contemporaneamente alle frontiere di tutti i cerchi di Gerschgorin; per questo motivo, dal Teorema~\ref{th:gerschgorin-3} segue che \(B\) è non singolare.
		
		\begin{figure}[tpb]
			\centering
			
			\begin{tikzpicture}[scale = 3]
				\node (a) at (0,0) {1};
				\node (b) at (1,0) {2};
				\node (c) at (1,1) {3};
				\node (d) at (0,1) {4};
				
				\draw[->] (a) to[bend left = 20] (b);
				\draw[->] (b) to[bend left = 20] (a);
				\draw[->] (b) to[bend left = 20] (c);
				\draw[->] (c) to[bend left = 20] (b);
				\draw[->] (c) to[bend left = 20] (d);
				\draw[->] (d) to[bend left = 20] (c);
				
				\draw[->] (a.north west) arc[start angle = 45, end angle = 315, radius = 1mm];
				\draw[<-] (d.north west) arc[start angle = 45, end angle = 315, radius = 1mm];
				\draw[->] (b.east) arc[start angle = 90, end angle = -180, radius = 1mm];
				\draw[->] (c.east) arc[start angle = -90, end angle = 180, radius = 1mm];
			\end{tikzpicture}
			
			\caption{Grafo orientato associato alla matrice \(B\) dell'Esempio~\ref{eg:gersch-2}.}\label{fig:grafo-gersch-2}
		\end{figure}
	\end{esempio}

	\begin{osservazione}
		Poiché \(\tra{A}\) ha gli stessi autovalori di \(A\), è possibile raffinare il criterio dei cerchi di Gerschgorin affermando che gli autovalori di \(A\) appartengono all'intersezione tra l'unione dei cerchi di Gerschgorin di \(A\) e l'unione dei cerchi di Gerschgorin di \(\tra{A}\).
	\end{osservazione}

\section{Metodo delle potenze}
	
	\noindent Il metodo delle potenze si usa per calcolare l'autovalore di massimo modulo di una matrice quadrata. Nel nostro caso supporremo sempre che la matrice \(A\) sia quadrata, di ordine \(n\), diagonalizzabile e tale che i suoi autovalori \(\lambda_1, \dots, \lambda_n\) verifichino\(\abs{\lambda_1} > \abs{\lambda_2} \ge \dots \ge \abs{\lambda_n}\).
	
	Ricordiamo che una matrice quadrata \(A\) di ordine \(n\) è diagonalizzabile se e solo se ammette \(n\) autovettori linearmente indipendenti. Ricordiamo anche che, se \(A\) ammette \(n\) autovalori distinti, allora è diagonalizzabile; se \(A\) è simmetrica o hermitiana, allora è diagonalizzabile.
	
	\begin{teorema}\label{th:metodo-potenze-converge}
		Data una matrice \(A \in M_n (\C)\) diagonalizzabile e con autovalori \(\lambda_1, \dots, \lambda_n\) tali che \(\abs{\lambda_1} > \abs{\lambda_2} \ge \dots \ge \abs{\lambda_n}\) e scelti \(u_1, \dots, u_n \in \C^n \setminus \Set{\vec{0}}\) tali che \(A u_k = \lambda_k u_k\) per ogni \(k \in \Set{1, \dots, n}\), se il vettore \(y_0 = \sum_{k = 1}^n \alpha_k u_k\) verifica \(\alpha_1 \ne 0\), allora la successione \((y_s)_{s \in \N}\) definita da
		\begin{equation}
			y_{s + 1} = A y_s
		\end{equation}
		per ogni \(s \in \N\) converge per direzione alla direzione di \(u_1\) e il \emph{quoziente di Rayleigh}
		\begin{equation}
			\rayleigh (y_s, A) = \frac{(A y_s, y_s)}{(y_s, y_s)}
		\end{equation}
		converge a \(\lambda_1\).
	\end{teorema}

	\begin{proof}
		Poiché \(A\) è diagonalizzabile, i vettori \(u_1, \dots, u_n\) esistono certamente e, essendo \(n\), formano una base di \(\C^n\). Visto che gli \(u_k\) sono autovettori relativi agli autovalori \(\lambda_k\), si ha
		\begin{align*}
			y_1 &= A y_0 = A \qty(\sum_{k = 1}^n \alpha_k u_k) = \sum_{k = 1}^n \alpha_k A u_k = \sum_{k = 1}^n \alpha_k \lambda_k u_k \\
			y_{s + 1} &= A y_s = A \qty(\sum_{k = 1}^n \alpha_k \lambda_k^s u_k) = \sum_{k = 1}^n \alpha_k \lambda_k^s A u_k = \sum_{k = 1}^n \alpha_k \lambda_k^{s + 1} u_k
		\end{align*}
		Da ciò segue che
		\begin{equation*}
			\frac{y_{s + 1}}{\lambda_1^{s + 1}} = \sum_{k = 1}^n \alpha_k \frac{\lambda_k^{s + 1}}{\lambda_1^{s + 1}} u_k = \alpha_1 u_1 + \sum_{k = 2}^n \alpha_k \qty(\frac{\lambda_k}{\lambda_1})^{s + 1} u_k
		\end{equation*}
		e, dato che \((\lambda_k / \lambda_1)^{s + 1} \xrightarrow{s \to \infty} 0\), si è provato che la direzione di \(y_s / \lambda_1^s\) converge a quella di \(u_1\).
		
		Dalla continuità del prodotto scalare usuale e di \(A\) segue che
		\begin{equation*}
			\begin{split}
				\lim_{s \to \infty} \rayleigh (y_s, A) &= \lim_{s \to \infty} \frac{(A y_s, y_s)}{(y_s, y_s)} = \lim_{s \to \infty} \frac{\qty(A \frac{y_s}{\lambda_1^s}, \frac{y_s}{\lambda_1^s})}{\qty(\frac{y_s}{\lambda_1^s}, \frac{y_s}{\lambda_1^s})} \\
				&= \frac{\qty(\lim_{s \to \infty} A \frac{y_s}{\lambda_1^s}, \lim_{s \to \infty} \frac{y_s}{\lambda_1^s})}{\qty(\lim_{s \to \infty} \frac{y_s}{\lambda_1^s}, \lim_{s \to \infty} \frac{y_s}{\lambda_1^s})} \\
				&= \frac{\qty(\alpha_1 A u_1, \alpha_1 u_1)}{\qty(\alpha_1 u_1, \alpha_1 u_1)} = \frac{(A u_1, u_1)}{(u_1, u_1)} = \frac{\cancel{(u_1, u_1)}}{\cancel{(u_1, u_1)}} \lambda_1 \\
				&= \lambda_1 \qedhere
			\end{split}
		\end{equation*}
	\end{proof}

	\begin{osservazione}
		Il Teorema~\ref{th:metodo-potenze-converge} mostra che il metodo delle potenze converge anche nel caso in cui \(\lambda_1 = \dots = \lambda_r\) per \(r > 1\), ma non quando esistono piú autovalori distinti di modulo massimo.
	\end{osservazione}

	\begin{osservazione}
		Per ovviare a problemi di \inglese{underflow} e \inglese{overflow}, in aritmetica di macchina si preferisce normalizzare ad ogni iterazione il vettore \(y_k\) ottenuto dal metodo delle potenze. Scelto, dunque, un vettore unitario \(t_0 = \sum_{k = 1}^n \alpha_k u_k\) con \(\alpha_1 \ne 0\), il metodo estrae le successioni \((t_k)_{k \in \N}\) e \((I_k)_{k \in \N}\) come segue:
		\begin{align}
			\gamma_k &= A t_{k - 1} &
			t_k &= \frac{\gamma_k}{\norm{\gamma_k}_2} &
			I_k &= \rayleigh (t_k, A)
		\end{align}
	\end{osservazione}

\section{Metodo delle potenze inverse}
	
	\noindent Il \emph{metodo delle potenze inverse} è una variante del metodo delle potenze usata per calcolare l'autovettore piú piccolo in modulo di una matrice quadrata \(A\) di ordine \(n\) e diagonalizzabile. Nel nostro caso, supporremo che gli autovalori \(\lambda_1, \dots, \lambda_n\) verifichino \(\abs{\lambda_1} \ge \dots \ge \abs{\lambda_{n - 1}} > \abs{\lambda_n} > 0\); in questo modo, possiamo ottenere \(\lambda_n\) applicando il metodo delle potenze alla matrice \(A^{-1}\).
	
	\begin{lemma}\label{lem:autovalori-inversa}
		Se una matrice quadrata \(A\) ha autovalori \(\lambda_1, \dots, \lambda_n\) tali che \(\abs{\lambda_1} \ge \dots \ge \abs{\lambda_n} > 0\) ed esistono \(u_1, \dots, u_n \in \C^n \setminus \Set{\vec{0}}\) tali che \(A u_k = \lambda_k u_k\) per ogni \(k \in \Set{1, \dots, n}\), allora \(A^{-1}\) ha come autovalori \(\xi_1, \dots, \xi_n\) che verificano \(\abs{\xi_1} \ge \dots \ge \abs{\xi_n} > 0\) e \(\xi_k = 1 / \lambda_{n - k + 1}\) per ogni \(k \in \Set{1, \dots, n}\) e i vettori \(v_k = u_{n - k + 1}\) sono tali che \(A^{-1} v_k = \xi_k v_k\).
	\end{lemma}

	Applicando il metodo delle potenze alla matrice \(A^{-1}\), supposto che \(A\) sia diagonalizzabile e che i suoi autovettori verifichino \(\abs{\lambda_1} \ge \dots \ge \abs{\lambda_{n - 1}} > \abs{\lambda_n} > 0\), si parte da un vettore unitario \(t_0 = \sum_{k = 0}^n \alpha_k u_k\) con \(\alpha_1 \ne 0\) e si ottiene la successione \((t_s)_{s \in \N}\) definita da
	\begin{align}
		\gamma_s &= A^{-1} t_{s - 1} &
		t_s &= \frac{\gamma_s}{\norm{\gamma_s}_2}
	\end{align}
	e convergente per direzione alla direzione di \(v_1 = u_n\). La successione dei quozienti di Rayleigh, invece, verifica
	\begin{equation*}
		\rayleigh (t_s, A^{-1}) = \frac{\qty(A^{-1} t_s, t_s)}{(t_s, t_s)} = (t_s, \gamma_{s + 1}) \to \xi_1 = \frac{1}{\lambda_n}
	\end{equation*}
	ovvero \(\rayleigh (t_s, A) \to \lambda_n\).
	
	In generale, fissato \(\mu \in \C\), è possibile calcolare con un algoritmo simile l'autovalore \(\lambda\) di \(A\) piú vicino a \(\mu\), qualora sia unico. Osservando, infatti, che
	\begin{equation*}
		A u = \lambda u \iff (A - \mu \uno_n) u = \lambda u - \mu u = \underbrace{(\lambda - \mu)}_{\sigma} u
	\end{equation*}
	segue che, se \(\sigma\) è un autovalore di minimo modulo di \(A - \mu \uno_n\), allora \(\lambda = \sigma + \mu\) è uno degli autovalori di \(A\) di minima distanza da \(\mu\) e che, se \(u\) è autovettore per \(\sigma\) relativamente ad \(A - \mu \uno_n\), allora lo è anche per \(\lambda\) relativamente ad \(A\). Qualora sia possibile applicare il metodo delle potenze inverse alla matrice \(A - \mu \uno_n\), scelto un vettore unitario \(t_0\), si definisce il metodo delle potenze inverse con \inglese{shift} con
	\begin{align}
		(A - \mu \uno_n) \gamma_k &= t_{k - 1} &
		t_k &= \frac{\gamma_k}{\norm{\gamma_k}_2} &
		\sigma_k &= \her{t_k} A t_k
	\end{align}

\section{Metodo QR}
	
	\noindent Il metodo \textsc{qr} si usa per calcolare efficientemente tutti gli autovalori di una matrice quadrata \(A\).
	
	\begin{teorema}[Fattorizzazione \textsc{qr}]\label{th:fattoriz-qr}
		Se \(A\) è una matrice quadrata di ordine \(n\), allora esistono una matrice \(Q\) unitaria e una matrice \(R\) triangolare superiore tali che \(A = Q R\).
	\end{teorema}
	
	Il Teorema~\ref{th:fattoriz-qr} garantisce l'esistenza di una fattorizzazione \textsc{qr} per \emph{qualunque} matrice quadrata. In generale, però, tale fattorizzazione non è unica, bensí determinata a meno di una matrice di fase. Qualora, invece, \(A\) sia non singolare, la fattorizzazione è unica se si richiede che i coefficienti diagonali di \(R\) siano positivi.
	
	Ricordiamo che, se \(H\) è simile a \(K\), cioè se esiste \(S\) invertibile tale che \(H = S^{-1} K S\), allora \(H\) e \(K\) hanno gli stessi autovalori -- e tale relazione è transitiva.
	
	\begin{lemma}\label{lem:a0-a1-simili}
		Data la fattorizzazione \textsc{qr} \(A_0 = Q_0 R_0\) e la matrice \(A_1 = R_0 Q_0\), le matrici \(A_0\) e \(A_1\) sono simili.
	\end{lemma}
	
	\begin{proof}
		È sufficiente notare che
		\begin{equation*}
			Q_0 A_1 \tra{Q_0} = Q_0 R_0 Q_0 \tra{Q_0} = A_0 \qedhere
		\end{equation*}
	\end{proof}

	Sulla base del Lemma~\ref{lem:a0-a1-simili} definiamo il metodo \textsc{qr}. Posto \(A_0 = A\), si calcolano le iterate \(A_k\) secondo la regola
	\begin{equation}
		A_k = Q_k R_k \implies A_{k + 1} = R_k Q_k
	\end{equation}
	ove ad ogni iterazione si calcola la fattorizzazione \textsc{qr} della matrice \(A_k\). Tutte le matrici \(A_k\) sono simili alla matrice \(A\) e, quindi, hanno gli stessi autovalori di \(A\).
	
	\begin{teorema}[Convergenza del metodo \textsc{qr}]\label{th:metodo-qr-converge}
		Se una matrice \(A \in M_n (\R)\) ha autovalori tutti distinti in modulo, ovvero \(\abs{\lambda_1} > \dots > \abs{\lambda_n}\), allora il metodo \textsc{qr} converge a una matrice triangolare superiore \(A_\infty\) che ha come entrata \((k, k)\) l'autovalore \(\lambda_k\).
		
		Se indichiamo con \(a_{i, j}^{(k)}\) l'entrata \((i, j)\) di \(A_k\) e \(\lambda_{i - 1} \ne 0\), allora per ogni \(i \in \Set{2, \dots, n}\)
		\begin{equation}\label{eq:metodo-qr-converge}
			\abs{a_{i, i - 1}^{(k)}} = \order{\frac{\abs{\lambda_i}}{\abs{\lambda_{i - 1}}}}^k
		\end{equation}
		
		Se \(A\) non ha gli autovalori tutti distinti in modulo, allora il metodo \textsc{qr} converge a una matrice triangolare a blocchi.
	\end{teorema}

	Nelle implementazioni del metodo \textsc{qr} si calcola, ricorrendo a un algoritmo messo a punto da Householder, una matrice di Hessenberg
	\begin{equation*}
		T =
		\begin{pmatrix}
			a_{1, 1} & \cdots & \cdots & \cdots       & a_{1, n} \\
			a_{2, 1} & \ddots &        &              & \vdots   \\
			0        & \ddots & \ddots &              & \vdots   \\
			\vdots   & \ddots & \ddots & \ddots       & \vdots   \\
			0        & \cdots & 0      & a_{n, n - 1} & a_{n, n}
		\end{pmatrix}
	\end{equation*}
	simile ad \(A\) e si applica a \(T\) il metodo \textsc{qr} descritto sopra. Si può dimostrare che, se \(A\) è simmetrica, allora \(T\) è tridiagonale simmetrica. È anche possibile mostrare che, se \(A_0\) è di Hessenberg, allora tutte le iterate \(A_k\) sono di Hessenberg e similmente per il caso in cui \(A_0\) sia tridiagonale.
	
	Se \(A\) è una matrice di Hessenberg, allora il metodo \textsc{qr} converge a una matrice triangolare a blocchi simile ad \(A\) e tale che gli autovalori di ogni blocco diagonale siano tutti uguali in modulo. Anche in questo caso vale la \eqref{eq:metodo-qr-converge} se \(\abs{\lambda_1} > \dots > \abs{\lambda_n}\).
	
	\begin{osservazione}
		L'algoritmo di Householder per trovare la matrice di Hessenberg simile ad \(A \in M_n (\R)\) richiede all'incirca \(5 n^3 / 3\) operazioni. Un algoritmo alternativo, messo a punto da Givens, ne richiede \(10 n^3 / 3\).
		
		Il metodo \textsc{qr} applicato a una matrice \(A \in M_n (\R)\) in forma di Hessenberg superiore richiede circa \(2 n^2\) moltiplicazioni ad ogni iterazione. In generale, invece, la fattorizzazione \textsc{qr} di una matrice generica mediante un algoritmo dovuto a Householder richiede \(\order{2 n^3 / 3}\) moltiplicazioni.
	\end{osservazione}

	\begin{nota}
		Talvolta risulta conveniente applicare un metodo \textsc{qr} con \inglese{shift} a una matrice di Hessenberg o tridiagonale \(A \in M_n (\R)\); esso si definisce con
		\begin{align*}
			A_0 &= A &
			A_k - \lambda_k \uno_n &= Q_k R_k &
			A_{k + 1} &= R_k Q_k + \lambda_k \uno_n
		\end{align*}
		ove di solito \(\lambda_k = (A_k)_{n, n}\).
	\end{nota}
	
\end{document}